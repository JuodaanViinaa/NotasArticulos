\documentclass[a4paper,12pt]{article}
\usepackage[utf8]{inputenc}
\usepackage[T1]{fontenc}
\usepackage[spanish]{babel}
\usepackage{csquotes}
\usepackage{anysize}
\marginsize{25mm}{25mm}{25mm}{25mm}

\title{Bayesian Data Analysis}
\author{Andrew Gelman, John B. Carlin, Hal S . Stern,\\David B. Dunson, Aki Vehrari, Donald B. Rubin}
\date{2014}

\begin{document}

{\scshape\bfseries \maketitle}

La inferencia Bayesiana consiste en el ajuste de un modelo de probabilidad a un conjunto de datos, y el resumen del resultado mediante una distribución de probabilidad de los parámetros del modelo.

La característica esencial de los métodos Bayesianos es su uso explícito de las probabilidades para cuantificar la incertidumbre sobre las inferencias basadas en el análisis estadístico de los datos. Los pasos del análisis Bayesiano pueden resumirse en:
\begin{enumerate}
	\item Establecer un {\itshape full probability model}: una distribución conjunta de probabilidades para todas las cantidades observables y no observables de un problema.
	\item Condicionar a los datos observados: calcular e interpretar la probabilidad posterior adecuada, es decir, la distribución de probabilidad condicionada de las cantidades no observadas de interés, dados los datos observados.
	\item Evaluar el ajuste del modelo y las implicaciones de la distribución posterior resultante. Como resultado de este paso se pueden alterar o expandir los modelos y repetir los pasos.
\end{enumerate}

El primer paso es difícil: ¿de dónde vienen los modelos? ¿Cómo se construyen las especificaciones de probabilidad apropiadas?

{\scshape\bfseries General notation for statistical inference}

La inferencia estadística buscar sacar conclusiones de datos numéricos acerca de cantidades que no han sido observadas. Se distinguen dos clases de {\itshape estimands} (cantidades no observadas para las cuales se hacen inferencias estadísticas): (1) cantidades potencialmente observables, como las observaciones futuras, y (2) cantidades que no son directamente observables, es decir, los parámetros que gobiernan el proceso hipotético que lleva a los datos observados (como los coeficientes de regresión). 

{\itshape Parameters, data, and predictions}

$\theta$ denota cantidades vectoriales inobservables o parámetros poblacionales de interés; $y$ denota los datos observados, y $\tilde{y}$ denota las cantidades desconocidas pero potencialmente observables 

\end{document}
