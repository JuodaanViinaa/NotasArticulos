\documentclass[a4paper,12pt]{article}
\usepackage[utf8]{inputenc}
\usepackage[T1]{fontenc}
\usepackage[spanish]{babel}
\usepackage{csquotes}
\usepackage{anysize}
\usepackage{graphicx}
\marginsize{25mm}{25mm}{25mm}{25mm}

\title{Optimal foraging theory: Prey density and travel requirements in Rattus novegicus}
\author{Roger L. Mellgren \and Linda Misasi \and Steven W. Brown}
\date{1984}

\begin{document}
{\scshape\bfseries \maketitle}

Psicólogos, ecólogos conductuales y etólogos se interesan por la conducta de forrajeo por al menos dos razones: (1) la crítica a la investigación de laboratorio (que indica que ignora las predisposiciones de los animales y puede distorsionar la verdadera naturaleza de los procesos de los sujetos), y (2) la emergencia de una perspectiva teórica conocida como teoría de forrajeo óptimo (la idea de que los individuos maximizan su adecuación inclusiva). Se maximiza para incrementar el éxito reproductivo bajo ciertas restricciones.

Se busca integrar el conocimiento de la forma en que contingencias ambientales afectan el forrajeo.

Las causas finales deben tener mecanismos proximales. El enfoque en estos mecanismos ha llevado a teoría detalladas de aprendizaje asociativo. El forrajeo óptimo es, en cambio, una teoría de causas finales basada en principios biológicos.
Este experimento busca evaluar los mecanismos que llevan a la conducta de forrajeo óptimo usando la base de datos de la psicología animal, la perspectiva teórica de la ecología conductual, u los métodos de esas áreas más la etología.

En este experimento se manipularon los requisitos de viaje colocando parches de comida en distintas alturas de varas verticales a los que se accedía escalando una escalera de clavos. La cantidad de comida enterrada en cada parche se podía manipular también.

Según la teoría de forrajeo óptimo, cuando se incrementa el tiempo de viaje entre parches, la utilización debería incrementar también, lo que se deriva de un análisis costo-beneficio en el que según incrementa el costo de viajar, el beneficio de permanecer incrementa también.

El teorema de valor marginal lleva a una segunda predicción del patrón de forrajeo: un forrajeador debería utilizar un parche dado hasta que la tasa de recompensa de permanecer forrajeando en ese parche sea igual a la tasa promedio de recompensa del ambiente: el valor marginal del hábitat. Una versión simplificada del teorema indica que el tiempo entre la última captura de una presa en el parche y el momento en que el forrajeador abandona debería ser constante en todos los tipos de parche (el {\itshape giving-up time}). 

Este experimento no permitió medir los tiempos precisos entre capturas de presa, pero se tiene una medida derivable del teorema de valor marginal: el número de presas que quedan en el parche después de que el sujeto abandona un parche debería ser constante entre tipos de parche. La tasa de recompensa de un parche caerá al ``valor marginal'' cuando el parche haya sido depletado hasta un nivel particular. Esta ``constante de sobra'' ({\itshape leftover constant}) debería ser mayor para parches que son de fácil acceso, y disminuir para parches difíciles dado que el tiempo de viaje debe ser incluido cuando se calcula la tasa de recompensa del hábitat.

{\scshape\bfseries Method}

El experimento se realizó en una habitación en la cual se colocaron 9 tablas de madera con clavos insertados a manera de escaleras. Las tablas tenían agujeros en los cuales se podía asegurar una caja de vinilo que se utilizaba como contenedor para los parches. Cada parche tenía aproximadamente 4cm de arena. Cuatro parches estaban colocados a 55cm de cada esquina, y los otros cuatro estaban centrados en las paredes. El experimentador observaba desde una silla elevada en el centro de la habitación.

Se utilizaron dos ratas que fueron inicialmente entrenadas para buscar comida en la arena. Las sesiones duraban 11.5 horas. Mientras una rata corría, la otra estaba en su casita solo con acceso a agua. Al acabar las sesiones se medía la cantidad comida y bebida, se reaprovisionaba, y se cambiaba de rata. Los parches se encontraban en condiciones sucesivas a 1, 2, 3 y 4 pies del piso. En una última condición se colocaron los parches a 4 pies de altura y las sesiones se redujeron a 1 hora. Había un total de 20 pellets de comida en cada parche.

En una fase posterior se hicieron sesiones de 1 hora cada 12 horas, y la cantidad de presas pasó a ser de 6 a 20 en incrementos de 2.

En las fases 2-5 la altura de los parches permaneció constante en 4 pies, y la cantidad de comida en un parche particular permaneció constante entre sesiones de esa fase o varió entre sesiones. En las fases 2 y 4 se usó localización variable, y cada una de las posibles cantidades fue dada una o dos veces en cada localización. En 3, 5, y 6 la localización fue constante, por lo que un número particular de presa se colocó en cada parche cada día. La localización de los números de presas fue aleatoria. En la fase 6la altura de los parches se varió moviéndolos arriba o abajo para una sesión en la mañana y uno en la tarde. El orden de las alturas entre sesiones fue 4, 1, 7, 1, 7, 4, 7, 1, 4, 4, 1, 7.

Se registró el orden de visita a los parches, tiempo de residencia, tiempo de viaje y visitas a las botella de agua. Al final de las sesiones la arena era revuelta para evitar artefactos derivados del olor.

{\scshape\bfseries Results}

La eficiencia del orden de visita se puede evaluar con dos estándares: un sujeto con memoria perfecta no repetiría parches en los primeros 8 parches visitados. Uno sin memoria, eligiendo aleatoriamente, visitaría un parche previamente visitado en la visita $k$ con probabilidad
\begin{equation}
	P_{k}(repeat) =
	1 - \frac{
		\left(\begin{array}{l}
			N\\
			k
		\end{array}\right)
	}{
	N^{k}	
	},
\end{equation}
donde $N$ es el número de parches.

Según el teorema de valor marginal, un parche se debe usar hasta que alcanza el valor marginal del ambiente. El valor marginal depende de la comida disponible en el hábitat y de los costos de viaje (energía y riesgo). Las ratas experimentan riesgo de depredación al pasar por áreas descubiertas, y riesgo de caerse al subir y bajar las escaleras. El riesgo no tiene una moneda de cambio independiente actualmente. 

Aun sin moneda de cambio explícita se pueden probar las predicciones de la teoría de forrajeo óptimo al medir un correlato del {\itshape giving-up time}: la cantidad de comida sobrante en cada parche, que debería ser constante entre parches. Esta constante se estima mediante la fórmula
$$
\frac{
	\mbox{Presas totales disponibles por {\itshape bout} - Presas promedio obtenidas por {\itshape bout}}
}{
	\mbox{número de parches}
}
$$
donde el promedio se calcula entre todos los parches del hábitat y todos los {\itshape bouts} de forrajeo bajo una circunstancia ambiental particular. Sustraer la constante de sobra de la cantidad disponible al inicio resulta en la cantidad óptima a comer en parches de densidades variables.

{\scshape Phase 1.} En la fase 1 la cantidad máxima de comida obtenible era de 160 unidades. El sujeto uno promedió 150; y el 2, 94. La altura de los parches incrementó progresivamente. Al reducir el tiempo de forrajeo a 1 hora disminuyó la comida consumida por sesión a 60 y 40. En algunas sesiones los sujetos solo utilizaban un parche.

{\scshape Phase 2.} En la fase dos los sujetos encontraron por primera vez una distribución desigual entre parches. Para controlar efectos de localización se colocó cada cantidad dos veces en cada posición. Se ajustó el uso de las presas en función de la densidad a las predicciones de la teoría de forrajeo óptimo. Las predicciones dada por la resta de la constante de sobra a la comida disponible predijeron bien los datos de ambos sujetos, por lo que el modelo de la constante de sobra es apoyado por los datos. En esta fase ambas ratas consumieron en promedio 65 unidades de comida y visitaron siempre más de un parche. Los tiempos de residencia incrementaron en función de la densidad de la presa, pero eran más variables que la cantidad consumida, quizá porque el tiempo incluye otras actividades distintas del forrajeo, como el acicalamiento. 

Los caminos de búsqueda eran eficientes. La probabilidad de una visita repetida era muy baja, muy cercana a la ``memoria perfecta''.

{\scshape Phases 3, 4, and 5.} La altura se mantuvo en 4 pies. En 3 y 5 la localización de las densidades de comida fue constante; en 4 se varió sistemáticamente. Se encontró que los sujetos tendieron a bajo-utilizar los parches de alta densidad, y a sobre-utilizar los de baja densidad.

Se comparó el modelo de la constante de sobra con los datos y se obtuvo un mal ajuste en las fases 3 y 5, y uno bueno en 4. Por lo tanto, las desviaciones de las predicciones son mayores cuando una cantidad particular de comida permaneció en la misma localización entre sesiones de forrajeo.

La eficiencia fue excelente con baja probabilidad de repetir parches.

{\scshape Phase 6.} Según la teoría de forrajeo óptimo, incrementar el requisito de viaje entre parches debería incrementar la utilización de parches. Esta predicción se cumplió: la cantidad consumida por parche incrementó con el costo de viaje. Además, el ajuste del modelo de constante de sobra incrementó cuando aumentó el costo de viaje. El modelo fue rechazado cuando los parches estaban a 1 pie de altura, a 4 pies solo se rechazó para un sujeto, y a 7 pies se aceptó en ambos. Por lo tanto, parece ser que incrementar la dificultad de viajar de un parche a otro facilita la conducta óptima.

Dado que incrementa la utilización de los parches con costos de viaje incrementados, la tasa de recompensa global (comida por unidad de tiempo) debería disminuir.

Dado que al enfrentarse a una altura novedosa los animales se tomaban mucho tiempo en comenzar a explorar, se decidió tomar como tiempo el promedio de los tres viajes más largos entre parches de cada sesión en cada altura, suponiendo que se visitasen al menos 5 parches en la sesión.

El promedio de parches visitados por día disminuyó con la altura incrementada, y la eficiencia de viajes nuevamente fue muy alta.

{\scshape\bfseries Discussion}

La teoría de forrajeo óptimo tuvo un excelente ajuste con los datos. Las desviaciones del modelo de constante de sobra y de forrajeo óptimo tendieron a ser sistemáticas. Los parches de baja densidad se sobre-utilizaron y los de alta densidad se bajo-utilizaron. Las desviaciones pueden tener varias explicaciones.

Para lidiar con la variabilidad en los ambientes, se ha dicho que los forrajeadores deben muestrear. Una implicación sería que los sujetos deberían persistir en parches con una recompensa menos que la óptima, lo que les permitirá medir con precisión la disponibilidad de comida. Por lo tanto, la sobre-utilización de parches pobres puede ser un proceso de muestreo.

Concordando con esta idea de muestreo, el orden de las visitas a los parches fue el mismo cuando la localización de las densidades varió entre sesiones y cuando no varió.

El muestreo parece manifestarse de dos formas: el forrajeador muestrea dentro de un parche pobre para asegurarse de que no es uno rico que solo temporalmente tiene una tasa de recompensa baja. También muestrea entre parches, visitando parches que consistentemente han sido pobres durante días. Esto dará una ventaja al forrajeador si la abundancia incrementa súbitamente en algún parche.

La bajo-utilización de parches ricos podría deberse a que juzgan mal el paso del tiempo y creen ya haber llegado a su tiempo umbral para abandonar, aunque no es claro por qué debería correr más rápido su reloj.

Para ajustarse a las predicciones de forrajeo óptimo, una rata tendría que poder discriminar el tiempo que pasa forrajeando del tiempo que pasa en otras actividades no-forrajeo, pero parece improbable que lleven dos relojes distintos. Así, parches con mucha comida (en los que se pasa mucho tiempo y por lo tanto es más probable que ocurran actividades no-forrajeo) el tiempo total incluirá más actividades no-forrajeo, lo que tendrá como efecto sesgar al reloj a una lectura muy grande. Así, el forrajeador subestima la tasa de recompensa y tiende a abandonar más temprano de lo que predice la teoría de optimalidad.

La bajo-utilización es inconsistente con un aspecto de la psicología conductual. Programas más ricos suelen llevar a más persistencia en las respuestas que programas pobres. Aun así, la persistencia en los parches más pobres es menor de la esperada.

La habilidad espacial de las ratas es muy refinada. Su estrategia se ha llamado {\itshape win-shift strategy}. El largo tiempo de sesión es mayor que el de experimentos típicos de laberintos radiales. En este caso no está claro lo que representa ``ganar'' en {\itshape win-shift}. ¿Encontrar un pellet?

El incremento en los tiempos de viaje incrementó la utilización de parches, lo que fue medido mediante los pellets restantes en cada parche visitado. La constante de sobra decrementó sistemáticamente con incrementos en la altura de los parches.

Un segundo hallazgo no predicho por la teoría de forrajeo óptimo pero esperado con base en el supuesto de que la optimalidad no es automática sino una función de una predisposición genética y restricciones experimentales, fue que el ajuste cuantitativo de la teoría de forrajeo óptimo a la utilización de parches incrementó con incrementos en los costos de viaje. Previamente se encontró que el uso óptimo era una función de factores ambientales que llevaban a incrementos en la sensibilidad a la cantidad total de comida en el ambiente. Entonces, la operación de los mecanismos proximales en la conducta relacionada con la adecuación debe ser una parte integral de la teoría de forrajeo.


\end{document}
