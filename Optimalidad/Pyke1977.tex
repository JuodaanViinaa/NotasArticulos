\documentclass[a4paper,12pt]{article}
\usepackage[utf8]{inputenc}
\usepackage[T1]{fontenc}
\usepackage[spanish]{babel}
\usepackage{csquotes}
\usepackage{anysize}
\usepackage{graphicx}
\marginsize{25mm}{25mm}{25mm}{25mm}

\title{Optimal Foraging: a Selective Review of Theory and Tests}
\author{G. H. Pyke \and H. R. Pulliam \and E. L. Charnov}
\date{1977}

\begin{document}
{\scshape\bfseries \maketitle}

La literatura de la teoría de forrajeo óptimo intenta predecir aspectos del forrajeo de los animales. Se asume que la adecuación asociada al forrajeo ha sido maximizada por la selección natural (Darwiniana) bajo ciertas restricciones. El argumento es:

\begin{itemize}
	\item La conducta de forrajeo muestra variación heredable, e implica variación en la contribución a generaciones futuras.
	\item Hay un rango de posibles conductas de forrajeo. Es decir, hay limitaciones en los sistemas.
	\item La selección natural favorece a quienes contribuyen más a las siguientes generaciones, por lo que con el tiempo la conducta de forrajeo promedio de las poblaciones cambiará dentro del rango de conductas posibles para llegar a la máxima adecuación. Además, las demandas del ambiente cambian más lentamente que la conducta de las poblaciones, por lo que la conducta promedio de una población debería estar cerca de lo óptimo (sujeto a sus restricciones).
\end{itemize}

Para usar esto como base para formar hipótesis se debe determinar la relación entre la adecuación y las variaciones en la conducta de forrajeo. Por ejemplo, la adecuación puede ser una función incremental de la tasa neta de ganancia energética. En ese caso la conducta tendría que maximizar esta moneda de cambio.

Siempre la conducta hipotética de forrajeo va a maximizar o minimizar algo. El proceso propuesto por Schoener para encontrar conducta óptima es: (1) elegir una moneda de cambio, (2) elegir las funciones apropiadas costo-beneficio ({\itshape i.e.,} establecer restricciones), y (3) resolver para el óptimo. La moneda es la energía, la función costo-beneficio es la relación entre la diera del animal y su tasa de ingesta de energía, y el óptima ser obtendrá determinando la dieta que maximiza la ingesta de energía bajo las restricciones. No hay fórmula para determinar la moneda de cambio adecuada, por lo que la labor del investigador es conocer la biología del animal para proponer una adecuada.

Lo más común es tomar la tasa neta de ingesta energética como moneda de cambio a maximizar. ¿Por qué sería esto adecuado? Si un animal tiene un requisito energético fijo, no gana nada por consumir más después de satisfacer el requisito, y necesita tiempo para otras actividades, entonces su adecuación estaría en minimizar el tiempo para obtener la energía necesaria. La tasa de ingesta mientras el animal forrajea es lo que se maximizaría. Animales así son conocidos como {\itshape time minimizers}. Si los animales tienen un tiempo limitado para alimentarse, la adecuación incrementa con la energía consumida. Animales así se conocen como {\itshape energy maximizers}. En ambos casos se maximiza la tasa de ingesta mientras el animal forrajea. La categoría en la que caiga un animal dependerá de cómo los factores de su ambiente operan para producir conflicto con el incremento de la ingesta energética. Para decidir qué factores (depredadores, competencia, etc.) deben incluirse en el análisis será necesario conocimiento exhaustivo de la biología de los animales. Pero la energía será casi siempre incluida en el análisis.

Se ha argumentado que la moneda de cambio más realista es la tasa neta de ingesta energética y que otros factores no son importantes.

Para elegir la moneda de cambio debe considerarse la escala temporal en la que se realiza la optimización. La conducta que maximiza la ingesta mensualmente puede ser distinta de la que maximiza la ingesta diaria o por hora. Dado que la adecuación es medida a lo largo de toda la vida, esta es la escala temporal relevante. Pero si la conducta de forrajeo en cierto momento no afecta la conducta de forrajeo óptimo en un momento posterior, entonces las predicciones de optimización son independientes de la escala temporal y cualquier escala valdrá.

Hay casos en que la conducta actual {\itshape sí} afecta a la conducta óptima futura, y se pueden dividir en tres clases: (1) conductas que ``comprometen'' al animal durante cierto tiempo (por ejemplo, la colocación de un nido compromete a buscar comida en las cercanías). Así, la escala temporal para la conducta de anidar es el período de anidación, pero la escala para la búsqueda de comida dentro de ese período puede ser mucho menor.

Cuando un animal visita un parche se compromete a pasar cierto tiempo dentro de él. El animal, se hipotetiza, toma cuatro decisiones interdependientes entre sí: qué tipos de parche visitar, cuánto tiempo permanecer, qué tipo de comida comer en cada tipo de parche, y qué camino de forrajeo emplear en cada tipo de parche. Estas cuatro decisiones dan lugar a cuatro categorías en que se puede colocar a la mayoría de literatura de forrajeo.

(2) La segunda clase son conductas que alteran las condiciones que afectarán al animal en el futuro, ya sea directamente o modificando el ambiente. Por ejemplo, explotar un parche altera la disponibilidad futura de la comida dentro de él. El efecto del animal en el parche será más evidente si tiene algún grado de exclusividad en él. Si tiene exclusividad puede ``manejar'' sus recursos para una ganancia sostenida, en lugar de maximizar su ingesta inicial al costo de obtener menores ganancias después.

(3) La tercera clase consiste en conductas que dan al animal información sobre el cambio en la distribución de los recursos. Si la distribución tiende a cambiar con el tiempo, es beneficioso para el animal explorar, aunque eso afecte la eficiencia inmediata del forrajeo.

¿Cuál es la función de estos análisis? Se ha intentado utilizar la teoría de forrajeo como predictora de fenómenos ecológicos distintos del forrajeo per se. Por ejemplo, la {\itshape compression hypothesis} predice los cambios en la elección de los consumidores en función del incremento del número de especies que compiten.

Se revisará la teoría y evidencia de las cuatro categorías de forrajeo mencionadas antes, es decir, dieta óptima, elección de parches, alocación de tiempo a los parches, y camino de forrajeo. Estos fenómenos forman una teoría microecológica que define los eventos a pequeña escala que determinan la teoría macroecológica de eventos mayores, como la competencia y la depredación.

{\scshape\bfseries Optimal diet}

Los artículos y sus resultados son similares. Un resultado muy general es que la dieta que maximiza la tasa neta de ingesta de energía o de masa ({\itshape i.e.,} la dieta óptima) se obtiene de esta forma:
\begin{enumerate}
	\item Cada tipo de comida tiene un valor asociado en calorías o en peso y un tiempo de manejo. Se jerarquizan según la razón entre su valor y su tiempo de manejo.
	\item Se usa tiempo además para buscar los items de comida. Este tiempo y el de manejo son excluyentes, y en conjunto forman todo el tiempo disponible para el forrajeo. Tomando en cuenta ambos tiempos y el valor de la comida se puede obtener la tasa neta de ganancia para toda dieta posible.
	\item La dieta óptima es determinada comenzando con el mayor volumen de las razones de valor a tiempo de manejo, y después añadiendo tipos de comida a la dieta de acuerdo con su orden jerárquico. Esto continúa mientras la razón de valor a tiempo de manejo para cada adición a la dieta sea mayor que la tasa neta de ingesta de comida de la dieta sin la adición. Cuando esta desigualdad se revierte se ha encontrado la dieta óptima.
\end{enumerate}

Esta dieta óptima tiene tres propiedades: (1) que un tipo de comida sea comido es independiente de su propia abundancia, y depende solo de la abundancia absoluta de tipos de comida de rango más alto, por lo que un animal no debería especializarse nunca en un tipo de comida menos preferido sin importar su abundancia. (2) Según incrementa la abundancia de un tipo de comida preferido incluido en la dieta, el número de tipos de comida menor preferidos incluidos en la dieta óptima disminuirá, es decir, incrementar la abundancia incrementará la especialización. Si la abundancia de un item ya incluido en la dieta se hace infinitamente grande, todo lo de menor rango será abandonado. (3) Si un item está incluido en la dieta debería ser consumido siempre que se encuentra, y nunca si no lo está. Es decir, no debe haber preferencias parciales. Sin embargo, si la teoría se extiende para incluir restricciones de dieta o variaciones al azar en la abundancia de comida, se pueden esperar preferencias parciales.

La única de estas predicciones apoyada por los datos es la segunda: incrementar la abundancia incrementa la especialización.

Se ha buscado probar la predicción de que la dieta óptima siempre incluye al tipo de comida de rango más alto,y que otros tipos se añaden en un orden dictado por su rango según disminuye la abundancia. Charnov ha encontrado evidencia que indica que, según disminuye la cantidad de comida en el estómago de una mantis (lo que le permite estimar la abundancia en el ambiente), ésta estará dispuesta a perseguir items que se encuentra más y más lejos (equivalente a comida con una menor razón ingesta/tiempo de manejo). Aunque no se puede descartar que la explicación esté más bien en el riesgo: transportarse más lejos implica mayor riesgo, por lo que solo debería hacerse si es importante obtener comida.

Werner y Hall evaluaron las dietas de peces según variaba la abundancia de sus presas favoritas. Lograron predecir con precisión el punto en el que tipos de presa menos preferidos entrarían en la dieta en función de la disminución de la abundancia del tipo preferido.

Krebs probó la predicción de que la inclusión de un tipo de comida en la dieta depende de la abundancia de otros tipos de comida más preferidos. La predicción fue apoyada por un experimento con pájaros carboneros. Sin embargo, el experimento mostró que, contrario a la predicción de que en un cierto umbral de disponibilidad de la presa más preferida la menos preferida debería ser abandonada abruptamente, el abandono fue más bien gradual.

En un estudio de campo Goss Custard encontró que los pájaros archibebes muestran una preferencia por las presas más grandes que es dependiente solo de la disponibilidad de presas grandes, pero no de las pequeñas.

Aunque estos éxitos son alentadores, debe tenerse en cuenta que se trata de situaciones simples. Situaciones más complejas en que se deba considerar, por ejemplo, la toxicidad de las plantas consumidas y sus interacciones llevarán a una teoría simple como esta a fallar.

Belovsky incluye algunas restricciones en su estudio que buscaba determinar la dieta óptima de un alce con tres distintas fuentes de comida sometido a varias limitaciones: la ingesta diaria de energía no debe ser negativa, hay un límite superior en el tiempo que el alce puede dedicar al forrajeo en entornos acuáticos y terrestres (dada la termorregulación del animal), y debe haber una cierta ingesta mínima diaria de sodio. Estas limitaciones fueron expresadas como funciones lineales de las cantidades de comida diarias de cada tipo bajo dos hipótesis: (1) el alce busca consumir la mayor cantidad posible por día dadas sus restricciones, y (2) el alce busca satisfacer sus necesidades en el menor tiempo posible. Aunque ambas hipótesis predecían bien los datos, la mejor era la de maximización de energía. Además, con su modelo predijo el tamaño óptimo de un alce adulto. En su predicción los alces macho deberían ser más pequeños, cuando la realidad es contraria. Esto indica que la ventaja reproductiva del tamaño se sobrepone a la desventaja energética, lo que indica que la elección de la moneda de cambio apropiada es importante.

Finalmente debe considerarse el trabajo de D. S. Wilson, que no probó predicciones de dieta óptima, sino que las asumió como verdaderas e investigó consecuencias de ese supuesto. Comparó aves tropicales contra templadas con respecto a la disponibilidad de comida. Se ha argumentado que en presas pequeñas el orden de rango es igual al orden del tamaño, por lo cual las presas más {\itshape pequeñas} que de hecho se incluyen en la dieta deberían señalar el punto de corte, y por lo tanto reflejar la tasa promedio de ingesta de comida durante una sesión de forrajeo.

Wilson analizó el contenido estomacal de las aves y definió el tamaño mínimo aceptable de comida como la clase más pequeña que compusiera al menos el 0.05\% del contenido total del estómago. Mostró que (1) el mínimo aceptable era más pequeño en los meses menos productivos de invierno, y (2) que en los trópicos era relativamente constante, pero era variable en lugares de clima más variable dada la menor disponibilidad.

{\scshape\bfseries Optimal patch choice}

El problema de elegir parches es análogo al problema de dieta óptima solo si cada parche debe encontrarse antes de ser aceptado o rechazado. Hasta ahora no se ha estudiado ese tipo de forrajeo. Si el animal tiene conocimiento de todos los parches puede pasar de uno preferido a otro (si el costo de viaje es poco).

Smith y Dawkins estudiaron aves en un entorno con distintos parches de comida a los que se tenía acceso por períodos breves (para que la abundancia de comida se alterara muy poco). En tal entorno lo óptimo sería alocar todo el tiempo en el parche más ventajoso, pero las aves no hicieron eso. En lugar de ello alocaron tiempos progresivamente menores a áreas progresivamente peores. Esto parece ser una adaptación a largo plazo a un entorno cambiante: si la densidad relativa de las áreas cambia en el entorno natural, es adaptativo que los animales muestreen todas ellas. En este caso la estrategia óptima a corto plazo no lo sería a largo plazo. Pero aun hace falta trabajo en esta teoría para determinar la alocación óptima de tiempo en áreas de distinta abundancia.

{\itshape En muchas situaciones las estrategias óptimas a corto y largo plazo difieren, y lo más probable es que la adecuación esté más estrechamente relacionada con el resultado a largo plazo.} Un ejemplo sería un animal con control casi exclusivo de un área de forrajeo: puede incrementar su ganancia a largo plazo al forrajear por debajo de la máxima eficiencia a corto plazo. Pero se predice que animales sin uso casi exclusivo de un área se decantarán por estrategias a corto plazo dado que no ganaría nada al conservar la comida para que otros la coman.

{\scshape\bfseries Optimal allocation of time to patches}

Al explotar un parche su redituabilidad disminuye. Gibb estudió éste fenómeno y propuso la hipótesis de ``{\itshape hunting by expectation}'' según la cual un animal aprende a esperar una cierta cantidad de comida de cada parche y lo abandona cuando obtiene esa cantidad de comida. El único estudio que ha buscado probar esta hipótesis de forma cuantitativa es el de Krebs, Ryan y Charnov. Sus resultados apoyaron a la teoría de forrajeo óptimo por encima de la hipótesis de {\itshape hunting by expectation} y de una modificación llamada {\itshape hunting by time expectation}. 

La predicción de la teoría es que un animal debería abandonar un parche cuando su tasa de ingesta cae hasta la tasa promedio de todo el hábitat, y que esta tasa de captura ``marginal'' debería ser igualado dentro del hábitat. Esta predicción se conoce como {\itshape marginal value theorem}. 

Para probar esta predicción se deben hacer supuestos sobre cómo los animales estiman la cantidades inherentes a la teoría.

Krebs, Ryan y Charnov evaluaron aves con dos presunciones implícitas: que tienen un {\itshape giving-up time} (tras un cierto tiempo sin recibir comida en un parche, lo abandonan), y que con base en él pueden estimar la tasa marginal de captura, y que ésta es inversamente proporcional al {\itshape giving-up time}. De esto se obtiene que el {\itshape giving-up time} debería ser constante dentro de un hábitat, y que debería ser menor en hábitats más ricos. Esto es precisamente lo que encontraron, lo que apoya a la teoría de forrajeo óptimo y se opone a la hipótesis de {\itshape hunting by expectation}.

La relación inversa entre la tasa marginal de captura y el {\itshape giving-up time} es razonable, pero hasta no aclarar su relación estadística no está claro por qué deberían ser proporcionales, aunque empíricamente ha resultado en predicciones acertadas.

También es posible que la conducta cambie en función de alteraciones en la disponibilidad de la comida independientes del depredador (como la emergencia cíclica de insectos en el agua en función de la hora del día y la estación).

{\scshape\bfseries Optimal patterns of movements of foraging animals and optimal speed of movement}

Un hallazgo general es que los animales suelen divagar hasta que una presa es encontrada, y entonces incrementan su tasa de giro, con lo que permanecen en la cercanía de la presa encontrada. Se ha sugerido que esto es una adaptación ante presas que se encuentran en cúmulos, pero no esa claro en qué grado se debe incrementar la tasa de giro, ni durante cuánto tiempo.

Cody y Pyke han construido modelos que simulan el movimiento animal basados en la suposición de que los animales se mueven desde puntos en una red uniformemente delimitada hacia los vecinos más cercanos de estos puntos, y que la comida se encuentra en los puntos ({\itshape i.e.,} se asume una distribución discreta y que los movimientos están restringidos a un área delimitada). Se asume que los animales se mueven según 4 probabilidades correspondientes a las cuatro direcciones, y que su movimiento es independiente de la comida distante, que no es detectada por los sentidos. Los autores pensaron que la tasa máxima de ingesta ocurriría si se minimiza el repaso de caminos, de modo que determinaron el patrón de movimientos que lo minimizara y lo compararon con los patrones observados. 

Cody evaluó a parvadas de aves y propuso que sus movimientos son óptimos dado que minimizan el repaso de caminos. En un área delimitada bajo el supuesto de que, al chocar con un límite el siguiente movimiento sería hacia atrás, encontró que el comportamiento observado se ajustaba al predicho.

Si embargo no todos los animales tratan a los límites del entorno como reflejantes, sino que se mueven hacia la izquierda, derecha o hacia atrás con ciertas probabilidades. Pyke ha mostrado que las simulaciones difieren de las observaciones en cuanto a la {\itshape direccionalidad} (el vector promedio de movimientos en la dirección hacia adelante). Esto podría significar (1) que la hipótesis del forrajeo óptimo está mal en este caso o (2) que los supuestos del modelo son incorrectos ({\itshape e.g.,} los animales sí detectan la comida distante y se mueven en función de ella). Esto parece ser apoyado por un estudio con abejorros en el que las predicciones hechas suponiendo que los parches distantes sí influyen en el movimiento fueron consistentes cualitativamente con las observaciones.

El futuro de esta área es más incierto dado que es necesario conocimiento cuantitativo de los sentidos que permiten a los animales detectar comida. Además se deberá determinar la distribución exacta de comida y la relación entre las claves usadas por los animales y la probabilidad de encontrar comida. Además será necesario considerar la velocidad de movimiento.

{\scshape\bfseries The future}

{\scshape A. Models which look at several factors at once}

En la aproximación actual se divide el forrajeo en varios problemas aislados, pero modelos recientes abordan más de uno a la vez, como el modelo de ``{\itshape central place foraging}'', que modela animales que parten de un nido central, consiguen comida, y luego regresan, con lo que considera la elección de parches y la amplitud de dieta.

{\scshape B. Information gathering, memory, and estimation of parameters}

Una presunción implícita de los modelos es que los animales ``saben'' o estiman los parámetros de las ecuaciones. Para ello deben poder recolectar información relevante, recordarla, y traducirla en estimados de los valores de los parámetros.

Para esto necesitan de sistemas sensoriales para recolectar información. Surge la pregunta, ¿por qué no son mejores sus sistemas sensoriales? Puede ser un equilibrio entre las ganancias de más información y los costos de obtenerla. Lo mismo ocurriría con la memoria.

Quizá el progreso más rápido estará en las áreas de cómo los animales traducen la información en estimados de parámetros dado que las medidas estadísticas como la precisión y el poder deben dar medidas cuantificables de las ganancias obtenibles del uso de mayores cantidades de información, además de sugerir estadísticos óptimos para los animales.

{\scshape C. Coevolution}

Las teorías de optimalidad deberían dar información sobre la naturaleza de sistemas coevolucionarios, como depredadores y presas, especies que compiten o especies simbióticas. Por ejemplo en el caso de los polinizadores lo que evoluciona son los patrones de movimiento entre las flores; y en el caso de las plantas evoluciona el número de flores por planta, su distribución y la cantidad de néctar por flor. La teoría de forrajeo óptimo permitiría predecir el movimiento de los polinizadores en un conjunto de características de las plantas. Y también sería necesaria una teoría que prediga, dado un patrón de movimiento de los polinizadores, las características óptimas de una planta. Al combinar ambas teorías se debería poder alcanzar un ``equilibrio coevolutivo''. 

{\scshape Community structure}

En muchas especies la obtención de recursos --y, por lo tanto, la depredación-- se relaciona directamente con el crecimiento de la población. Igualmente la similitud de estrategias de depredación determina la coexistencia competitiva entre e intra comunidad. Así, la depredación es el centro de la estructura de la comunidad.

La teoría de forrajeo permitirá entender las comunidades solo si puede predecir las relaciones tróficas --quién come qué, cuándo y dónde.

{\scshape Optimality thinking in population biology}

El punto de vista de la optimalidad en la depredación ha probado ser útil. Ha sugerido hipótesis comprobables que puntos de vista más tradicionales no hubiesen considerado.

Estos modelos son sin duda simplistas, pero son un comienzo necesario. Quizá la teoría de forrajeo óptimo continuará proveyendo un entendimiento predictivo de cómo los animales forrajean y dará indicios para varias áreas relacionadas de la biología.

\end{document}
