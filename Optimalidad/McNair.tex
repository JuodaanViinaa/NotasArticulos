\documentclass[a4paper,12pt]{article}
\usepackage[utf8]{inputenc}
\usepackage[T1]{fontenc}
\usepackage[spanish]{babel}
\usepackage{csquotes}
\usepackage{anysize}
\usepackage{graphicx}
\marginsize{25mm}{25mm}{25mm}{25mm}

\title{Optimal giving-up times and the Marginal Value Theorem}
\author{James McNair}
\date{1982}

\begin{document}
{\scshape\bfseries \maketitle}

Se ha usado el teorema de valor marginal (MVT) para predecir que los {\itshape giving-up times} (GUT) serán iguales en todos los parches de un hábitat. Sin embargo, el modelo en que se basa MVT no está hecho para hacer predicciones del GUT.

El mismo GUT no debería aplicarse a parches de distinta calidad. Más bien, GUT más largos deberían usarse en parches más ricos. En algunos hábitat será mejor una estrategia de GUT óptimo; en otros, una de tiempo de residencia (RT) óptimo.

Hay muchas estrategias óptimas, pero todas predicen que la ingesta energética por parche y el tiempo de residencia serán mayores en parches más ricos, y que ambas incrementarán según incrementa el tiempo de viaje entre parches.

{\scshape\bfseries Origin of the GUT concept}

Tinbergen lo estudió primero sin nombrarlo en cuervos. Se definía como una cualidad de la persistencia de los animales, una expresión de el esfuerzo que el depredador está dispuesto a asignar al perseguir un item más de una presa particular.
Después lo estudiaron Smith y Dawkins, y luego Krebs. El concepto de GUT tiene un origen empírico. Es un indicador natural de la respuesta de un depredador ante la incertidumbre de un ambiente distribuido en parches.

{\scshape\bfseries From MVT to GUT: a misleading step}

Krebs (1974) fue la primera publicación del MVT, que se resume así: un animal forrajea en un hábitat en el que la comida está en parches de tipos $1, 2, 3,\ldots, n$. En el parche la ingesta acumulativa de energía ({\itshape yield}) se asume como una función diferenciable del tiempo, de modo que tiene sentido hablar de la tasa de ingesta instantánea en un parche. Esta tasa es inicialmente positiva pero declina según se depleta el parche. El forrajeador pasa un tiempo fijo de residencia $T_{i}$ en el parche tipo $i$ y abandona. Después debe viajar por un tiempo $\tau$ antes de encontrar el siguiente parche. Se asume que el animal puede identificar parches. Al saber las frecuencias relativas de encuentro de distintos tipos de parches se puede escribir una fórmula de la tasa de ingesta energética promediada en un número grande de visitas a parches. Esta {\itshape tasa de ingesta del hábitat} es una función de os tiempos de residencia $T_{1}, T_{2},\ldots, T_{n}$. ¿Cómo deberían elegirse los $T_{i}$ para maximizar la ingesta? Según el MVT, se deberían escoger los $T_{i}$ de modo que cada tasa instantánea de ingesta, evaluada en el $T_{i}$ apropiado, sea igual a la tasa del hábitat, evaluada en los mismos valores de $T_{1}, T_{2},\ldots,T_{n}$. No se puede conocer la tasa del hábitat a menos que se conozcan las $T$, por lo que un animal debe saber de antemano cuánto permanecerá en un parche antes de entrar en él, lo que es poco plausible biológicamente.

El modelo en que se basa el MVT predice tiempos de residencia, y lleva a una estrategia óptima de RT. Aun así, autores han tratado de aplicarlo para el GUT asumiendo que todos los GUT deberían ser iguales entre parches. El sentido común nos dice que el GUT debería ser mayor en parches que se sabe que son ricos (mayor persistencia).

La predicción de que los GUT deberían ser siempre iguales viene de dos pasos: primero, se aplica el MVT a situaciones en las que no se debería. Por ejemplo, cuando la comida se encuentra en momentos aleatorios, la ingesta resultante tendría grandes saltos y decremento gradual y la tasa instantánea de ingesta tendría saltos erráticos entre valores positivos y negativos, contrario a las hipótesis de MVT.

Segundo, se asume que $\frac{1}{observed\,GUT}$ representa la tasa instantánea de ingesta en el momento de abandono del parche.

El primer paso se puede remediar sustituyendo la ingesta acumulada media en un parche (función del tiempo de residencia) por el {\itshape yield} (función aleatoria). Entonces se puede ya pensar en ejemplos en los que los supuestos del MVT se cumplan. El segundo paso no se puede remediar.

{\itshape Me salté medio artículo porque no parece servirme y porque me aburrió. Pasaré directo a la discusión. Si veo que algo me serviría, quizá lo lea después.}

{\scshape\bfseries Discussion}

En la mayor parte de los aspectos, las propiedades de una estrategia GUT son similares a las de la estrategia RT desarrollada por Charnov y otros. Mientras los parches tengan ganancias decrecientes, ambas estrategias predicen que los forrajeadores pasarán más tiempo en los parches ricos que en los pobres, y tanto la residencia como el {\itshape yield} deberían incrementar según incrementa el tiempo de viaje.

Modelos de este tipo reciben críticas similares, como ser muy simples. Por ejemplo, una complicación no contemplada es el riesgo de un forrajeador de convertirse en presa. Otro problema es cómo un forrajeador aprende en dónde se localizan los parches ricos, en especial cuando las localizaciones cambian en el tiempo. Esto lleva a la necesidad de muestrear. Aun así, buena cantidad de artículos apoyan las predicciones de estos modelos.

Hay evidencia que apoya la predicción de que mejores parches llevan a mayor tiempo de residencia y mayor {\itshape yield}; y de que mayor tiempo de viaje lleva a mayores tiempos de residencia.

El GUT no tendría por qué ser idéntico en parches de distinta calidad en el mismo hábitat. Los forrajeadores deberían ser más persistentes en parches de mejor calidad.

Cowie y Krebs (1979) han argumentado que medir los GUT no es una buena forma de probar la teoría del forrajeo óptimo. Su razonamiento es que si los forrajeadores abandonaran parches caprichosamente, el GUT promedio sería más pequeño en un hábitat lleno de parches ricos que en uno con parches pobres, lo que estaría engañosamente de acuerdo con la teoría de forrajeo óptimo. El mismo forrajeador mostraría GUTs más pequeños en los parches ricos que en los pobres en el mismo hábitat. 

Ejemplos numéricos dejan claro que ninguna estrategia, GUT o RT, es superior en todos los casos, y que otras estrategias podrían ser superiores a ambas. Sería interesante saber qué condiciones favorecen cada tipo de estrategia. 


\end{document}
