\documentclass[a4paper,12pt]{article}
\usepackage[utf8]{inputenc}
\usepackage[T1]{fontenc}
\usepackage[spanish]{babel}
\usepackage{csquotes}
\usepackage{anysize}
\usepackage{graphicx}
\marginsize{25mm}{25mm}{25mm}{25mm}

\title{Foraging decisions: patch choice and exploitation by pigeons}
\author{John Hanson \and Leonard Green}
\date{1989}

\begin{document}
{\scshape\bfseries \maketitle}

No hay experimentos que analicen la preferencia de parches. Lo poco que hay no ha utilizado encuentros secuenciales, que son un supuesto central en la teoría de forrajeo óptimo.

En selección de parches un forrajeador debe elegir qué parches incluir en su set de tipos de parche explotables. Los supuestos de este problema son análogos a los de selección de presas: (1) las búsquedas entre e intra parche son actividades mutuamente excluyentes; (2) los parches se encuentran secuencial y aleatoriamente; (3) los tipos de parches se reconocen al instante; (4) la curva de ganancia acumulativa asociada con cada tipo de parche es negativamente acelerada; (5) los tipos de parche se valúan diferencialmente de acuerdo con la ganancia energética; y (6) el forrajeador tiene información precisa de todos los parámetros ambientales.

En un entorno con dos parches que cumpla estas condiciones, la elección de parche se describe con el modelo clásico de elección de presas con los términos redefinidos para un entorno en parches. Un forrajeador debería aceptar ambos parches si 
\begin{equation}
	\frac{
		\lambda_{r}E_{r}
	}{
		1 + \lambda_{r}h_{r}
	}
	<
	\frac{
		E_{p}
	}{
		h_{p}
	}
\end{equation}
donde $E_{r}$ y $E_{p}$ son las ingestas netas de energía de los tipos de parche rico y pobre, y $h_{r}$ y $h_{p}$ es el costo medio de manejo del parche asociado con cada tipo. La tasa de encuentro con el tipo de parche rico, $\lambda_{r}$, se expresa en términos de oportunidades por unidad de costo de búsqueda. Si $S$ es el costo promedio de moverse de un tipo de parche a otro, y $P(r)$ es la probabilidad de que un parche encontrado sea del tipo rico, entonces $\lambda_{r} = \frac{P(r)}{S}$. Igualmente, $\lambda_{p} = \frac{(1-P(r))}{S}$.

$E$ puede verse como el número de capturas de presa en el parche, y $h$ como el tiempo de residencia o la energía requerida para remover $E$ presas del parche. A diferencia del modelo de elección de presas en el que la ganancia energética y los costos de manejo son fijos para cada tipo de presa, en selección de parches esos parámetros varían con la conducta del forrajeador. Se pueden determinar los valores esperados de $E$ y $h$ en un ambiente si se asume que el forrajeador maximizará la tasa de captura en él.

Se pueden hacer varias predicciones cualitativas: (1) siempre se aceptará el parche de tipo más rico al encontrarlo. (2) La selectividad disminuirá al bajar la tasa de encuentro con el parche más rico. (3) La aceptación del tipo más pobre será independiente de su propia tasa de encuentro $\lambda_{p}$ (ese parámetro no aparece en la desigualdad 1). (4) la aceptación del tipo más pobre debería ser todo-o-nada.

A diferencia de la elección de parches, la explotación de parches (el tiempo pasado en un parche, la cantidad de presas atrapadas en él, las reglas para determinar el abandono) ha recibido mucha atención. Estos estudios se centran en la eficacia predicha por el teorema de valor marginal, que indica que un parche debería ser abandonado cuando su tasa de recompensa baja hasta igualar al promedio del ambiente.

Esto lleva a varias predicciones cualitativas: (1) los parches más ricos deberían explotarse más que los pobres. (2) Según baja la calidad del ambiente, el tiempo y esfuerzo usados en cada parche deberían incrementar. (3) En un mismo ambiente los parches deberían explotarse hasta niveles comparables.

Estos experimentos variarán la tasa de encuentro con cada parche manipulando dos parámetros: el costo de búsqueda y la proporción rico/pobre de los parches. Todo con el propósito de probar las predicciones de la teoría de forrajeo óptimo sobre la elección y explotación de parches.

{\scshape\bfseries General methods}

Se utilizaron 6 palomas en cajas de condicionamiento con dos teclas.

Al comenzar una sesión se encendía la tecla izquierda en ámbar. Responder en ella avanzaba un programa de razón variable (VR). Completarlo encendía la tecla derecha en rojo o verde y comenzaba un programa de VR seguido del cual se entregaba comida. Tras cada entrega de comida se incrementaba el requisito de VR del eslabón final. Había dos alternativas: la más rica comenzaba en VR 2 e incrementaba en la mitad de su valor con cada entrega (2, 3, 4.5, etc). La más pobre era 10 veces mayor. La alternativa mostrada era determinada probabilísticamente según $P(r)$ y $1-P(r)$.

Se podía rechazar la alternativa picando la tecla izquierda una vez, lo que regresaba a la fase de búsqueda y contaba para su nuevo requisito. La aceptación de la alternativa se indicaba con una respuesta en la derecha. Aun así, se podía rechazar la alternativa en cualquier momento antes o después de la entrega de recompensa.

Al cumplir el requisito de la tecla derecha se entregaban 3 s de acceso a comida y se encendían de nuevo las dos teclas tal como antes. Las sesiones terminaban al aceptar 20 oportunidades de la tecla derecha o al pasar una hora.

Los datos examinados fueron las probabilidades de aceptar los dos tipos de parches. Debía haber estabilidad en ambas medidas para pasar a otra condición.

Este procedimiento cumple las presunciones de forrajeo óptimo mencionadas antes y mantiene la estructura de forrajeo en un ambiente en parches. La tecla izquierda funciona como tecla de búsqueda entre parches. Se modela el requerimiento de tiempo y de esfuerzo. Las oportunidades de tecla derecha corresponden con oportunidades de captura de múltiples presas disponibles. Las búsquedas intra y entre parches son mutuamente excluyentes, y el encuentro de tipos de parche es probabilístico. Además los parches son discriminables y hay ganancias decrecientes. La presunción de valoración diferencial dependería de la conducta, y la de conocimiento del ambiente se satisfacía al esperar hasta la estabilidad en la conducta para hacer una evaluación.

Todos los sujetos pasaron inicialmente por el experimento 1 y después se distribuyeron en otros. Cada uno pasó por combinaciones distintas de dos de los tres experimentos restantes.

{\scshape\bfseries Experiment 1}

La probabilidad de encontrar el parche rico, $P(r)$ se fijó en .5. Se estudiaron valores de costo de búsqueda de 5, 10, 20, 40, 60, 80, 120, y 160.

Según incrementó el costo de búsqueda, la tasa de encuentro con ambos parches bajó (baja la calidad del ambiente). Dados los parámetros ambientales y las curvas de ganancias acumuladas se predijo un punto de indiferencia en $\lambda_{r} = 0{.}013$, es decir, en un costo de búsqueda correspondiente a VR 40, por lo que más abajo de VR 40 se predice una estrategia especialista; y más arriba, generalista.

{\bfseries Results and discussion}

Se aceptó casi toda oportunidad en el parche rico. Todos los sujetos comenzaron a aceptar el parche pobre mientras incrementó el costo de búsqueda. Si se toman valores de $P(accept\,poor)$ menores a 0.1 como indicativos de una estrategia especialista, entonces todos los sujetos salvo dos fueron especialistas debajo de VR 40, y tres fueron generalistas arriba de VR 40. Los demás mostraron preferencias parciales.

Los sujetos mostraron más esfuerzo en explotar los parches según incrementó el costo de búsqueda. Aún cuando ambos parches eran aceptados se realizaban más respuestas en los parches ricos que en los pobres.

El número de presas capturadas por parche incrementó con el costo de búsqueda, y se capturaron más presas en los parches más ricos.

Al abandonar un parche se registraba el valor del programa de razón que estaba en efecto. Esto mostraba un índice del grado de depleción del parche, La teoría de forrajeo óptimo predice que ambos parches deberían depletarse a niveles equivalentes, {\itshape i.e.,} sus VR de abandono deberían ser similares. También se predice que al aumentar el costo de búsqueda los parches deberían depletarse más severamente. Ambas cosas fueron observadas, aunque no se observó una correspondencia exacta debido a la disparidad en los incrementos del VR de cada parche.

La cantidad de respuestas desde la última captura en el parche hasta el abandono es la medida de {\itshape giving up}. Las correlaciones entre el costo de respuesta y las respuestas de {\itshape giving up} no fueron significativas. Aunque es interesante que cerca de un tercio de las ocasiones los sujetos abandonaron un parche inmediatamente después de recibir una recompensa.

En resumen, los resultados del experimento 1 están en acuerdo cualitativamente con las predicciones de la teoría de forrajeo óptimo con respecto a la elección de parches. La selectividad fue mayor en costos bajos, los parches ricos casi siempre fueron aceptados, y la aceptación del parche pobre estuvo cerca del todo-o-nada. Las estrategias predichas (especialista y generalista) fueron acertadas el 82\% de las ocasiones. Según incrementó el costo de búsqueda, se explotó en mayor medida los parches. Todos los sujetos sobre-explotaron los parches de forma relativa a las predicciones cuantitativas de forrajeo óptimo.

{\scshape\bfseries Experiment 2}

En el segundo experimento se mantuvo constante el costo de búsqueda en VR 20, y se varió la probabilidad $P(r)$ entre condiciones de 0.1 a 0.9. En este caso también se predice un punto de indiferencia en $\lambda_{r} = 0{.}013$, que es equivalente a una probabilidad de $P(r) =  0{.}25$. Debajo de 0.25 se predice una estrategia generalista; arriba, especialista. Los valores estudiados fueron de .1, .25, .5, .75, y .9.

{\bfseries Results and discussion}

Los sujetos aceptaron todas las oportunidades del parche rico, y se volvieron más selectivos según incrementó $P(r)$. Cuando $P(r)$ era mayor que .25, dos sujetos rechazaron todas las oportunidades pobres, como se predijo. Los dos restantes mostraron aceptación parcial.

La estrategia predicha concordó con la observada en el 53\% de las ocasiones.

Dos sujetos son responsables de los fallos en las predicciones. Uno de ellos aceptó siempre las oportunidades pobres aún en $P(r) = 0{.}9$, lo que es entendible si se toma en cuenta que en esa probabilidad se encontraban en promedio solo dos oportunidades pobres por sesión.

Hubo más respuestas por parche según decrementó $P(r)$ (al bajar la calidad del ambiente). Lo mismo sucedió con el número de presas tomadas de cada parche. Cuando ambos parches eran aceptados, se tomaban más presas del parche rico que del pobre. En general se tomaron más presas por parche de las que se predice con la teoría de forrajeo óptimo.

Los valores de VR al momento de abandonar el parche incrementaron según decrementó $P(r)$, lo que indica una mayor depleción de recursos cuando bajó la calidad ambiental.

No hubo tendencia para la medida de {\itshape giving-up} para variar junto con los cambios en $P(r)$. En la mayoría de los casos las {\itshape giving-up responses} eran iguales a cero.

De nuevo los resultados están en acuerdo cualitativo con la teoría de forrajeo óptimo. Se aceptaron todas las oportunidades ricas, con aumentos en $P(r)$ incrementó la selectividad, y con sus disminuciones incrementó la explotación. Pero en este caso no fue aparente la aceptación todo-o-nada del parche pobre. Igual que en el experimento 1, el nivel de explotación fue mayor que el predicho.

{\scshape\bfseries Experiment 3}

Según la teoría de forrajeo óptimo, variar la tasa de encuentro con el parche rico es suficiente para determinar el cambio en la conducta de elección de parches. Este experimento pretendía cambiar solo la tasa de encuentro del parche rico manteniendo constante la tasa del pobre.

Covariando el costo de búsqueda y $P(r)$ es posible construir una serie de ambientes en los cuales el encuentro con el parche pobre es constante pero el del parche rico varía. Esto se hace pareando costos de búsqueda altos con valores bajos de $P(r)$, y después según el costo de respuesta bajó, $P(r)$ fue incrementado. La tasa de encuentro con el parche pobre estuvo fija en 0.025 parches por respuesta de búsqueda, mientras la tasa del parche rico varió de 0.003 a 0.225 parches por respuesta de búsqueda.

{\bfseries Results and discussion}

Casi todas las oportunidades del parche rico fueron aceptadas. Los sujetos se hicieron más selectivos cuando la tasa de encuentro con el parche rico incrementó. En los valores más altos de $\lambda_{r}$ se rechazaron todas las oportunidades del parche pobre, mientras que en los valores bajos se aceptaron casi todas. 

Los sujetos mostraron aceptación todo-o-nada del parche pobre en 10 de las 12 condiciones. La estrategia predicha fue observada en el 83\% de las condiciones.

Se observaron más respuestas por parche en ambientes más pobres. Cuando ambos parches eran aceptados, se emitían más respuestas en el rico. Se observaron más respuestas por parche de las predichas, al igual que número de presas tomadas por parche. El número de presas tomadas incrementó cuando bajó la tasa de encuentro con el parche rico.

El VR en efecto en el momento de abandono incrementó en ambos parches según bajó la tasa de encuentro con el parche rico.

No hubo ninguna tendencia en la medida de {\itshape giving up}.

De nuevo, los resultados están cualitativamente de acuerdo con las predicciones del forrajeo óptimo. Todas las oportunidades ricas fueron aceptadas, la selectividad incrementó cuando incrementó el encuentro con el parche rico, la aceptación todo-o-nada del parche pobre fue cercanamente aproximada, u los parches se explotaron en mayor medida con valores bajos de tasa de encuentro del parche rico. La elección de parche si cambió en función de exclusivas variaciones de la tasa de encuentro con el parche rico.

{\scshape\bfseries Experiment 4}

La teoría de forrajeo óptimo predice que variaciones en la tasa de encuentro con el parche pobre no tendrán efecto en la elección. Este experimento busca probar esa predicción manteniendo fija la tasa de encuentro con el parche rico y variando solo la del parche pobre.

Esto se puede lograr, igual que en el experimento 3, covariando el costo de búsqueda y $P(r)$. En este caso se parearon costos relativamente altos con valores altos de $P(r)$ y al revés. Así, la tasa de encuentro con el parche rico permaneció fija en 0.025 parches por respuesta de búsqueda, mientras que la del pobre varió entre 0.003 y 0.225.

{\bfseries Results and discussion}

De nuevo aceptaron casi todas las oportunidades del parche rico. En este caso se predecía una estrategia especialista en todas las condiciones. En lugar de ello se observó que las oportunidades del parche pobre eran más aceptadas cuando bajaba la tasa de encuentro con él. Es de importancia resaltar que en la condición con menos oportunidades del parche pobre éste solo se presentaba dos veces. La estrategia especialista se observó solo en el 58\% de las condiciones. 

Se registraron más respuestas de las predichas a los parches. No hubo ninguna tendencia a variar la cantidad de respuestas en función del cambio en la tasa de encuentro con el parche pobre. Se realizaron más respuestas en los parches ricos que en los pobres cuando ambos eran aceptados, y también se capturaron más presas.

No hubo tendencia a variar el número de presas tomadas según varió la tasa de encuentro con el parche pobre. De nuevo, se capturaron más presas de las predichas por la teoría de forrajeo óptimo. 

A diferencia de los otros experimentos, el valor del VR al abandonar el parche pobre fue mayor que al abandonar el rico. Este patrón indica una mayor depleción del parche pobre.

De nuevo, los sujetos tendían a abandonar un parche inmediatamente después de obtener una presa.

En resumen, no se apoyó la predicción de que la variación en la tasa de encuentro del parche pobre es inconsecuente para la elección. Solo un sujeto mostró conducta especialista en todos los valores de $\lambda_{p}$. Los sujetos de nuevo sobreexplotaron los parches de forma relativa a lo predicho por la teoría de forrajeo óptimo.

Dado que la calidad ambiental era invariante no se encontró una tendencia para incrementar o decrementar las respuestas, las presas tomadas, o lo VR en efecto al abandonar.

{\scshape\bfseries General discussion}

Como predice la teoría de forrajeo óptimo, los sujetos aceptaron el parche rico casi siempre. Estos datos son comparables con estudios en el modelo de elección de presa.

Como predice la teoría de forrajeo óptimo, cuando bajó la calidad ambiental (disminuyendo la tasa de encuentro con el parche rico), disminuyó la selectividad. También eso es consistente con estudios en el modelo de elección de presas.

A diferencia de lo que predice la teoría de forrajeo óptimo, no se encontró una decisión de todo-o-nada en cuanto a la aceptación de los parches pobres. En cambio, estudios con el modelo de selección de presa sí han encontrado un acuerdo cercano con esa predicción.

Tomando en cuenta todos los experimentos, la conducta de los organismos se ajustó a las predicciones en cuando a la estrategia utilizada en el 72\% de las ocasiones.

La teoría de forrajeo óptimo predice que la variación de la tasa de encuentro con el parche rico es necesaria y suficiente para modificar la selección. Se probó que es suficiente, pero no que sea necesaria, pues variar la tasa del parche pobre también tuvo efectos en la selección.

A este respecto, los estudios del modelo de selección de presas son menos claros y hay hallazgos contradictorios. Aunque la tendencia general parece ser que la selectividad disminuye cuando disminuye la tasa de encuentro con la presa pobre, tal como en este experimento.

En este caso, es posible que la explicación esté en una mayor sensibilidad al costo de búsqueda que a los cambios en la proporción de parches. En los experimentos 1, 3 y 4 los sujetos aceptaron en mayor medida los parches obres cuando incrementó el costo de búsqueda, mientras $P(r)$ estuvo fijo, aumentó y disminuyó, respectivamente. En cambio, en el experimento 2 en el cual se mantuvo constante encosto de búsqueda pero varió $P(r)$ se encontró variación en la conducta, pero no de forma tan abrupta como en 1, 3 y 4. Es posible que los organismos midan la riqueza del ambiente dando mayor peso al tiempo promedio de búsqueda que a la abundancia relativa de los tipos de parche, lo cual explicaría estos resultados.

La teoría de forrajeo óptimo predice que mientras baja la calidad del ambiente los parches deberían explotarse con mayor severidad. Estos patrones fueron observados en estos experimentos.

El teorema de valor marginal puede pensarse como una predicción de una expectativa de {\itshape hunting-by-rate}, es decir, el organismo espera una tasa media de captura de presa en un ambiente, y cuando un parche llega a esa media, el forrajeador abandona.
Otras explicaciones de cazar-por-expectativa son {\itshape hunting-by-number} (el organismo espera un cierto número de presas por parche, y al encontrarlas abandona; con esto se esperaría la obtención de la misma cantidad de presas en todos tipos de parches) y {\itshape hunting-by-time} (el organismo pasa el mismo tiempo en todos los tipos de parches). Ninguna de esas alternativas fue apoyada por los datos.

Los parches fueron sobreexplotados. Es de interés señalar que esto no sucedió de forma consistente, como si existiese un punto de ajuste más arriba de lo predicho, sino que los sujetos sobreexplotaron a diferentes niveles.

Se ha dicho que los {\itshape giving-up times} deberían ser un indicador de la tasa de captura de un parche, y que todos los parches en un entornos deberían compartir un GUP. Según baja la calidad ambiental, el GUP debería alargarse. Ese no fue el caso en este experimento: no hubo ninguna tendencia para incrementar las {\itshape giving-up responses} cuando la calidad ambiental disminuyó, y de hecho a menudo eran iguales a cero.

La tendencia a abandonar inmediatamente después de un reforzador es entendible dada la naturaleza ``{\itshape quantal}'' de las presas. En ambientes naturales permanecer en un parche da un incremento constante en la ingesta, pero en esta situación las presas son discretas y separadas por tiempos vacíos. Abandonar después de esperar un tiempo largo es costoso, de modo que es mejor abandonar inmediatamente después de recibir comida.

Se ha aplicado la hipótesis de reducción de la demora al forrajeo. Según esta hipótesis los organismos minimizan las demoras a reforzamiento. Esto debería poder extenderse a ambientes distribuidos en parches.

En este caso se podría esperar que los organismos minimicen el esfuerzo en lugar de la demora dada la naturaleza de la tarea. Un organismo debería permanecer en un parche cuando el número esperado de respuestas en él hasta el siguiente reforzador es menor que el número esperado de respuestas hasta encontrar un nuevo parche y recibir el primer reforzador en él.

Este modelo de {\itshape menor esfuerzo} hace predicciones distintas y menos acertadas que las predicciones de la teoría de forrajeo óptimo. Mientras la teoría de forrajeo óptimo subestimó el grado de explotación de parches, el método de menor esfuerzo lo sobrestimó.

Los resultados de estos experimentos apoyan la mayoría de las predicciones cualitativas de la teoría de forrajeo óptimo con respecto a elección y explotación de parches. Un fracaso está en el hallazgo de que las variaciones en la tasa de encuentros con el parche rico no era necesaria para modular la conducta. Las predicciones sobre las estrategias de elección de parches fueron precisas, pero las de el grado de explotación subestimaron consistentemente el nivel observado.


\end{document}
