\documentclass[a4paper,12pt]{article}
\usepackage[utf8]{inputenc}
\usepackage[T1]{fontenc}
\usepackage[spanish]{babel}
\usepackage{csquotes}
\usepackage{anysize}
\usepackage{graphicx}
\usepackage{hyperref}
%\usepackage{amsfonts}
%\usepackage{tikz}
%\usepackage{amsmath}
\marginsize{25mm}{25mm}{25mm}{25mm}

\title{Signaled reinforcement: Effects of signal reliability on choice between signaled and unsignaled alternatives}
\author{Stephanie Gomes-Ng \and Atena C. Macababbad \and John Y.H. Bai \and Darren Baharrizki \and Douglas Elliffe \and Sarah Cowie}
\date{2020}

\begin{document}
{\scshape\bfseries \maketitle}

Se ha encontrado que la presencia de estímulos discriminativos disminuye la tasa de respuesta durante programas de reforzamiento: las respuestas incrementan exclusivamente en presencia de los $E^{D}$ y disminuyen el resto del tiempo, resultando en una reducción neta de la tasa de respuestas.

Se investigó si, en programas concurrentes, la tasa de respuestas ante una alternativa tiene un efecto sobre la tasa de la alternativa complementaria, o si acaso es solamente la tasa relativa de {\itshape reforzamiento} lo que determina la tasa de respuestas de la alternativa complementaria.

Se utilizaron palomas en un procedimiento en el cual había dos programas vigentes concurrentemente: un VI 150s en una tecla (tecla no señalada) y un un VI 30s en otra (tecla señalada). Las teclas podían encenderse en amarillo o en rojo, y se manipuló la probabilidad de reforzamiento en presencia de los estímulos, es decir, su confiabilidad como estímulos discriminativos.

En la primera condición ambas teclas se encendieron en amarillo. En las condiciones posteriores la tecla señalada fue encendida ocasionalmente en color rojo, y responder en ella llevaba a reforzamiento o la hacía regresar a color amarillo. Se varió entre condiciones la probabilidad de que la tecla entregara reforzamiento entre 1 y .2, es decir, la confiabilidad de la tecla encendida en rojo como predictora de reforzamiento.

Aunque se encontró el efecto esperado de disminución en la tasa de respuestas ante la tecla señalada según incrementó la confiabilidad del color rojo como predictor, no se encontró un efecto sobre la tasa de respuestas ante la tecla complementaria. Es decir, las tasas relativas de respuesta continuaron igualando a las tasas relativas de reforzamiento en ausencia de la señal.


\end{document}
