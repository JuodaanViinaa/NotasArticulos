\documentclass[a4paper,12pt]{article}
\usepackage[utf8]{inputenc}
\usepackage[T1]{fontenc}
\usepackage[spanish]{babel}
\spanishdecimal{.}
\usepackage{csquotes}
\usepackage{anysize}
\usepackage{graphicx}
\marginsize{25mm}{25mm}{25mm}{25mm}

\title{Does a negative discriminative stimulus function as a punishing consequence?}
\author{Vikki J. Bland \and Sarah Cowie \and Douglas Elliffe \and Christopher A. Podlesnik}
\date{2018}

\begin{document}
{\scshape\bfseries \maketitle}

Los tratamientos de reforzamiento diferencial de conducta alternativa (DRA), aunque efectivos, tienen desventajas, como el no reducir inmediatamente la conducta problema, o resultar en incrementos de la misma si el tratamiento se ve comprometido.
Una alternativa es el castigo, pero es controvertido dados los efectos secundarios que ocasiona.
Una propuesta es encontrar castigos que disminuyan la conducta sin generar respuestas emocionales condicionadas.
Un ejemplo de tal castigo es un estímulo discriminativo relacionado con un evento desfavorable (como la presentación de un choque o la ausencia de reforzamiento).


\end{document}
