\documentclass[a4paper,12pt]{article}
\usepackage[utf8]{inputenc}
\usepackage[T1]{fontenc}
\usepackage[spanish]{babel}
\usepackage{csquotes}
\usepackage{anysize}
\usepackage{graphicx}
\usepackage{hyperref}
%\usepackage{amsfonts}
%\usepackage{tikz}
%\usepackage{amsmath}
\marginsize{25mm}{25mm}{25mm}{25mm}

\title{Timing or counting? Control by Contingency Reversals at Fixed Times or Number of Responses}
\author{Michael Davison \and Sarah Cowie}
\date{2019}

\begin{document}
{\scshape\bfseries \maketitle}

La conducta puede ser controlada tanto por {\itshape timing} del tiempo pasado desde un evento marcador como por conteo de eventos ocurridos desde un marcador. Se ha buscado con bastante éxito disociar ambos procesos dado que suelen correlacionar, y se ha encontrado que el {\itshape timing} suele ser preferido por los organismos para el control de estímulos. Se ha sugerido incluso que el conteo no es natural comparado con el {\itshape timing} en animales no-humanos, y que solo será utilizado en situaciones extremas donde no haya otra señal confiable que se pueda utilizar. El {\itshape timing} controla mejor la conducta incluso cuando el conteo es más confiable.

Aun así, el control de estímulos puede estar dividido entre ambos procesos. De hecho, quizá el control por conteo puede entenderse en los mismos términos que se entiende el control por {\itshape timing}: el reforzamiento parece  afectar a ambos procesos de formas similares, y las discriminaciones por número se conforman a la let generalizada de Weber.

El control por tiempo depende de
\begin{itemize}
    \item Detección de un marcador en el que inicia un periodo.
    \item Control de estímulos por el tiempo transcurrido.
    \item Una contingencia detectable respuesta-reforzador en un tiempo consistente.
\end{itemize}

Es decir, el animal debe detectar el marcador, debe haber un cambio detectable en las contingencias, y se debe detectar la relación temporal consistente entre el marcador y el cambio en las contingencias.

El control por conteo requiere de los mismos componentes, pero el tiempo es sustituido por un conteo numérico.

No es claro si el control puede cambiar entre estímulos temporales y numéricos mediante manipulaciones procedimentales. ¿Se puede entender al control por tiempo y número en los mismos términos que el control por otros estímulos, como colores y luces?

Estos experimentos pretenden probar si las palomas cuentan o estiman tiempo en un procedimiento de reversión basada en tiempo, y si el control puede ser modificado por manipulaciones que hacen que una dimensión (tiempo o número) sea más confiable que la otra.

Cada entrega de reforzador marcaba el inicio de un ensayo en el cual ocurriría una reversión en las tasas de reforzamiento (1:9 o 9:1) entre dos teclas. En algunas condiciones el marcador señalaba un tiempo fijo hasta la reversión; en otras, una cantidad fija de respuestas hasta la reversión.

Hay una correlación entre tiempo y cantidad de respuestas, pero una de las dimensiones siempre señalaría más claramente el cambio en las contingencias. Si las palomas pueden contar y estimar tiempo igual de bien, deberían cambiar su estrategia en función de la dimensión que sea más confiable como predictora.

Dado que los resultados de la fase 1 favorecían al {\itshape timing}, en la fase 2 se agregaron {\itshape blackouts} que ocurrían entre respuestas para disociar al número de respuestas emitidas del tiempo pasado respondiendo.

\section{Method}

Se utilizaron cinco palomas con experiencia en procedimientos similares en cajas operantes con tres teclas.

Las sesiones comenzaban con las dos teclas laterales encendidas en amarillo. Cada ensayo comenzaba con la entrega de un reforzador en un programa de IV. En la primera parte del ensayo una tecla tenía 9 veces más probabilidades que la otra de entregar comida, pero la probabilidad se revertía tras un punto fijo en el ensayo. La localización de ese punto variaba entre condiciones según el número de respuestas realizadas o el tiempo pasado desde el inicio del ensayo. Cada ensayo terminaba con la obtención del reforzador y el siguiente ensayo comenzaba en cuanto terminaba el reforzador. Así, cada ensayo marcaba el inicio de un período tras el cual ocurriría la reversión.

En la fase 1 había reversiones basadas en tiempo y en número de respuestas. Las condiciones distintas aseguraban que una dimensión era mejor predictora de la reversión, pero la correlación entre ambas indicaba que aun la menos confiable seguiría siendo predictora con cierta certidumbre. Para ello se aplicó la fase 2, en la cual se insertaron {\itshape blackouts} después de cada picotazo. Los {\itshape blackouts} tenían la misma longitud dentro de un ensayo, pero variaban entre ensayos. Esa variación debilitaba la correlación entre tiempo y número y hacía ala dimensión relevante mejor predictora.

Los reforzadores se hacían disponibles con un IV 45s consultando una puerta de probabilidad de 0.22 cada 1 segundo. Cuando la puerta asignaba un reforzado, una segunda puerta era ajustada en .1 o .9, dependiendo de la condición y si la reversión lo asignaba a la tecla izquierda o derecha. Cuando la localización era elegida, la siguiente respuesta entregaba la comida. Tras un cierto tiempo o número de respuestas las probabilidades usadas para asignar el reforzador eran revertidas entre teclas. Si un reforzador era asignado en una tecla pero no era tomado antes de la reversión, se reasignaba a la otra tecla tras la reversión. En el caso del conteo se utilizaba el mismo procedimiento, pero las reversiones ocurrían tras un número fijo de respuestas.
Las sesiones terminaban tras 60 minutos o 60 reforzadores.

\section{Results}

Se ajustó un modelo ojival a los datos de las palomas. Los resultados apoyan la conclusión de que las palomas estiman el tiempo sin importar si las reversiones ocurrían con base en tiempo o en respuestas. La fase 2 pretendía apoyar esta conclusión de forma experimental, más que con análisis de datos al disminuir la correlación entre el tiempo y la cantidad de respuestas.

En ambas fases se encontró un dominio del control por tiempo sin importar si las reversiones eran basadas en conteo o en tiempo.

Se ha encontrado que las habilidades de conteo difieren entre individuos: se ha encontrado que algunas palomas pueden contar hasta 8 eventos con precisión, mientras que otras difícilmente cuentan más allá de 3. Quizá por esta razón la estimación temporal es preferida. Aunque en este procedimiento la preferencia por el {\itshape timing} emergió a pesar de utilizar números pequeños y hacer más confiable al conteo.

Se ha sugerido que hay un mismo proceso subyacente para el conteo y el {\itshape timing} debido a que comparten propiedades psicofísicas escalares.

Es posible que el control por conteo no emergiera ni siquiera en situaciones en que el número de respuestas era pequeño debido a que en su entrenamiento inicial los conteos eran demasiado grandes, y el control por tiempo simplemente se mantuvo en condiciones posteriores. También es posible que la experiencia previa de las palomas en el procedimiento pero no en este experimento amplificara cualquier preferencia pequeña que pudiesen tener por el {\itshape timing}.

Se sugiere que hay un cierto acumulador que se utiliza en ambos procesos ({\itshape timing} y conteo), pero en dos modos: un modo continuo para el {\itshape timing} y uno de pulsos para el conteo. Quizá para algunas especies es más difícil utilizar el modo de pulso dadas limitaciones de memoria.

Aunque los animales de este experimento mostraron preferencia por el {\itshape timing} por encima del conteo, la evidencia más general indica que los animales pueden hacer ambas cosas, quizá con un mecanismo común. La conducta será controlada por una cosa o la otra dependiendo de la historia de aprendizaje y las frecuencias relativas de reforzadores obtenidos por utilizar cada dimensión.

Una teoría de conducta bajo el control de tiempo en procedimientos comunes (como FOPP y el procedimiento de pico) está incompleta sin una teoría efectiva de reforzamiento. Esto no significa que se requieran reforzadores para aprender las relaciones entre eventos, sino que las variaciones en reforzadores en procedimientos de {\itshape timing} operante actúan de un modo similar a las que actúan en eventos marcadores, pero con una dimensión hedónica adicional que determina la localización de la respuesta y permite la medición del control por tiempo transcurrido.


\end{document}
