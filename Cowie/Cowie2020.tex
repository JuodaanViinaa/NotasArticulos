\documentclass[a4paper,12pt]{article}
\usepackage[utf8]{inputenc}
\usepackage[T1]{fontenc}
\usepackage[spanish]{babel}
\usepackage{csquotes}
\usepackage{anysize}
\usepackage{graphicx}
\usepackage{hyperref}
%\usepackage{amsfonts}
%\usepackage{tikz}
%\usepackage{amsmath}
\marginsize{25mm}{25mm}{25mm}{25mm}

\title{Being there on time: Reinforcer effects on timing and locating}
\author{Sarah Cowie \and Michael Davison}
\date{2020}

\begin{document}
{\scshape\bfseries \maketitle}

La correlación entre eventos ambientales ejerce control sobre la conducta suponiendo que las condiciones futuras señaladas por un {\itshape evento marcador} sean relevantes en términos de los {\itshape affordances}, disposiciones, e historia filogenética del organismo. Eso se conoce como {\itshape control de estímulo}.

Las correlaciones entre eventos ocurren en una escala temporal dada, y su efectividad para ejercer control de estímulos depende de la habilidad de los organismos para recordar los eventos en la escala temporal relevante: no se puede aprender una correlación si un evento se olvida antes de que el siguiente tenga lugar.

El recuerdo impreciso---generalización---en tiempo o espacio distorsionará la relación discriminada entre eventos y conducta.

El control por correlaciones depende también de los reforzadores que forman parte de ellas, por ejemplo, cuando la disponibilidad de reforzadores cambia entre localizaciones en un tiempo dado, las diferencias en el porcentaje de reforzadores obtenidos antes y después del tiempo causan diferencias confiables en el patrón de elección a través del tiempo a pesar de que la correlación entre localización y reforzador permanezca igual.


\end{document}
