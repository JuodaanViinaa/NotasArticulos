\documentclass[a4paper,12pt]{article}
\usepackage[utf8]{inputenc}
\usepackage[T1]{fontenc}
\usepackage[spanish]{babel}
\usepackage{csquotes}
\usepackage{anysize}
\usepackage{graphicx}
\usepackage{hyperref}
%\usepackage{amsfonts}
%\usepackage{tikz}
%\usepackage{amsmath}
\marginsize{25mm}{25mm}{25mm}{25mm}

\title{Some weaknesses of a response-strength account of reinforcer effects}
\author{Sarah Cowie}
\date{2020}

\begin{document}
{\scshape\bfseries \maketitle}

La ley del efecto de Skinner dice solo que los reforzadores fortalecen la conducta, pero no predice si el fortalecimiento será específico a una respuesta, un patrón de respuestas o una clase, ni cuánto fortalecimiento ocurrirá o cuánto durará.

Se ha sugerido que el concepto de {\itshape fuerza de la respuesta} ya llegó al fin de su vida útil.

Hay tres preguntas que una aproximación de {\itshape fuerza de la respuesta} no ha podido responder:
\begin{itemize}
    \item Qué se fortalece
    \item Por cuánto tiempo
    \item En qué medida
\end{itemize}

La aproximación presentada aquí integra ideas de aproximaciones que no requieren del concepto de fuerza. Se enfoca en las correlaciones extendidas en el tiempo y los {\itshape affordances} y disposiciones de los organismos.

\section{What gets strengthened?}

La fuerza no puede medirse, sino que se infiere de la conducta. ¿Qué dimensión conductual debería medirse para medir la fuerza? Existe la frecuencia, la probabilidad, tiempo entre respuestas, latencia, duración, intensidad, etc.
El problema con la fuerza es que tiene múltiples interpretaciones, y al ser así corremos el riesgo de tener una ciencia que no es reductible a lo observable, lo que puede ser cercano a un constructo hipotético.

Además, una explicación de fuerza no permite saber {\itshape a priori} qué clase de conducta será fortalecida. Programas de reforzamiento distintos llevan a patrones de respuesta drásticamente diferentes, y estas respuestas no son predecibles con una explicación de fuerza de la respuesta.
Por ejemplo, habrá más respuestas ante un programa VI más delgado que ante uno FI más rico, como si menos reforzadores incrementasen más la fuerza. Por otro lado, cuando reforzadores de dos conductas son entregados bajo el mismo tipo de programa, pero una conducta produce reforzadores antes que la otra, un reforzador {\itshape reducirá} la probabilidad de la conducta a la que sigue.
Del mismo modo, un castigo incrementará la probabilidad de la conducta a la que sigue si su ocurrencia correlaciona con eventos apetitivos posteriores para la misma conducta.

Decir que se trata de instancias donde se fortalecen {\itshape clases} de conducta ayuda en poco: aunque experimentalmente tratamos con respuestas discretas, no debe perderse de vista que es solo por conveniencia, pero en la realidad el flujo conductual es altamente variable. Así, la noción de fuerza no nos da una manera de predecir qué {\itshape clase} de respuesta será fortalecida. Explicaciones {\itshape ad hoc} solamente vuelven a la teoría infalseable.

Se puede decir que los efectos de los reforzadores son por su función dual como fortalecedores y estímulos discriminativos, pero esta aproximación viene al costo de la parsimonia y solo lleva a más preguntas: ¿en qué condiciones el poder fortalecedor de un reforzador triunfa sobre su poder como estímulo discriminativo?

Podría argumentarse que la conducta está bajo un amplio abanico de influencias---motivación, sensibilización, competencia---, pero si el fortalecimiento no permite entender o predecir los efectos de los eventos cuyo mero nombre determina, ¿qué propósito tiene?

Se podría decir que los reforzadores fortalecen el control de estímulos más que la conducta mediante sus regularidades temporales. Sin embargo, los organismos pueden detectar las correlaciones confiables en el ambiente sin la necesidad de reforzadores. Por ejemplo, ratas entrenadas para responder en la presencia de olores novedosos aprenden rápidamente a elegir esos olores nuevos por encima de otros ya percibidos en la misma sesión experimental. Es decir, la conducta es controlada por un estímulo nunca antes experimentado ni reforzado. Así, la fuerza mediante reforzamiento no es necesaria ni suficiente para explicar la conducta, sin importar que se le atribuya la fuerza a la conducta o al contexto de estímulos en el que sucede.

\section{How much strengthening occurs?}

Algunos reforzadores---más grandes, inmediatos, de mejor calidad---son más efectivos que otros, pero no todas las diferencias en efectividad se relacionan con diferencias que tendrían sentido lógico. Por ejemplo, en el efecto de {\itshape differential outcomes} el aprendizaje es más rápido y preciso cuando dos o más conductas distintas llevan a reforzadores distintos, es decir, los mismos reforzadores fortalecen más la conducta cuando correlacionan con una conducta específica que cuando no, aun si la motivación y experiencia permanecen constantes. Una aproximación de fuerza de respuesta no permite explicar ni predecir este fenómeno.

De igual forma, reforzadores relevantes para la conducta que los produce son más efectivos. Por ejemplo, el pájaro mielero chillón aprende contingencias de {\itshape win-stay} más rápidamente cuando la recompensa son invertebrados que cuando es néctar, quizá debido a que en la naturaleza las fuentes de néctar son limitadas, de modo que la estrategia de {\itshape win-switch} es más adecuada para obtener reforzadores futuros, lo cual no sucede con los invertebrados. Aquí, la historia del organismo---personal y filogenética--- es importante para determinar los efectos de los reforzadores.

\section{How does strength change over time?}

La aproximación de fuerza de la respuesta indica que cada reforzador debería hacer la conducta a la que sigue más probable hasta un máximo dictado por la naturaleza de la respuesta y los {\itshape affordances} del animal. Como ejemplo, Skinner mostró que un único reforzador podía generar un incremento rápido y duradero en el comportamiento de una rata, pero que es un efecto altamente variable.

¿Qué haría la rata si su conducta dependiera de la probabilidad de que su conducta lleve a condiciones futuras deseable, y no de la ``fuerza''? La generalización de la experiencia en un entorno distinto---una fase en la cual la palanca a veces estaba disponible pero no producía consecuencias, y a veces la comida estaba disponible pero no contingentemente---permitiría a la rata comportarse de acuerdo con lo que es probable que produzca condiciones deseables. Para una rata hambrienta lo deseable es la comida. Entre las pocas conductas disponibles en la caja operante---acicalarse, dormir, etc---, pocas cosas podrían haber sido seguidas por comida, pero con el tiempo la ausencia de covarianza se hace clara y la coincidencia deja de tener control. Aquello que más consistentemente produce comida son las presiones a la palanca. Si hay, por ejemplo, 15 presiones no reforzadas y una reforzada, eso sigue siendo superior a todas las ocasiones en que la rata se ha acicalado y no ha recibido comida, por lo tanto la correlación de la presión a la palanca con la entrega de comida resulta ganadora. Así, la elección de la conducta de la rata es obvia sin apelar a la fuerza.

En condiciones más controladas se ha mostrado que la fuerza de respuesta explica poco los efectos de un solo reforzador en la conducta: cuando una respuesta había sido regularmente seguida por reforzamiento en la historia reciente del organismo, y otra respuesta nunca antes reforzada producía un reforzador, la respuesta recién reforzada incrementaba en frecuencia a expensas de la respuesta con la historia más extensa de reforzamiento. Pero en ausencia de más reforzadores, la respuesta recién reforzada dejaba de ocurrir rápidamente y la conducta regresaba a la otra respuesta, como si la ausencia de reforzadores hiciese a esa respuesta más {\itshape fuerte}.

Sobre la variabilidad de respuestas, el reforzamiento repetido debería eventualmente hacer que la conducta se vuelva estática y uniforme, pero los reforzadores casi nunca erradican la variabilidad en las respuestas: los reforzadores repetidos ocasionan desplazamiento en el rango de conductas emitidas. La variabilidad no cambia con la fuerza de las respuestas, sino con la contingencia y la predictabilidad---cierto grado de variabilidad es adaptativo.

En ocasiones la conducta cambia sin ningún cambio en el reforzamiento. Esto sucede cuando hay un cambio regular y predecible en las contingencias durante las sesiones. Por ejemplo, los humanos pierden el tiempo al cambiar de una actividad más preferida a una menos preferida, pero no al revés. Este tipo de cambios en ausencia de reforzamiento parecen guiados por las condiciones futuras probables. Quizá, entonces, la experiencia pasada no determina la fuerza de las respuestas, sino que permite al organismo generalizar al presente y predecir y responder al futuro. Argumentar que los reforzadores fortalecen el control de estímulos no ayuda a entender los cambios que ocurren antes de que cambien los estímulos o reforzadores, o a explicar conductas que cambian en ausencia de cambios inminentes en las condiciones de estímulos. Se podría decir que la gimnasia mental requerida para mantener la fuerza como una explicación socava la orgullosa tradición de la parsimonia en el análisis de la conducta.

En resumen, la fuerza de la respuesta añade poco a nuestra habilidad para predecir y controlar la conducta. Podemos mantenerla en nuestras explicaciones y pasar nuestro enfoque a entender los efectos del control de estímulos, historia filogenética, motivación, etc., sobre la conducta; o eliminar la fuerza de la respuesta de nuestras explicaciones por completo.

\section{An approach without strength}

Esta aproximación será casi asociativa. Bajo ella, las regularidades en el entorno controlan la conducta de un modo que nos permite comportarnos de acuerdo con lo que es probable de ocurrir en el futuro. El control por regularidades no depende de los reforzadores.

Si rechazamos la fuerza de la respuesta, entonces los reforzadores pierden su estatus como estímulos especiales, y se vuelven meramente discriminativos. Denotan eventos valiosos para el organismo sin función especial. El poder de los reforzadores vendría entonces de su potencial para satisfacer una necesidad actual del organismo, o de su señalamiento de un futuro en que las necesidades o deseos serán satisfechos.

Las correlaciones que ejercen control sobre la conducta como resultado de la experiencia pueden pensarse como lo que Baum llama {\itshape conditional inducers}.

Como la correlación, la contigüidad da un grado de información sobre la estructura del ambiente y puede ser una fuente de control. En ausencia de correlación, cuando el entorno es novedoso, la contigüidad provee información no-redundante de la estructura. Con la exposición repetida el control pasa de la contigüidad a la correlación.

Los eventos señalizadores no necesitan estar físicamente presentes en el ambiente: la memoria le permite a los organismos traer señalizadores del pasado hacia la situación actual.

En un marco sin fuerza de respuesta, las conductas operantes y respondientes difieren solamente en términos de la secuencia de eventos que controla la conducta: no son controlados por mecanismos fundamentalmente diferentes.

La habilidad de un señalizador para controlar la conducta depende no solo de lo bien que predice las condiciones futuras, sino también de la habilidad de los organismos para {\itshape detectar} su relación a tales condiciones. Además, una relación incongruente con la historia personal o filogenética puede ser más lenta de aprender.

Que un señalizador controle o no la conducta depende del futuro que señala y del valor de ese futuro para el organismo en el presente. Cuando múltiples señalizadores señalen a diferentes futuros cuyo valor depende de la disposición del organismo, entonces esta disposición determinará qué señalizador ejerce control. En ocasiones la competencia entre señalizadores es inevitable y se obtienen resultados subóptimos, por ejemplo, un estímulo pareado con comida ocasiona conducta de búsqueda de comida aun cuando esas conductas prolongan el tiempo hasta la siguiente entrega.

\section{Conclusion}

Aunque las ideas de control por correlación no son nuevas, lo novedoso del artículo es el enfoque en la ausencia de fuerza de respuesta.

La fuerza ha sido una metáfora útil como regla de dedo que ha permitido predecir con cierta precisión los cambios en la conducta que se podrán ver en varias situaciones, pero aunque nos permita hacer predicciones con cierto éxito, no nos permite entender por qué en ocasiones fallamos en obtener los cambios deseados en la conducta.

Quizá es momento de que los conductistas desechen el fortalecimiento en favor de la exploración de otros mecanismos.


\end{document}
