\documentclass[a4paper,12pt]{article}
\usepackage[utf8]{inputenc}
\usepackage[T1]{fontenc}
\usepackage[spanish]{babel}
\usepackage{csquotes}
\usepackage{anysize}
\usepackage{graphicx}
\usepackage{hyperref}
%\usepackage{amsfonts}
%\usepackage{tikz}
%\usepackage{amsmath}
\marginsize{25mm}{25mm}{25mm}{25mm}

\title{Some weaknesses of a response-strength account of reinforcer effects}
\author{Sarah Cowie}
\date{2020}

\begin{document}
{\scshape\bfseries \maketitle}

La ley del efecto de Skinner dice solo que los reforzadores fortalecen la conducta, pero no predice si el fortalecimiento será específico a una respuesta, un patrón de respuestas o una clase, ni cuánto fortalecimiento ocurrirá o cuánto durará.

Se ha sugerido que el concepto de {\itshape fuerza de la respuesta} ya llegó al fin de su vida útil.

Hay tres preguntas que una aproximación de {\itshape fuerza de la respuesta} no ha podido responder:
\begin{itemize}
    \item Qué se fortalece
    \item Por cuánto tiempo
    \item En qué medida
\end{itemize}

La aproximación presentada aquí integra ideas de aproximaciones que no requieren del concepto de fuerza. Se enfoca en las correlaciones extendidas en el tiempo y los {\itshape affordances} y disposiciones de los organismos.

\section{What gets strengthened?}

La fuerza no puede medirse, sino que se infiere de la conducta. ¿Qué dimensión conductual debería medirse para medir la fuerza? Existe la frecuencia, la probabilidad, tiempo entre respuestas, latencia, duración, intensidad, etc.
El problema con la fuerza es que tiene múltiples interpretaciones, y al ser así corremos el riesgo de tener una ciencia que no es reducible a lo observable, lo que puede ser cercano a un constructo hipotético.

Además, una explicación de fuerza no permite saber {\itshape a priori} qué clase de conducta será fortalecida. Programas de reforzamiento distintos llevan a patrones de respuesta drásticamente diferentes, y estas respuestas no son predecibles con una explicación de fuerza de la respuesta.
Por ejemplo, habrá más respuestas ante un programa VI más delgado que ante uno FI más rico, como si menos reforzadores incrementasen más la fuerza. Por otro lado, cuando reforzadores de dos conductas son entregados bajo el mismo tipo de programa, pero una conducta produce reforzadores antes que la otra, un reforzador {\itshape reducirá} la probabilidad de la conducta a la que sigue.
Del mismo modo, un castigo incrementará la probabilidad de la conducta a la que sigue si su ocurrencia correlaciona con eventos apetitivos posteriores para la misma conducta.

Decir que se trata de instancias donde se fortalecen {\itshape clases} de conducta ayuda en poco: aunque experimentalmente tratamos con respuestas discretas, no debe perderse de vista que es solo por conveniencia, pero en la realidad el flujo conductual es altamente variable. Así, la noción de fuerza no nos da una manera de predecir qué {\itshape clase} de respuesta será fortalecida.


\end{document}
