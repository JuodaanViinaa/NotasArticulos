\documentclass[a4paper,12pt]{article}
\usepackage[utf8]{inputenc}
\usepackage[T1]{fontenc}
\usepackage[spanish]{babel}
\usepackage{csquotes}
\usepackage{anysize}
\usepackage{graphicx}
\usepackage{hyperref}
\marginsize{25mm}{25mm}{25mm}{25mm}

\title{Choosing a future from a murky past: A generalization-based model of behavior}
\author{Sarah Cowie \and Michael Davison}
\date{2022}

\begin{document}
{\scshape\bfseries \maketitle}

\section{Introduction}

La organización de los eventos en orden define la relación de estímulos, conductas y consecuencias. El orden permite recordar el tiempo relativo de los eventos como un componente de la estructura del entorno. La memoria de la estructura es lo que controla el comportamiento, es decir, el futuro existe solo hasta el punto en que el pasado puede recordarse para crearlo. Recordar permite guiarnos hacia metas.

Cuantos más grados de libertad en el entorno, más importante es recordar para guiar el comportamiento.

La percepción y el recordar son imprecisos por naturaleza, y por lo tanto ponen límites en el grado en que se puede desarrollar control. Incluso en entornos altamente estables la elección no sigue perfectamente a la estructura del ambiente, {\itshape e.g.,} sub-igualación. Estas imprecisiones son más evidentes cuando las respuestas son similares entre sí o cuando eventos relacionados y ordenados están separados en el tiempo.

Sin importar la estructura real, lo que controla el comportamiento es la estructura experimentada y recordada. El proceso por el que la percepción y la memoria causan que la estructura percibida sea distinta de la real se denomina {\itshape generalización}.

La misma generalización nubla la estructura percibida y permite responder a estímulos que no son idénticos a los del pasado, es decir, no será necesariamente guiada por un proceso de exploración adaptativo.

No se puede saber en qué momento termina la percepción y comienza la memoria, así que se pretende explorar solo cómo la generalización que resulta en recuerdo imperfecto llega a estar bajo el control del ambiente.

La aproximación actual, que la generalización que permite navegar en entornos con estructuras que no son idénticas es igual a la generalización en entornos cuya estructura sí es idéntica, tiene la ventaja de la parsimonia.

Se utilizará una aproximación cuantitativa dado que permite formalizar las relaciones entre eventos sin ambigüedad.

\section{The generalization-over-dimensions model}

Esta aproximación usa reforzadores pasados para predecir la conducta futura---una traducción directa en lugar de basarse en propiedades estimadas de los estímulos encontrados. Los reforzadores obtenidos son desplazados a través de dimensiones de los estímulos para modelar el proceso de generalización. La tasa de reforzadores después de la generalización determina la estructura detectada. Se asume que la conducta iguala estrictamente a la tasa de reforzadores {\itshape percibida}.

\vspace{2mm}
{\centering\fbox{\parbox{0.95\textwidth}{%
            Supuestos del modelo:

            \begin{itemize}
                \item Reforzadores obtenidos ante un estímulo se generalizan a otros en esa dimensión según su distancia y la cantidad de reforzadores obtenidos.
                \item La generalización ocurre en todas las dimensiones relevantes.
                \item Los reforzadores aparentes ante cualquier combinación estímulo-respuesta son la suma de todos los reforzadores generalizados actuando sobre ella.
                \item La elección es igual a la suma de todos los reforzadores aparentes para las dos alternativas de una condición de estímulo.
            \end{itemize}
}}}
\vspace{2mm}

Si hay una diferencia constante entre cualquier par de estímulos, se asume que la generalización mutua es constante también. Así, si $m$ es la probabilidad de detectar correctamente los dos estímulos:
\[
    \log\frac{B_1}{B_2} = \log\frac{R'_1}{R'_2} = \log \left(\frac{
            mR_1 + (1 - m) R_2
        }{
            mR_2 + (1 - m) R_1
    }\right) + \log c
,\]
de modo que una proporción constante de reforzadores $R$ obtenida para cada respuesta 1 y 2 es reasignada a la respuesta alternativa. $B$ se refiere a los conteos de respuestas a ambas alternativas, y $R'$ a los números aparentes de reforzadores tras la generalización. $\log c$ es el sesgo en las respuestas como en igualación generalizada. Se asume en este caso que la generalización entre $B_1$ y $B_2$ es simétrica.

En dimensiones continuas como el tiempo se usa un proceso de redistribución Gaussiano donde la desviación estándar varía dependiendo de la posición del reforzador a lo largo de una dimensión.

Esta aproximación transforma la estructura experimentada en una recordada para predecir la conducta mediante la redistribución de reforzadores a través de las dimensiones relevantes.

Esta aproximación es similar a los gradientes excitatorios e inhibitorios de Spence, con la diferencia de que la generalización resulta de la dispersión de reforzadores a través de contextos similares, y no de la dispersión de la excitación o inhibición.

Se utiliza el mismo principio para generalizar a través de todas las dimensiones relevantes, cuales quiera que sean, lo que resulta parsimonioso. Los mismos supuestos se usan ya sea que la generalización sea capturada por la desviación de una distribución Gaussiana o por una proporción.

Los reforzadores deben tener el mayor efecto en el contexto en el cual realmente ocurrieron, e impacto disminuido al alejarse de él. El grado de generalización depende de la desviación estándar de la distribución Gaussian: mayor desviación implica mayor generalización.

Dado que la cantidad de reforzadores dada es la misma, el área bajo a función de densidad de probabilidad que representa el proceso de generalización permanece igual, es decir, según incrementa la desviación estándar, la altura del pico de la distribución Gaussiana disminuye. Esto puede cambiar al integrar los reforzadores de fondo que se generalicen a conducta frontal, o reforzadores frontales que se generalicen a conducta de fondo, pero es complicado y aun no se investiga.

Se asume que dado el proceso de generalización, según incrementa la tasa de reforzamiento también debería incrementar el traslape entre las distribuciones, por lo que la generalización deberá ser más amplia con tasas de reforzamiento más altas. Aunque es contraintuitivo, se ha encontrado evidencia apoyando esta noción.

Cuando el tiempo es una dimensión relevante, se ha encontrado que la generalización puede modelarse con una desviación que incrementa según aumenta el tiempo desde un evento marcador, lo que refleja condiciones incrementalmente menos claras en tiempos más tardíos en el ensayo. Esto debido a que la evidencia indica que eventos más distantes en el tiempo son más difíciles de diferenciar, además de que para duraciones mayores hay mayor probabilidad de que el tiempo sea llenado con otros eventos similares---y, por lo tanto, generalizables---que causen interferencia.

Sumando las últimas ideas, se obtiene que variaciones en el número de reforzadores obtenidos en tiempos más tempranos o más tardíos tendrá efectos asimétricos en la elección.

El modelo captura bien situaciones en que el tiempo es una variable relevante, al igual que situaciones de generalización en dimensiones no-temporales. La aproximación parece ser buena para transformar reforzadores experimentados a través del tiempo en un output organísmico.

\subsection{Generalization: where to from here?}

El modelo parece dar cuanta del comportamiento de ratas, palomas e incluso humanos. Aunque esta en progreso, parece robusto. Se señalan algunas direcciones posibles para la aproximación de {\itshape generalization-over-dimensions}.

\subsection{Generalization across wider ranges of events}

Es complicado modelar las relaciones entre más de dos categorías de estímulos, tasas de respuesta, o respuestas. Un caso particular es el de las relaciones entre conductas y reforzadores planeados y extraños. Una manera de pensar en ello es cómo las relaciones explícitas, determinadas por el experimentador, estímulo-respuesta-consecuencia, son discriminables de las relaciones extrañas.

Es necesaria más investigación sobre arreglos de 3 o más relaciones estímulo-conducta-consecuencia. Sería útil simular los efectos de reforzadores de fondo para una conducta altamente discriminable.

\subsection{Generalization from multiple pasts: which past, which future?}

El pasado siempre incluye más de una situación de estímulo. ¿Cómo se decide qué experiencias se generalizan? Una aproximación es redistribuir algunos de los reforzadores discriminados obtenidos en cada situación (en cada, tiempo, a cada respuesta, en cada contexto) a la otra situación de estímulo.

\subsection{Generalization across a more extended past -- the effect of learning history}

La experiencia pasada dirige la conducta en el presente hacia eventos valiosos futuros. Se podría suponer que eventos más lejanos en el pasado se generalizarán menos hacia el presente, pero el mero paso del tiempo no parece ser explicación suficiente: en renovación, las respuestas que fueron reforzadas más lejos en el pasado, y extinguidas más recientemente, ocurrirán de nuevo cuando el presente comience a parecerse al contexto pasado más distante en el que la conducta produjo reforzadores, a pesar de la experiencia más reciente en extinción. Así, la similitud entre pasado y presente puede ser crítica para determinar el grado en que distintos pasados se generalizan para controlar la conducta.

La pregunta es {\itshape qué pasado o pasados} se generalizan más fuertemente al presente. Información al respecto puede venir de observar en qué contextos un animal seguirá respondiendo dad su experiencia previa.

En situaciones en que las contingencias de reforzamiento cambian sin que lo hagan también las condiciones de señalización, los organismos comienzan a adaptarse más rápidamente a los cambios. Parece ser la experiencia a largo plazo con estas relaciones la que determina qué elementos del complejo de estímulos presente pasa a controlar la conducta.

El orden de los estímulos también puede controlar la conducta, por ejemplo, cuando componentes ricos y pobres de un programa múltiple siguen una secuencia estricta, la conducta en los momentos finales de un componente será afectada por lo que un organismo encontrará en el siguiente.

Todo esto sugiere que la experiencia pasada se mapea en la conducta presente de un modo que depende no solo de lo que sucedió en la situación pasada, sino también de lo que ésta señaló sobre la condiciones de reforzamiento futuras. Una aproximación que traduzca la experiencia pasada en conducta presente enfocada en el futuro tendrá que incorporar la forma en que experiencias de un pasado más extendido gobiernan el grado en que distintos estímulos controlan la conducta.

\subsection{Generalization to unfamiliar presents}

Cuanto más se parece el presente a una situación pasada, más se aproximará la conducta presente a la pasada. Sin embargo, en ocasiones la conducta puede generalizarse a situaciones nunca antes experimentadas, como en el caso del fenómeno de {\itshape peak shift}.

La aproximación de Spence de gradientes de excitación e inhibición puede dar cuenta del fenómeno con base en la sumación de la excitación de $S^{+}$ y la inhibición de $S^{-}$. La aproximación de {\itshape generalization-over-dimensions} es la suma de reforzadores después de la generalización lo que explica el {\itshape peak shift}.

\subsection{Where and how the murky past is transported to the present}

No se hace alusión a los mecanismos neurobiológicos del sistema de generalización. Esto es parte de una tradición analítico-conductual de explicar las relaciones entorno-conducta sin dependencia de la forma en que funcionan los sistemas perceptuales y de memoria.

Aun así, entender la naturaleza biológica del transporte de la experiencia pasada al presente puede ser útil, y el modelo cuantitativo señalado aquí puede ser fructífero para ello.

\subsection{Conclusions}

Es adaptativo comportarse de acuerdo con el futuro, pero éste solo existe en tanto es predecible con base en el pasado. Los organismos aprenderán a navegar el presente usando cualquier estímulo que parezca correlacionar con eventos subsecuentes. Eventos que sucedan de forma ordenada controlarán la conducta. La generalización ensombrece el orden aparente haciendo poco claro al pasado e incierto al futuro, pero también permite a los organismos navegar en entornos novedosos. Entender los factores que hacen que un pasado se generalice al presente más que otro es un paso crítico para entender la conducta.


\end{document}
