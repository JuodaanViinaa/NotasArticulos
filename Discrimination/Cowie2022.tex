\documentclass[a4paper,12pt]{article}
\usepackage[utf8]{inputenc}
\usepackage[T1]{fontenc}
\usepackage[spanish]{babel}
\usepackage{csquotes}
\usepackage{anysize}
\usepackage{graphicx}
\usepackage{hyperref}
\marginsize{25mm}{25mm}{25mm}{25mm}

\title{Choosing a future from a murky past: A generalization-based model of behavior}
\author{Sarah Cowie \and Michael Davison}
\date{2022}

\begin{document}
{\scshape\bfseries \maketitle}

\section{Introduction}

La organización de los eventos en orden define la relación de estímulos, conductas y consecuencias. El orden permite recordar el tiempo relativo de los eventos como un componente de la estructura del entorno. La memoria de la estructura es lo que controla el comportamiento, es decir, el futuro existe solo hasta el punto en que el pasado puede recordarse para crearlo. Recordar permite guiarnos hacia metas.

Cuantos más grados de libertad en el entorno, más importante es recordar para guiar el comportamiento.

La percepción y el recordar son imprecisos por naturaleza, y por lo tanto ponen límites en el grado en que se puede desarrollar control. Incluso en entornos altamente estables la elección no sigue perfectamente a la estructura del ambiente, {\itshape e.g.,} sub-igualación. Estas imprecisiones son más evidentes cuando las respuestas son similares entre sí o cuando eventos relacionados y ordenados están separados en el tiempo.

Sin importar la estructura real, lo que controla el comportamiento es la estructura experimentada y recordada. El proceso por el que la percepción y la memoria causan que la estructura percibida sea distinta de la real se denomina {\itshape generalización}.

La misma generalización nubla la estructura percibida y permite responder a estímulos que no son idénticos a los del pasado, es decir, no será necesariamente guiada por un proceso de exploración adaptativo.

No se puede saber en qué momento termina la percepción y comienza la memoria, así que se pretende explorar solo cómo la generalización que resulta en recuerdo imperfecto llega a estar bajo el control del ambiente.

La aproximación actual, que la generalización que permite navegar en entornos con estructuras que no son idénticas es igual a la generalización en entornos cuya estructura sí es idéntica, tiene la ventaja de la parsimonia.

Se utilizará una aproximación cuantitativa dado que permite formalizar las relaciones entre eventos sin ambigüedad.

\section{The generalization-over-dimensions model}

Esta aproximación usa reforzadores pasados para predecir la conducta futura---una traducción directa en lugar de basarse en propiedades estimadas de los estímulos encontrados. Los reforzadores obtenidos son desplazados a través de dimensiones de los estímulos para modelar el proceso de generalización. La tasa de reforzadores después de la generalización determina la estructura detectada. Se asume que la conducta iguala estrictamente a la tasa de reforzadores {\itshape percibida}.

\vspace{2mm}
{\centering\fbox{\parbox{0.95\textwidth}{%
            Supuestos del modelo:

            \begin{itemize}
                \item Reforzadores obtenidos ante un estímulo se generalizan a otros en esa dimensión según su distancia y la cantidad de reforzadores obtenidos.
                \item La generalización ocurre en todas las dimensiones relevantes.
                \item Los reforzadores aparentes ante cualquier combinación estímulo-respuesta son la suma de todos los reforzadores generalizados actuando sobre ella.
                \item La elección es igual a la suma de todos los reforzadores aparentes para las dos alternativas de una condición de estímulo.
            \end{itemize}
}}}
\vspace{2mm}

Si hay una diferencia constante entre cualquier par de estímulos, se asume que la generalización mutua es constante también. Así, si $m$ es la probabilidad de detectar correctamente los dos estímulos:
\[
    \log\frac{B_1}{B_2} = \log\frac{R'_1}{R'_2} = \log \left(\frac{
            mR_1 + (1 - m) R_2
        }{
            mR_2 + (1 - m) R_1
    }\right) + \log c
,\]
de modo que una proporción constante de reforzadores $R$ obtenida para cada respuesta 1 y 2 es reasignada a la respuesta alternativa. $B$ se refiere a los conteos de respuestas a ambas alternativas, y $R'$ a los números aparentes de reforzadores tras la generalización. $\log c$ es el sesgo en las respuestas como en igualación generalizada. Se asume en este caso que la generalización entre $B_1$ y $B_2$ es simétrica.

En dimensiones continuas como el tiempo se usa un proceso de redistribución Gaussiano donde la desviación estándar varía dependiendo de la posición del reforzador a lo largo de una dimensión.


\end{document}
