\documentclass[a4paper,12pt]{article}
\usepackage[utf8]{inputenc}
\usepackage[T1]{fontenc}
\usepackage[spanish]{babel}
\spanishdecimal{.}
\usepackage{csquotes}
\usepackage{anysize}
\usepackage{graphicx}
\marginsize{25mm}{25mm}{25mm}{25mm}

\title{Studies of wheel-running reinforcement: parameters of Herrnstein's (1970) response-strength equation vary with schedule order}
\author{Terry W. Belke}
\date{2000}

\begin{document}
{\scshape\bfseries \maketitle}

En procedimientos de una sola fuente de reforzamiento programada la relación respuesta-reforzador toma la forma de una hipérbola rectangular según la ecuación
\[
B_{1} = \frac{
	kR_1
}{
	R_1 + R_O
}
,\]
donde $B_1$ es la tasa de respuesta predicha, $R_1$ es la tasa de reforzamiento obtenido, y $k$ y $R_O$ son parámetros ajustados. $k$ es la asíntota de tasa de respuestas, y se ha interpretado como el índice de un componente moto de una respuesta reforzada. $R_O$ es la tasa de reforzamiento que mantiene la mitad de la tasa asintótica de respuesta. Describe cuán rápido la tasa de respuesta se eleva hacia la asíntota según incrementa la tasa de reforzamiento. Se interpreta típicamente como el reforzamiento de fuentes extrañas, pero también como un índice de eficacia de reforzamiento.

Se ha mostrado que las tasas de respuestas incrementan y decrementan a lo largo de una sesión aun con tasas de reforzamiento constantes.


\end{document}
