\documentclass[a4paper,12pt]{article}
\usepackage[utf8]{inputenc}
\usepackage[T1]{fontenc}
\usepackage[spanish]{babel}
\usepackage{csquotes}
\usepackage{anysize}
\usepackage{graphicx}
\marginsize{25mm}{25mm}{25mm}{25mm}

\title{Does a negative discriminative stimulus function as a punishing consequence?}
\author{Vikki J. Bland \and Sarah Cowie \and Douglas Elliffe \and Christopher A. Podlesnik}
\date{2018}

\begin{document}
{\scshape\bfseries \maketitle}

En {\itshape applied behavior analysis} se suele utilizar reforzamiento diferencial de conductas alternativas ({\itshape differential reinforcement of alternative behavior}, DRA), pero no siempre es la mejor alternativa dado que hay conductas que necesitan disminuirse de inmediato. Para ellas se suele usar castigo, aunque es controversial y no siempre funciona.

¿Es peor exponer a los clientes a estímulos aversivos o dejarlos en su miseria?

Sería mejor utilizar estímulos no-dañinos que funcionen como castigos, como estímulos asociados con consecuencias desfavorables. Davidson estudió el efecto de un estímulo pareado con un choque sobre una conducta no relacionada. Sin embargo, sus resultados no fueron concluyentes, aunque sugieren que estímulos discriminativos correlacionados con estímulos aversivos pueden castigar conducta operante, igual que los castigos primarios.

Otras investigaciones sugieren solamente que los estímulos discriminativos negativos disminuyen las respuestas de observación, pero no dan información sobre una posible disminución en las respuestas mantenidas por reforzamiento primario.

Otras más (Auge, 1977, Thompson, 1965) sugieren que estímulos discriminativos que señalan consecuencias negativas ({\itshape i.e.,} la ausencia relativa de reforzamiento primario) pueden castigar las respuestas mantenidas por reforzadores primarios. Otros estímulos que señalen la completa omisión de reforzamiento deberían funcionar aun mejor como inhibidores (Honig, Boneau, Burstein & Pennypacker, 1963).

Este estudio pretende averiguar si la tasa de picoteo mantenida por la comida en presencia de un estímulo positivo sería reducida si el picoteo también produce un breve estímulo negativo. Se investiga también si la presencia del estímulo negativo alterará la elección sin necesidad de cambiar las tasas de reforzamiento (es decir, si castigará la elección).

Palomas fueron entrenadas en un programa múltiple de dos componentes. En uno, respuestas a una tecla verde eran reforzadas con acceso a comida según un programa VI15s. En el otro, las respuestas a una tecla roja nunca eran reforzadas. Tras obtener discriminación, se presentaron dos sesiones de prueba distribuidas en ciclos repetidos de línea base. En las pruebas las respuestas a S+ resultaban en la presentación de S- durante 1.5s de acuerdo con un programa de VI5. 




\end{document}
