\documentclass[a4paper,12pt]{article}
\usepackage[utf8]{inputenc}
\usepackage[T1]{fontenc}
\usepackage[spanish]{babel}
\spanishdecimal{.}
\usepackage{csquotes}
\usepackage{anysize}
\usepackage{graphicx}
\marginsize{25mm}{25mm}{25mm}{25mm}

\title{On the law of effect}
\author{R. J. Herrnstein}
\date{1970}

\begin{document}
{\scshape\bfseries \maketitle}

Los animales no se quedan con la primera acción exitosa, sino que la mejoran hasta llegar a un desempeño óptimo.
Se necesita más que una ley estática del efecto para una teoría persuasiva.
Decir que los animales se adaptan tampoco es suficiente: la adaptación es una pregunta, no una respuesta, y ambas explicaciones tienen evidencia en contra.
Por ejemplo, se ha encontrado que al pasar a un organismo de un programa de intervalo a uno de razón, su patrón de respuestas cambia.
Una explicación de ``stamping-in'' no es suficiente dado que un programa de razón refuerza todos los patrones con igual probabilidad.
Además, la dirección de este cambio es impredecible: para algunos animales incrementa la tasa de respuesta, mientras que para otros disminuye.
Ambos fenómenos violan la ley del efecto, pero solo el incremento es adaptativo.

Otro hallazgo que viola ambos principios está en el programa conjuntivo:
En este programa están en efecto dos programas simultáneos---intervalo y razón---de tal suerte que se refuerza la primera respuesta después de una cierta cantidad de respuestas y el paso de una cierta cantidad de tiempo.
Un programa de intervalo por sí mismo penaliza implícitamente tasas altas de respuesta dado que incrementa la cantidad de trabajo por el mismo reforzamiento.
Un programa de razón no penaliza del mismo modo, y de hecho puede favorecer tasas altas.
El programa conjuntivo encadena ambas propiedades, dado que la tasa de reforzamiento es proporcional a la de respuesta solo para tasas de respuesta que no sean más grandes que $\frac{n}{t}$ (donde $n$ es el criterio de respuestas; y $t$, el de tiempo).

El hallazgo crítico estuvo en un procedimiento donde el componente de intervalo estuvo fijo en 15 minutos, pero el de razón varió de 0 a 240 respuestas.
Aunque las palomas respondían alrededor de 300 veces por reforzador cuando el requisito de razón era 0, incrementarlo apenas a 10 o 40 ralentizó detectablemente las respuestas.
Requisitos mayores causaron decrementos progresivamente mayores en las respuestas, lo que redujo la tasa de reforzamiento.
Esto no se presta a un análisis de ley del efecto: ¿qué podría ``estamparse'' cuando las respuestas y el requisito de respuestas varían inversamente?
Además, el comportamiento tampoco parece adaptativo, pues el animal podría emitir más respuestas en el intervalo fijo de las que demanda el requisito de número, y aun así la conducta es deprimida.

Estos problemas son mejor explicados por la fuerza de respuesta: en el primer caso, si el primer efecto tras el cambio de intervalo a razón es un incremento en la tasa de reforzamiento, entonces incrementará la tasa de respuesta, lo que incrementará más aun la tasa de reforzamiento, etc. Si el primer efecto es un decremento en la tasa de reforzamiento, eso disminuirá la tasa de respuestas, lo que disminuirá más la tasa de reforzamiento, etc.
Esta relación dinámica solo ocurre con programas de razón dado que mantienen proporcionalidad entre la tasa de reforzamiento y la de respuesta.

La fuerza también explica el resultado del programa conjuntivo: cualquier incremento en el intervalo entre reforzamientos causado por el incremento en el requisito de razón disminuirá la fuerza.

Estos ejemplos muestran que ni el ``stamping-in'' ni la adaptación, ni ambos, explican lo que explica la fuerza de la respuesta.

\section{Reinforcement as strengthening}




\end{document}
