\documentclass[a4paper,12pt]{article}
\usepackage[utf8]{inputenc}
\usepackage[T1]{fontenc}
\usepackage[spanish]{babel}
\usepackage{csquotes}
\usepackage{anysize}
\usepackage{graphicx}
\marginsize{25mm}{25mm}{25mm}{25mm}

\title{Precis of Semantic Cognition: A Parallel Distributed Processing Approach}
\author{Timothy T. Rogers \and James L. McClelland}
\date{2008}

\begin{document}
{\scshape\bfseries \maketitle}



% Crítica Borsboom y Visser: Los modelos de redes realmente no son más que modelos estadísticos de los cuales difícilmente puede decirse que aprendan conceptos y relaciones en tanto que no son capaces de extraer la información relevante del mundo natural, sino que esta les es alimentada por los investigadores. Los autores tienen como opciones (1) decir que los modelos hacen más que tratar los datos del mismo modo que lo haría cualquier paquete estadístico con técnicas multivariadas, o (2) decir que la cognición semántica no es más que estadística multivariada, en cuyo caso incluso se podría decir que SPSS puede formar conceptos y tener una forma de cognición semántica.

% Quizá podría decirse que el mundo natural no es tan claramente delimitado como los contextos artificiales usados en la demostración de R y M. Difícilmente se encontrarán dominios que no tengan ningún tipo de traslape y que además contengan individuos que compartan exactamente el mismo tipo de relaciones pero que a la vez sean perfectamente diferenciables entre sí. Siendo así, ¿el modelo podría de la misma manera distinguir que los dominios son análogos a pesar de no compartir ninguna característica?  Sin embargo, ¿esto es realmente una crítica? ¿Contradice el argumento de que los modelos de redes pueden encontrar características interesantes en los estímulos ambientales?

% Parece que de estas similitudes inferidas el modelo aprende a tratar a las relaciones como "tener" y "poseer" como casi idénticas a pesar de tener poco traslape. Estas correspondencias no son capturadas por otros análisis multivariados.

% -----------------------

% Crítica Feeney, Crisp y Wilburn: Los autores parecen plantear que el razonamiento inductivo y la cognición semántica son nombres diferentes para la misma cosa. Los críticos difieren. No todo se puede explicar por conocimiento y la forma de adquirirlo. Como ejemplo se ha encontrado que el IQ parece estar relaciona con la sensibilidad de los individuos a ciertas características de los argumentos inductivos. Esto puede implicar que el razonamiento esta basado en más que procesos que permiten calcular la similitud entre representaciones. La sensibilidad a la diversidad en los argumentos, que implica saber que muestras mayores hacen inferencias más sólidas, es otra muestra de que may más que la mera similitud en el pensamiento.

% Los autores responden que no abordan la inferencia deductiva explícitamente, pero que creen que el procesamiento de silogismos lógicos no es indicativo de la naturaleza del pensamiento humano, sino que es una habilidad adquirida dependiente de otros aspectos de la cognición. Para ellos, además, los procesos de control de otros procesos ocurren mediante la manipulación interna de representaciones y la re-utilización de outputs como re-inputs, en lugar de mediante un sistema de procesamiento separado.

% -----------------------

% Crítica de Hampton: Sugieren (1) utilizar la capa de contexto o relación. Esta capa determina el tipo de relación entre un sustantivo y una propiedad (pino ES UN árbol). Quizá esa capa estaría mejor usada si solo codifica el contexto. Agrupar propiedades por su sintaxis puede no tener sentido en el mundo real (cuando ES aplica para "valioso", "molesto" o "malo para la salud", por ejemplo). Quizá la capa de contexto podría codificar el tipo de propiedad usando criterios semánticos y no sintácticos (e.g., parte, apariencia, función, origen, etc.) (2) La categoría (relación IS A) es una propiedad especial que da más inferencias útiles sobre los objetos. Esta diferencia puede implementarse estructuralmente. (3) El modelo no puede manejar distintos tipos de verdades (tautológica, necesaria, genérica). Esto puede ser positivo dado que las personas también son malas en eso, pero sería bueno saber si el modelo puede hacer la distinción.

% Los autores reconocen que en esa área es en donde el modelo necesita más enriquecimiento. El rango de variación en propiedades de objetos entre contextos es muy rico.

% -----------------------

% Crítica Kemp y Tenenbaum: Los autores parecen creer que los modelos estructurados y el procesamiento paralelo operan en el mismo nivel de análisis y por lo tanto son competidores, pero ese no es el caso: muchos fenómenos pueden ser compatibles con ambas aproximaciones si se asume que operan en niveles distintos. Además los obstáculos planteados por los autores para los modelos estructurados no son irresolubles. Incluso esos modelos se antojan más prometedores dado que parecen poder lidiar naturalmente con problemas que para PDP son complicados (relaciones anidadas, conocimiento abstracto sobre relaciones, relaciones distintas con significado casi idéntico). Además los modelos estructurados hacen contacto directo con investigación psicológica previa, aunque PDP puede motivar investigación futura.

% Esto puede surgir del malentendido de que los modelos conexionistas solo pueden representar a items como similares cuando estos comparten características observables. Sin embargo ya se ha mostrado que esto no es necesariamente cierto: el modelo puede identificar a items como similares si comparten relaciones similares con otros items en su contexto, aun si los dos en cuestión no comparten ninguna característica. Los modelos conexionistas bien son capaces de ese tipo de relaciones abstractas.

% -----------------------

% Crítica Kropff y Treves: La red presentada por los autores es un paso adelante desde constructos basados en mera lógica (como la teoría-teoría). Aunque no es claro si puede agotar todo el insight potencial que las aproximaciones de redes neuronales pueden proveer sobre la arquitectura de la cognición semántica, específicamente dar cuenta de la naturaleza aparentemente discreta de la cognición semántica.

% El paradigma de backpropagation no refleja hallazgos recientes, como la tendencia de los patrones de actividad neural a caer en ocasiones en estados atractores discretos. Los críticos proponen ir más allá de las redes entrenadas con backpropagation y considerar redes corticales con mecanismos Hebbianos.

% Los autores dicen que, aunque las redes con atractores son cool, lo abrupto de sus transiciones a menudo es sobre-enfatizado. Aunque hay razones para preferir modelos con atractores, quizá las discontinuidades en el desarrollo no son una de ellas.

% -----------------------

% Crítica MacDonald: La contribución de los autores es la cantidad de fenómenos que consiguen simular con una única arquitectura robusta. El modelo ciertamente abarca muchos casos, como el aprendizaje dirigido por errores y la covariación coherente. Pero algo que parece escapársele es la formación y persistencia de ilusiones (creencias en ausencia de evidencia, evidentes para el creyente y resistentes a la evidencia invalidante).

% Las ilusiones parecen comenzar como una sensación vaga que se cristaliza con el tiempo y se vuelve más específica. Pero no son aleatorias: las ilusiones de personas deprimidas son deprimentes, las de maníacos son expansivas. No parecen emerger de asociaciones aleatorias.

% En la ilusión de Capgras el paciente piensa que un ser querido fue reemplazado por un impostor. Esto parece derivarse de vías subcorticales lesionadas que impiden la respuesta emocional usual ante los rostros, lo que lleva a pensar que hay algo malo con ellos. Es decir, una sensación anormal lleva a una explicación cognitiva a posteriori.

% No es claro cómo la susceptibilidad a ilusiones encaja en el modelo de R y M. ¿Cómo se puede 'romper' al modelo para volverlo iluso?

% Los autores responden que la teoría que explica el síndrome de Capgras tiene una interpretación directa en el modelo: la disrupción en las conexiones del pool semántico integrativo y un pool límbico separado podría producir el reconocimiento parcial de una persona sin la respuesta emocional asociada (no me satisface la explicación).

% -----------------------

% Crítica Majid y Huettig: Los autores no consideran en absoluto los datos de cross-linguistic naming, que parecen plantear problemas serios para el modelo.

% -----------------------


\end{document}
