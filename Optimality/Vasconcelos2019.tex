\documentclass[a4paper,12pt]{article}
\usepackage[utf8]{inputenc}
\usepackage[T1]{fontenc}
\usepackage[spanish]{babel}
\usepackage{csquotes}
\usepackage{anysize}
\usepackage{graphicx}
\marginsize{25mm}{25mm}{25mm}{25mm}

\title{Evolved psychological mechanisms as constraints on optimization}
\author{Marco Vasconcelos \and Valeria V. González \and Alejandro Macías}
\date{2019}

\begin{document}
{\scshape\bfseries \maketitle}

La conducta también es susceptible de ser seleccionada naturalmente, por lo que, con el tiempo, parece ser diseñada para maximizar la adecuación inclusiva.

Investigadores ingenuos toman estas premisas como indicadoras de que toda la conducta debería ser perfectamente adaptable a cualquier demanda ambientas. Si la conducta fue esculpida por la acción optimizante de la selección natural, la conducta misma debería ser óptima. Esto implicaría que el agente optimizador es el organismo y no la selección natural. La conducta actual es adaptativa solo en un ambiente similar al ambiente que le dio forma. Todo se trata de una paridad entre el dominio de selección y el dominio de evaluación.

No se espera conducta óptima en todas las situaciones. Se espera una ejecución razonablemente buena de sus mecanismos, porque lo contrario serían eliminados por la selección, pero aun así puede haber desviaciones de la optimalidad.

Las desviaciones de la optimalidad son interesantes no por cuestionar las aproximaciones normativas, sino por exponer situaciones en las cuales los mecanismos resultan mal. Permiten entender su significancia adaptativa.

Zentall ha mostrado varios ejemplos de conducta subóptima: {\itshape less is better effect, ephemeral choice task, midsession reversal task}. Su conclusión general es que esa conducta puede explicarse por heurísticos evolucionados que funcionan bien en la naturaleza pero que a veces fallan en condiciones de laboratorio.

Un heurístico es un atajo para encontrar soluciones rápidas y eficientes, si no necesariamente óptimas. Encontrar esos heurísticos implica un {\itshape behavioral gambit} en el que los mecanismos psicológicos existentes no restringen la emergencia de reglas hechas a la medida para cada situación. Se argumenta que esta noción está mal guiada, en especial para conductas flexibles que pueden cambiar en la vida de un individuo.

La idea detrás de los heurísticos (para psicólogos cognitivos) o reglas de dedo (para ecólogos conductuales) es que la selección natural da a los organismos soluciones específicas para problemas que enfrentan repetidamente a través de generaciones. Esto implica la existencia de una colección amplia de reglas y de un mecanismo rector para seleccionar la regla adecuada. Sin embargo, es difícil saber en qué situaciones sería ventajoso un heurístico y qué determina su uso. Además, desdeña el uso de teorías amplias de aprendizaje en favor de algoritmos novedosos pero estrechos.

Se argumenta que dada la variabilidad de los ambientes, la selección natural debió dotar a los organismos con mecanismos generales de amplio dominio. Se favorece buscar mecanismos que se desempeñan bien en promedio en muchas tareas sobre una colección de reglas complejas para un conjunto reducido de problemas.

La búsqueda de mecanismos debería ir de la mano con preguntas funcionales para entender cómo problemas funcionales limitan los mecanismos que pueden evolucionar y cómo los mecanismos existentes restringen las posibles soluciones a un problema.

Al encontrar preferencias paradójicas uno debería preguntarse cómo pueden originarse de mecanismos conocidos operando en circunstancias distintas de las que les dieron forma. Solo al ser necesario se podría suplementar a los mecanismos conocidos con propuestas novedosas.

Los mecanismos conocidos son posibles adaptaciones, pero también imponen posibles restricciones en la optimización.Reconociendo esta naturaleza dual, las preferencias subóptimas pueden servir como herramientas para entenderlos y no para desmentirlos.


\end{document}
