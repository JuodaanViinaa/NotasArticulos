\documentclass[a4paper,12pt]{article}
\usepackage[utf8]{inputenc}
\usepackage[T1]{fontenc}
\usepackage[spanish]{babel}
\usepackage{csquotes}
\usepackage{anysize}
\usepackage{graphicx}
\usepackage{hyperref}
%\usepackage{amsfonts}
%\usepackage{tikz}
%\usepackage{amsmath}
\marginsize{25mm}{25mm}{25mm}{25mm}

\title{Overmatching under food uncertainty in foraging pigeons}
\author{Patrick Anselme \and Neslihan Wittek \and Fatma Oeksuez \and Onur Güntürkün}
\date{2022}

\begin{document}
{\scshape\bfseries \maketitle}

En entornos distribuidos en parches, los organismos abandonan un parche cuando su recompensa cae de cierto umbral en un intento por maximizar. Sin embargo, los organismos también exploran opciones poco o nada reforzadas o asociadas con falta de información, por ejemplo, intentan obtener comida no garantizada a pesar de tener grandes cantidades de comida a su disposición.
Aunque esto no parece favorecer la maximización, puede ser útil para proveer a los animales información.

Cuando la información no es suficiente, los organismos no deberían igualar la tasa de ganancia energética predicha por la teoría de forrajeo óptimo.

En experimentos pasados se han usado tablas de forrajeo con agujeros cubiertos con cinta negra para esconder comida. Palomas podían extraer 60 unidades de comida de 60 agujeros, o 60 unidades de 180 agujeros (condición de incertidumbre). En incertidumbre el peso de las palomas tiende a ser mayor a pesar de comer lo mismo. Además se producen menos picoteos por agujero, una distribución de picoteos más homogénea, y una mayor tendencia a evitar agujeros adyacentes.

La ley de igualación dicta que los animales igualarán las tasas de recompensa con sus tasas de respuesta, y es una herramienta útil para estimar cómo el comportamiento se desvía de lo que ``debería'' ser.

Se pretende explorar la igualación usando una tabla de forrajeo con 180 agujeros dividida en tres zonas: una con 100\% de agujeros con alimento (R), otra con 33.3\% de agujero s con alimento (U), y otra con 0\% de agujeros con alimento (NR). Se espera que el tiempo pasado en cada zona siga las proporciones de $3/4$, $1/4$, y $0/4$. La ley de igualación predice también que el tiempo por visita y los picotazos por visita serán equivalentes, es decir, R$=$U y NR$=$0. En contraste, si hay otros factores involucrados además de la tasa de ganancia energética la habrá más inversión temporal en la localización sin reforzamiento y en la parcialmente reforzada.

\section{Experiment 1}

Se utilizaron 10 palomas con experiencia en la tabla, pero no en elección.

La tabla era una placa de madera de 120 $\times$ 70 cm, con 9 filas de 20 agujeros a distancias regulares.
La tabla estaba cubierta con cinta negra, y cada agujero tenía una cruz cortada para facilitar el acceso sin revelar el contenido.

El experimento consistió en pretest, test, y post-test. En pre y post las tres áreas fueron delimitadas con tiras de cinta. Un área tenía comida consistente, otra inconsistente, y otra no tenía comida. Sus localizaciones variaron cada día. Un letrero señalaba el área de la comida consistente.

Tras 10 días de pretest las palomas pasaron al test, en el cual no se dividieron las áreas con cinta y solo se colocaron 60 unidades de comida aleatoriamente en los 180 agujeros.

Tras 10 días más se pasó al post-test. La meta del diseño era determinar si la incertidumbre de comida (test) alteraba la conducta de forrajeo de un modo que los efectos fuesen reversibles entre pre y post-test o los efectos fuesen atenuados/amplificados entre pre y post-test.

Los datos se organizaron de una manera que permitiera comparar los resultados del área con recompensa segura R con su área no recompensada más cercana NR-Prox y más lejana NR-Dist. Se promediaron las últimas tres sesiones para obtener conducta en estado estable.

\subsection{Results}

Los resultados en general sugieren que las palomas buscaban comida en el área señalada de la tabla, pero también dedicaban algo de tiempo a forrajear en áreas no recompensadas. Eran insensibles a la distancia entre el área señalada y las áreas no señaladas con respecto al tiempo pasado y el número de picotazos, aunque visitaban el área más distal más a menudo.

\section{Experiment 2}

Se entrenaron 9 palomas ingenuas en una tabla de 60 agujeros para aprender a buscar comida. Después se utilizó la tabla de 180 agujeros dividida en tres áreas de 90, 60 y 30 agujeros. En el área de 30 todos los agujeros tenían comida; en la de 60 ninguno; y en la de 90 había 30 agujeros al azar con comida. Cada área estaba asociada con un color: uno para comida confiable, otro para ausencia confiable, y una mezcla para incertidumbre.

Los resultados generales indican que las palomas invertían tiempo y picotazos en proporción a la tasa de reforzamiento en el área, lo que parece consistente con la ley de igualación. Sin embargo, debe notarse que la tasa de reforzamiento en cada área iba cambiando a lo largo de la sesión según las palomas iban agotando los agujeros con comida. Un análisis mostró sobre-igualación por visita en el área con comida inconsistente.

\section{General discussion}

Como indican la teoría de forrajeo óptimo y la ley de igualación, la presencia de comida en el área R atrajo a las palomas, pero sorprende que las palomas entrenadas inspeccionaban y picaban en el área no reforzada.

Las palomas parecían dispuestas a pasar tiempo explorando opciones de poca calidad. ¿Por qué esa propensión era mayor en el post-test? Quizá se deba a el tratamiento intermedio: se ha mostrado que exposición repetida a incertidumbre sensibiliza el responder ante una señal. Aunque esta interpretación no indica por qué eso ocurriría solamente en el área distal y no la proximal.


\end{document}
