\documentclass[a4paper,12pt]{article}
\usepackage[utf8]{inputenc}
\usepackage[T1]{fontenc}
\usepackage[spanish]{babel}
\spanishdecimal{.}
\usepackage{csquotes}
\usepackage{anysize}
\usepackage{graphicx}
\marginsize{25mm}{25mm}{25mm}{25mm}

\title{Exploratory search: Information matters mora than primary reward}
\author{Patrick Anselme}
\date{2023}

\begin{document}
{\scshape\bfseries \maketitle}

La teoría de forrajeo óptimo indica que los animales intentarán maximizar la recompensa por tiempo a largo plazo, pero describe agentes ideales con conocimiento completo del entorno, es decir, se centra en el problema de la explotación y no en la exploración.
Sin embargo, la exploración es fundamental para la maximización.

La exploración puede verse como adaptación a las variables que interfieren con la maximización.

En algunas especies, como las hormigas y abejas, individuos separados hacen tareas de exploración y explotación.
Esta disociación indica que son procesos separados que dependen de procesos bioconductuales diferentes.

Para algunos, la exploración es producto de la curiosidad, {\itshape i.e.,} aversión a la incertidumbre y su evitación mediante la colecta de información.
Para otros, la exploración regula la activación hacia abajo cuando es alta, y hacia arriba cuando es baja.
Aquí, se argumenta que la curiosidad no es requisito de la exploración, pues se ha encontrado en organismos sin cerebro.
Más bien, la recompensa no garantizada es vista como un desafío por superar.
La incertidumbre, aunque aversiva, no suele generar evitación.
Los organismos evolucionaron para enfrentar desafíos en el ambiente, y esta reacción a la incertidumbre es benéfica para la salud y supervivencia.

La explotación es un fenómeno local relacionado con el consumo; la exploración es global, y pretende determinar dónde, cuándo y cómo encontrar recompensas o sus claves asociadas.
La exploración es una inversión para el futuro.
Sin embargo, si las claves, recompensas, o incluso el hambre, no motivan la exploración, ¿qué lo hace?

Tres principios se proponen: {\itshape consistency tracking}, los organismos expuestos a recompensa no garantizada buscan información sobre la consistencia de pareos clave-recompensa más que las claves y recompensas mismas; {\itshape incentive effort}, al estar expuestos a recompensa no garantizada, los organismos alargan la exploración para compensar la falta de control cognitivo que tienen en la situación; y {\itshape behavioral variability}, los organismos expuestos a recompensas no garantizadas amplían el rango espacial o temporal de su comportamiento dado que esto incrementa las posibilidades de nuevos encuentros exitosos.

\section{Resource exploitation and the local influence of cues}

La expectativa o detección de un estímulo apetitivo lleva a una aproximación no-aleatoria a él.
La saliencia incentiva (``wanting'') es el proceso que transforma la representación cerebral de una mera memoria o percepción a un incentivo motivacionalmente potente.
El proceso depende de la liberación de dopamina en el estriado ventral y dorsal.
Para algunos organismos en procedimientos pavlovianos el pareamiento entre comida y señal hace a la señal atractiva en sí misma ({\itshape sign-trackers}).
Esto parece concordar con las nociones dichas: la procuración de recompensa es el factor a optimizar, pues una clave que predice mayor calidad o cantidad de comida debería ser aproximada y preferida sobre otras.
Lo mismo sucede en procedimientos instrumentales: reforzadores de mayor calidad generan preferencia.
Pero el argumento del artículo es que la influencia de asociaciones clave-consecuencia y acción-consecuencia es limitada, dado que son importantes al explotar locaciones bien conocidas, y por tanto no pueden explicar la conducta a una escala mayor.

Hay una distinción clásica entre búsqueda focal y general, sin embargo, si ambas buscan claves y recompensas, ¿qué distingue una de la otra? ¿Qué se busca en cada una? ¿Y cómo deciden los animales cambiar entre ellas? La saliencia incentiva y la ley de igualación pueden explicar la búsqueda focal, donde se aproxima un estímulo detectado o esperado, pero no es así con la búsqueda general, que no pretende aproximar un estímulo, sino contrarrestar los efectos de la incertidumbre mediante la exploración.
La explicación de la exploración requiere una teoría enfocada en la información y no en la explotación inmediata de claves y recompensas.

\section{Consistency tracking}

Breves ocurrencias de explotación durante la exploración (consumir un item encontrado al explorar) no deben llevar a concluir que la exploración no es más que explotación a mayor escala.
En la exploración los organismos intentan determinar {\itshape cuándo, dónde y cómo} encontrar asociaciones que predigan con la mayor consistencia posible la presencia de recompensas.
Sin {\itshape consistency tracking} la supervivencia estaría sujeta solo al control de encuentros positivos y negativos.

{\itshape Consistency tracking} es adaptativo porque maximiza las buenas decisiones y minimiza las malas en un ambiente.
Este principio explicaría por qué ocurre la búsqueda exploratoria en entornos nuevos en ausencia de expectativa de recompensa (donde se pueden encontrar posibles consistencias de clave-recompensa), pero no se muestra en entornos familiares donde la recompensa es esperada (y ya se conocen las consistencias).

Los procedimientos de respuesta de observación han mostrado que los animales valoran la consistencia de las señales y las consecuencias, pero quizá la evidencia más robusta venga del procedimiento de elección subóptima.
Se han propuesto diversas explicaciones, como el contraste, la reducción de la demora, y otras, pero ninguna da una explicación completa de los hallazgos.

Un hallazgo indica que si las demoras de entrega son menores en la alternativa subóptima que en la óptima, la elección subóptima es más probable.
Es posible que los organismos hayan evolucionado para percibir una clave predictora después de una demora como más consistente, y que estén mostrando una preferencia por la consistencia de los pareos clave-comida (información) que por la comida misma (recompensa).

Un experimento con humanos comparó tres modelos de búsqueda de información no instrumental.
El modelo con el mejor ajuste con los datos fue aquél que suponía a la incertidumbre como un estado aversivo, y los investigadores concluyeron que al buscar información no instrumental, las personas buscan la resolución de la incertidumbre más que disfrutar de la anticipación de consecuencias futuras.

\section{Conditioning, higher-order conditioning, and occasion-setting}

Un {\itshape occasion setter} es un estímulo que indica si está activa una contingencia específica (un $E^{D}$).
Este fenómeno puede contribuir a explicar la elección subóptima.
El estímulo del eslabón inicial puede funcionar como un {\itshape occasion setter} que indica el contexto (la opción) que contiene información; en cierto sentido, se crea un contexto que resuelve la ambigüedad.
La otra alternativa, en cambio, no indica la presencia de información.

\section{Hoarding behavior as an example of consistency tracking}

Guardar comida fuera del propio cuerpo para evitar que otros tengan acceso a ella es acumulación.
La acumulación puede verse como un intento por disminuir la variabilidad en la disponibilidad de comida para minimizar el riesgo posterior de inanición.

\section{Incentive effort}

Esfuerzo incentivo significa que la incertidumbre hace al esfuerzo auto-motivador.

Se argumenta que la incertidumbre motiva la búsqueda porque es percibida como un desafío por resolver.
Idea extraña, dado que implica que para los organismos es reforzante ``resolver'' un desafío.

En el procedimiento de PCA los animales responden más ante un estímulo que inconsistentemente predice comida que a uno consistente.
Si el esfuerzo ante un desafío no es reforzante por sí mismo, ¿por qué los animales responden más bajo la incertidumbre? Sinceramente no me convence.

Una explicación simple sería que la atribución de saliencia incentiva es mayor para un estímulo inconsistente que para uno consistente.
Sin embargo, se ha evaluado la saliencia incentiva en condiciones de certidumbre e incertidumbre y no se han encontrado diferencias, por lo que quizá no se puede atribuir a ello la diferencia en esfuerzo.

Cuando el entorno es incierto y el aprendizaje no ayuda, la única opción para sobreponerse a la aversión de la incertidumbre es trabajar más.
Esto se mostró en un experimento con palomas en el que se colocó comida en una tabla con agujeros en dos condiciones: una de incertidumbre con 30 de 90 agujeros con comida; y otra de certidumbre con 30 de 30 agujeros con comida.
Las palomas pasaban más tiempo y se esforzaban más por visita en la condición de incertidumbre.
Me parece aventurado concluir que la inconsistencia hace que las palomas estén dispuestas a trabajar más, pero eso es lo que se concluye.

Aunque se piensa que la igualación es óptima, quizá en un entorno de incertidumbre la estrategia óptima para mantenerse con vida es la sobre-igualación (responder más de lo esperado con base en la tasa de reforzamiento por agujero).

\section{Contrafreeloading as an example of incentive effort}

Bajo ciertas circunstancias los animales prefieren comida ganada sobre comida gratuita.
Por ejemplo, jerbos a los que se da un tazón con 1000 semillas libres y otro con 200 semillas escondidas en arena consumen más del segundo tazón, y esto ocurre aunque no estén hambrientos, por lo que la aproximación a la comida no parece ser la función principal.

Se propone que la presencia de incertidumbre es un factor que motiva el contrafreeloading: cuando se resuelve la incertidumbre la alternativa libre es preferida.
Aunque esto tiene beneficios evolutivos a largo plazo (el {\itshape qué}), no indica los mecanismos proximales (el {\itshape cómo}).

La explicación del esfuerzo incentivo indica que los organismos invierten más tiempo y esfuerzo para compensar su falta de control cognitivo.

\section{Behavioral variability}



\end{document}
