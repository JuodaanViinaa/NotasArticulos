\documentclass[a4paper,12pt]{article}
\usepackage[utf8]{inputenc}
\usepackage[T1]{fontenc}
\usepackage[spanish]{babel}
\spanishdecimal{.}
\usepackage{csquotes}
\usepackage{anysize}
\usepackage{graphicx}
\marginsize{25mm}{25mm}{25mm}{25mm}

\title{Observing responses and serial stimuli: searching for the reinforcing properties of the $S^{-}$}
\author{Rogelio Escobar \and Carlos A. Bruner}
\date{2009}

\begin{document}
{\scshape\bfseries \maketitle}

Wyckoff describió un procedimiento para estudiar la conducta de observación en el que componentes de FI 30s y extinción se alternaban aleatoriamente.
Una respuesta en un pedal convertía el programa mixto en uno múltiple, indicando con un color el programa activo.

Dado que las respuestas resultan en la presentación de estímulos inicialmente neutros, el procedimiento es importante para estudiar el reforzamiento condicionado.
Es útil porque separa la respuesta que produce comida de la que produce reforzamiento condicionado.
Pero el hecho de que la respuesta produzca tanto $S^{+}$ como $S^{-}$ produce problemas de interpretación.

El consenso indica que las condiciones que establecen a un estímulo condicionado en condicionamiento Pavloviano son las mismas que establecen a un reforzador condicionado.

Aunque varios estudios apoyan la noción de que $S^{+}$ funciona como reforzador condicionado y $S^{-}$ como estímulo aversivo, algunos estudios reportan hallazgos incongruentes con respecto a la función de $S^{-}$.

Perone y Baron (1980) expusieron a humanos a un programa mixto VI-EXT con dos teclas como respuestas de observación: una de ellas producía tanto $S^{+}$ como $S^{-}$ según el componente efectivo; la otra, únicamente $S^{+}$ (no tenía efecto durante EXT).
Se encontró que los participantes preferían la tecla que producía ambos estímulos, y un hallazgo posterior mostró que $S^{-}$ por sí mismo sostenía la conducta de observación.

Se ha buscado relacionar esos hallazgos con una explicación asociativa encontrando artefactos en su procedimiento.
Se mostró que $S^{-}$ solo refuerza la observación cuando la respuesta que produce reforzamiento primario implica esfuerzo considerable.
Sin embargo, en el experimento original de Perone y Baron no se dieron instrucciones explícitas sobre la respuesta de observación, y en la réplica sí.



\end{document}
