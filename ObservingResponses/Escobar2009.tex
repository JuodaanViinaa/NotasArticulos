\documentclass[a4paper,12pt]{article}
\usepackage[utf8]{inputenc}
\usepackage[T1]{fontenc}
\usepackage[spanish]{babel}
\spanishdecimal{.}
\usepackage{csquotes}
\usepackage{anysize}
\usepackage{graphicx}
\marginsize{25mm}{25mm}{25mm}{25mm}

\title{Observing responses and serial stimuli: searching for the reinforcing properties of the $S^{-}$}
\author{Rogelio Escobar \and Carlos A. Bruner}
\date{2009}

\begin{document}
{\scshape\bfseries \maketitle}

Wyckoff describió un procedimiento para estudiar la conducta de observación en el que componentes de FI 30s y extinción se alternaban aleatoriamente.
Una respuesta en un pedal convertía el programa mixto en uno múltiple, indicando con un color el programa activo.

Dado que las respuestas resultan en la presentación de estímulos inicialmente neutros, el procedimiento es importante para estudiar el reforzamiento condicionado.
Es útil porque separa la respuesta que produce comida de la que produce reforzamiento condicionado.
Pero el hecho de que la respuesta produzca tanto $S^{+}$ como $S^{-}$ produce problemas de interpretación.

El consenso indica que las condiciones que establecen a un estímulo condicionado en condicionamiento Pavloviano son las mismas que establecen a un reforzador condicionado.

Aunque varios estudios apoyan la noción de que $S^{+}$ funciona como reforzador condicionado y $S^{-}$ como estímulo aversivo, algunos estudios reportan hallazgos incongruentes con respecto a la función de $S^{-}$.

Perone y Baron (1980) expusieron a humanos a un programa mixto VI-EXT con dos teclas como respuestas de observación: una de ellas producía tanto $S^{+}$ como $S^{-}$ según el componente efectivo; la otra, únicamente $S^{+}$ (no tenía efecto durante EXT).
Se encontró que los participantes preferían la tecla que producía ambos estímulos, y un hallazgo posterior mostró que $S^{-}$ por sí mismo sostenía la conducta de observación.

Se ha buscado relacionar esos hallazgos con una explicación asociativa encontrando artefactos en su procedimiento.
Se mostró que $S^{-}$ solo refuerza la observación cuando la respuesta que produce reforzamiento primario implica esfuerzo considerable.
Sin embargo, en el experimento original de Perone y Baron no se dieron instrucciones explícitas sobre la respuesta de observación, y en la réplica sí.

Un experimento posterior mostró que en ausencia de instrucciones específicas sobre la función del $S^{-}$ éste era preferido sobre otro estímulo no correlacionado con la entrega de reforzamiento; pero en presencia de instrucciones específicas, la preferencia era contraria.
Esto desafía explicaciones pavlovianas del reforzamiento condicionado, pero es consistente con la explicación de la información: tanto $S^{+}$ como $S^{-}$ tienen poder reforzante por su reducción en la incertidumbre.
Pero la explicación no es convincente a la luz de la evidencia que muestra su insuficiencia para explicar la conducta de observación.
Así, estos resultados más bien indican que la función del $S^{-}$ no es bien entendida aún, y es posible que se haya establecido una relación accidental entre $S^{-}$ y la entrega de reforzamiento.

En un experimento Allen y Lattal presentaron un programa VI-EXT a palomas, y encontraron que $S^{-}$ puede mantener las respuestas solo si la respuesta productora de comida durante el componente de EXT disminuye la frecuencia de reforzamiento durante el otro componente.
Concluyeron que $S^{-}$ adquiere propiedades reforzantes dad la contingencia remota impuesta durante el componente de reforzamiento.

Un experimento de Escobar y Bruner presentó a ratas un programa mixto RI-EXT en el que responder en una palanca producía comida, y responder en la otra producía estímulos correlacionados con el programa en efecto.
Un grupo experimentó un período no señalado entre EXT y RI en el cual las respuestas de observación no tenían efecto; el otro grupo experimentó el mismo período pero colocado entre RI y el siguiente EXT.
Para el primer grupo disminuyó las respuestas de observación durante EXT.
Para el segundo grupo no hubo cambios sistemáticos.
Esto sugiere que los efectos de $S^{-}$ en la observación están ligados a su relación temporal con el componente de reforzamiento.

Un problema en el procedimiento es que $S^{+}$ y $S^{-}$ se definen solo por su correlación con el componente en efecto y no se toma en cuenta su relación temporal con la entrega de reforzamiento.
Se ha encontrado que los estímulos pueden ser aversivos, reforzantes o neutrales dependiendo del intervalo estímulo-reforzamiento.
Pero en procedimientos de observación el intervalo varía de forma no sistemática, y esa variación puede ser responsable de los efectos aparentemente contradictorios del $S^{-}$.
Por ejemplo, una respuesta de observación que ocurre al final de EXT estaría temporalmente cerca del reforzamiento y puede volverse reforzante.

Aquí se reportan experimentos en los que se varió la relación temporal entre $S^{-}$ y el inicio del componente de reforzamiento.

\section{Experiment 1}

Al procedimiento de observación se agregaron estímulos distintos presentados secuencialmente durante sub-intervalos sucesivos del componente EXT.
Es decir, las respuestas de observación producían un reloj, lo que permitía controlar el intervalo mínimo y máximo entre cada $S^{-}$ y el componente de reforzamiento.
El propósito fue determinar el efecto de la relación temporal entre $S^{-}$ y el componente de reforzamiento sobre la frecuencia de las respuestas de observación.

Ya otros experimentos han utilizado relojes.
Uno de ellos utilizó tres estímulos que se alternaban durante los tercios del intervalo entre reforzamiento, y se encontró que las respuestas de observación incrementaban durante el segundo subintervalo, pero disminuían en el tercero.
Según Kendall, las respuestas de observación podrían sostenerse exclusivamente con el estímulo más cercano al reforzamiento, así que replicó el procedimiento con tres condiciones: una idéntica, una en la que se eliminaron los estímulos de los dos primeros subintervalos, y una en la que se eliminó el estímulo del último subintervalo.
En las primeras dos replicó el hallazgo de más respuestas en el segundo subintervalo y menos en el tercero; en la tercera condición las respuestas de observación eran cercanas a cero.
Kendall concluyó que solo el estímulo contiguo al reforzamiento era un reforzador condicionado.

Este estudio replica el procedimiento de reloj opcional, pero con componentes EXT de duración variable.
Para determinar si las respuestas de observación durante EXT eran mantenidas por los distintos estímulos presentados durante EXT o solo por $S^{+}$, un grupo control fue expuesto a un procedimiento en el que un solo $S^{-}$ estaba presente durante todo el componente EXT.

\subsection{Método}

Se utilizaron seis ratas Wistar.
El experimento se corrió en cajas con un solo panel de dos palancas con luz y un comedero.
Las ratas fueron entrenadas en un programa FR1 que pasó progresivamente a RI 20 s.

Un grupo experimentó un programa múltiple que alternaba componentes de reforzamiento y EXT.
El componente de reforzamiento duraba 20 s y entregaba comida en un solo momento aleatorio después de una respuesta en la palanca izquierda.
Se dividía el tiempo en cuatro segmentos de 5 seguidnos y se seleccionaba al azar uno de ellos.
La primera respuesta emitida en ese segmento era reforzada.
Este componente era señalado con una luz parpadeante.

La duración de EXT era seleccionada al azar entre 20 y 100 s.
EXT era dividido en segmentos de 20 s, y se señalaba el segmento con un tono de intermitencia incremental, de modo que al estar en el segmento más lejano (100 a 80 s) el tono era constante, mientras en el segmento más cercano (20 a 0 s) la intermitencia era de 0.1 s.

El grupo control solo difería en que un solo tono constante era presentado durante los segmentos de EXT.

Estas contingencias estuvieron en efecto 30 días y tenían por propósito entrenar la discriminación.

Para el procedimiento de observación los programas pasaron a ser mixtos, y una respuesta en la palanca derecha revelaba el estímulo asociado con el programa en efecto.
Respuestas de observación durante el componente de reforzamiento producían su estímulo durante 5 s; respuestas en el componente EXT producían su estímulo asociado, interrumpido por un cambio en el segmento o en el programa en efecto.
Para el grupo control, responder en EXT producía el tono continuo durante 5 s.

Estas condiciones estuvieron en efecto por 30 días.

\subsection{Results}




\end{document}
