\documentclass[a4paper,12pt]{article}
\usepackage[utf8]{inputenc}
\usepackage[T1]{fontenc}
\usepackage[spanish]{babel}
\spanishdecimal{.}
\usepackage{csquotes}
\usepackage{anysize}
\usepackage{graphicx}
\marginsize{25mm}{25mm}{25mm}{25mm}

\title{Observing responses: maintained by good news only?}
\author{Alan Silberberg \and Edmund Fantino}
\date{2010}

\begin{document}
{\scshape\bfseries \maketitle}

¿Las respuestas de observación son mantenidas por su correlación con el reforzamiento primario, o porque proveen información sobre su disponibilidad?
La prueba crítica es ver si un estímulo asociado con extinción es reforzante, y la evidencia indica que no.

Se ha mostrado que cuando la producción de malas noticias correlaciona con la oportunidad de descansar durante una tarea difícil o de involucrarse en otra actividad, las malas noticias son también un estímulo discriminativo para reforzamiento.

Para humanos se ha mostrado que el $S^{-}$ es reforzante cuando su ausencia correlaciona con la proximidad de reforzamiento.
Se pretende descubrir si el mismo efecto ocurre en aves.
Se les expuso a dos condiciones en las que información positiva y negativa estaban disponibles concurrentemente.
En una, las malas noticias estaban indirectamente correlacionadas con buenas noticias, mientras que las buenas noticias no estaban bien correlacionadas con la presentación de reforzamiento.
¿Se preferirán las malas noticias en una condición donde la buenas noticias se pueden inferir de ellas?

\section{Methods}

Se utilizaron cuatro palomas de sexo y raza desconocida, con experiencia en programas de razón progresiva.

Se usaron dos cámaras de condicionamiento con tres teclas (pero solo se utilizó una).

Antes del experimento todas las palomas pasaron por un programa de VI 4 min.

El experimento principal fue un programa múltiple VT 4 min VT 4 min.
Para dos palomas un componente era señalado por una luz verde en la tecla central y otro por una luz roja. La finalización de un componente era controlada por un programa VT 2 min.
Al acabar un componente toda iluminación era extinguida por 1 s seguido de la siguiente re iluminación.
El color encendido era elegido con probabilidad de 0.5.

Responder en presencia del componente verde ocasionaba la superimposición de tres líneas horizontales (buenas noticias) si el siguiente reforzador se entregaría dentro de los siguientes 20 s; de lo contrario, presionar la tecla no tenía efecto.
En la tecla roja el estímulo superimpuesto era una línea vertical (malas noticias) que aparecía si el siguiente reforzador estaba a más de 120 s de distancia.

En las otras dos palomas era similar excepto porque (a) los programas eran señalados por luces amarillas y verdes; (b) el encendido de las buenas noticias ocurría solo si ocurría una respuesta en la tecla amarilla cuando el siguiente reforzador estaba a menos de 120 s de distancia; y (c) el encendido de las malas noticias ocurría solo si se respondía a la tecla verde cuando el siguiente reforzador estaba a más de 20 s de distancia.

Antes de la segunda condición los sujetos pasaron por dos sesiones de pre entrenamiento.
Después, las aves pasaron a las mismas contingencias que en la primera condición salvo porque los estímulos señaladores fueron revertidos.

En la tercera condición cada par de palomas fue expuesto a las contingencias del par complementario.
Para todas las aves responder en el componente de malas noticias producía la señal de malas noticias si el reforzamiento estaba a más de 20 s de distancia; responder en el de buenas noticias producía el estímulo de buenas noticias si el reforzamiento estaba a menos de 120 s de distancia.

\section{Results}

Una clave temporalmente cercana a la entrega de comida (buenas noticias) soportó más respuestas que una clave que señalaba que la comida estaba distante.
La diferencia estaba presente cuando el estímulo de buenas noticias era más informativo que el de malas noticias, pero también cuando la ausencia de malas noticias era mejor predictor del reforzamiento inminente (malas noticias más informativas que buenas noticias).
``Informativo'' se refiere en este caso a la información sobre la proximidad temporal del reforzamiento.

\section{Discussion}

Las buenas noticias mantienen la observación y las malas no.
Incluso cuando las malas noticias eran más predictivas todas las aves respondieron a mayor tasa por las buenas noticias.
Estas tasas de respuesta se mantuvieron a pesar de que l;as buenas noticias podían más fácilmente inferirse del $S^{-}$ que del $S^{+}$.

Parte de la conclusión parece ser que las aves evitan la información cuando el mensaje transmitido es negativo.
¿Cómo encaja esto en la elección subóptima, en donde las palomas prefieren la información aunque sea mayormente negativa?


\end{document}
