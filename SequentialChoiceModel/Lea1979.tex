\documentclass[a4paper,12pt]{article}
\usepackage[utf8]{inputenc}
\usepackage[T1]{fontenc}
\usepackage[spanish]{babel}
\usepackage{csquotes}
\usepackage{anysize}
\marginsize{25mm}{25mm}{25mm}{25mm}

\title{Foraging and Reinforcement Schedules in the Pigeon: Optimal and Non-Optimal Aspects of Choice}
\author{S. E. G. Lea}
\date{1979}

\begin{document}
{\scshape\bfseries \maketitle}

Si la meta de un organismo es maximizar la energía por unidad de tiempo (E/T) que recibe variando de la probabilidad $P_i$ de perseguir el tipo i de presa cuando esta es encontrada, debería apegarse a estos principios:

\begin{itemize}
	\item 1. Los valores de $P_i$ deben ser de 0 o 1; valores intermedios nunca son óptimos.
	\item 2. La preferencia debe depender solo de la razón $E_i/h_i$ entre el valor energético de la presa i ($E_i$) y el tiempo de manejo que toma perseguirla y consumirla ($H_i$).
	\item 3. $P_i$ debería ser 1 si y solo si $E_i/h_i$ excede al valor de E/T derivado de siempre perseguir a otras presas preferibles sobre i, y nunca perseguir a presas sobre las que i es preferible.
	\item 4. Como consecuencia del punto anterior, la persecución de la presa i no debería depender de su densidad en el ambiente, sino de la densidad de todos los demás tipos de presas más preferibles.
\end{itemize}

Estas predicciones se desempeñan bastante bien al predecir la conducta de los organismos. Aunque a veces son violadas, los casos así son excepciones en la literatura de forrajeo. Sin embargo, en estudios de conducta instrumental aprendida a menudo hay reportes de conducta no-óptima. ¿Por qué pasa esto? Podría ser porque los estudios de forrajeo son ecológicamente relevantes (la hipótesis de validez ecológica). Otra opción es que la diferencia esté en la estructura de la tarea (en estudios de condicionamiento se presentan alternativas simultáneas, a diferencia de los estudios de forrajeo), o en los parámetros específicos que han tenido las opciones usadas hasta ahora.

Este experimento busca distinguir las hipótesis planteando una situación similar a la de forrajeo, e introduciendo parámetros típicos de la literatura de forrajeo y de condicionamiento. La hipótesis de estructura predice conducta uniformemente óptima en esta situación, mientras que la hipótesis paramétrica predice que la conducta será óptima con los parámetros típicos de forrajeo y subóptima con los demás. Las predicciones de la hipótesis de validez ecológica dependen de lo que se considera importante para hacer a una situación ecológicamente válida. Si el ambiente físico es importante, la conducta debe ser uniformemente subóptima, pues la caja de Skinner es artificial. Si la validez es más compleja que eso, este experimento no puede desacreditar esa hipótesis.

{\scshape\bfseries Procedimiento}

Las sesiones comenzaban con un IF $t$s en la tecla central (solo se usaban la cental y la derecha). Al cumplirlo se entraba a la fase de elección. Al entrar en ella se podía presentar uno de dos ``tipos de presa'', lo que encendía la tecla derecha en verde o rojo. Entonces la paloma podía (a) continuar respondiendo en la tecla central (tras tres picoteos se reiniciaba la fase de búsqueda), (b) dejar de responder por completo, lo que tenía eventualmente el mismo efecto que responder en la tecla central, o (c) picar en la tecla de color, lo que iniciaba la fase de manipulación, apagando la tecla central e iniciando un segundo programa de IF con $h_{\mbox{\footnotesize S}}$ o $h_{\mbox{\footnotesize L}}$ segundos, dependiendo del tipo de presa. Completar este IF resultaba en $E_{\mbox{\footnotesize S}}$ o $E_{\mbox{\footnotesize L}}$ segundos de acceso a comida, que en el caso de una de las presas eran seguidos de un tiempo $d$s de detención en el que ninguna tecla era encendida.

Así, las presas diferían en color de tecla, requerimiento de IF en estado de manejo, tiempo de acceso a comida y presencia o ausencia de detención. $h_{\mbox{\footnotesize S}}$ siempre era de 5 segundos, y $h_{\mbox{\footnotesize L}}$ era de 20 segundos; $E_{\mbox{\footnotesize S}}$ y $E_{\mbox{\footnotesize L}}$ eran normalmente de 2-5 segundos, pero variaron en algunas condiciones; y $d$ era normalmente de cero, pero varió en algunas condiciones.

Las sesiones tenían 80 ensayos (80 entradas al estado de búsqueda).

El experimento evaluó varias condiciones con parámetros de $t$, $p$, $E$ y $d$ distintos

{\bfseries Probando los efectos de la densidad general de la presa}

En el primer bloque de condiciones se evaluó el efecto del parámetro $t$, el valor de IF del estado de búsqueda. La meta era probar la elección mientras los valores de densidad de presa cambiaban entre valores altos ($t$ baja) y valores bajos ($t$ alta). El valor de $t$ comenzó en 20s y luego fue subido y bajado para crear un rango de probabilidades de aceptar una presa.

{\bfseries Probando los efectos de densidad de un tipo de presa cuando la densidad del otro se mantiene constante}

La tasa de encuentro con la presa de mayor tiempo de manejo por unidad de tiempo pasada en la etapa de búsquda puede calcularse como $p/t$, y la tasa para la otra presa es igual a $(1-p)/t$. Estas cantidades son llamadas {\slshape long density} y {\slshape short density}. Al manipular $p$ y $t$ estas densidades se podían ajustar individualmente. Primero, en el bloque 2, la {\slshape long density} se mantuvo constante en 0.025 ocurrencias por segundo en estado de búsqueda, mientras la {\slshape short density} se varió a 0.003, 0.025 y 0.225. Después, en el bloque 3, se hizo lo contrario manteniendo a la {\slshape short density} constante en 0.025 y variando {\slshape long density} en los mismos valores.
En el bloque 4 se hizo un diseño factorial en el cual ambas densidades tomaron valores de 0.10 y 0.011.

{\bfseries Probando la efectividad relativa del tamaño y demora de la recompensa}

Con $p$ ajustada a 0.5, $d$ a cero, y $t$ a 5 segundos, se corrió una condición con $E_{\mbox{\footnotesize S}}$ y $E_{\mbox{\footnotesize L}}$ en 2.5 segundos. Luego esto se comparó con otra condición con iguales valores de $p$, $d$ y $t$, en la que $E_{\mbox{\footnotesize S}}$ fue ajustado a 2 segundos y $E_{\mbox{\footnotesize L}}$ fue ajustado a 8 segundos. Dado que las palomas podían obtener al menos cuatro veces más grano en la segunda condición que en la primera, los tipos de presa se ajustaron para hacerlas equivalentes en $E/T$. Estas condiciones eran el bloque 5.

{\bfseries Probando el efecto de la detención post-recompensa}

Con $p$ en 0.5, $t$ en 5 segundos, y $E_{\mbox{\footnotesize S}}$ y $E_{\mbox{\footnotesize L}}$ en 2.5 segundos, se corrió una condición con $d$ igual a 30 segundos en la presa con menor tiempo de manejo. Esto buscaba hacer que la presa con el mayor tiempo de manejo fuese la más redituable en términos de $E/T$. Para comparación se corrió otra condición con todos los parámetros idénticos salvo por $d$ que fue ajustado a cero. Estas condiciones eran el bloque 6.

{\scshape\bfseries Resultados}

{\bfseries Observaciones generales en la conducta}

Todas la aves picaban de inmediato la tecla asociada con el menor tiempo de manejo pre-recompensa. Con la tecla asociada con el mayor tiempo hubo mayor variabilidad: algunas aves pausaban algunos segundos antes de picar, y llegaban a picar la tecla del centro un par de veces antes de responder finalmente en la tecla derecha ({\itshape\bfseries Nota de Daniel: esto se parece mucho al supuesto de SCM, que asume que la demora a la respuesta es un correlato de la tendencia a rechazar la alternativa}). Las respuestas en la tecla derecha eran estables una vez que las aves se comprometían con ella, salvo por algunos casos de aceleración al acercarse el final del intervalo. En la tecla central, el patrón de respuesta era el característico de un programa FI.

La probabilidad de aceptar la presa de menor tiempo de manejo siempre fue de 1, pero la probabilidad de la otra presa no, lo que se desvía de los supuestos de optimalidad mencionados antes. Esta probabilidad $p$ de aceptar la presa de mayor tiempo de manejo es la variable dependiente principal del experimento.

{\bfseries Efecto de la densidad general de la presa}

Como predice el forrajeo óptimo, hubo un incremento en $p$ (probabilidad de aceptar la presa con tiempo largo) conforme se incrementaba $t$ (el valor del intervalo fijo de la tecla central). Sin embargo, el cambio no fue exactamente el predicho (de 0 a 1 al pasar el umbral de 7.5s).

{\bfseries Efecto de la densidad de una presa cuando la densidad de la otra se mantiene constante}

Cuando la densidad de la presa con el mayor tiempo de manejo se mantivo constante en 0.025 Hz mientras que la densidad de la otra se varió de 0.025 a 0.225 Hz, la probabilidad $p$ bajó sistemáticamente para todas las aves.

El efecto de variar la densidad de la presa de mayor tiempo mientras la densidad de la otra presa se mantuvo en 0.025 fue mucho menos consistente.
Incrementos tanto en la densidad de la presa con tiempo corto como en la presa con tiempo largo llevaron a decrementos en $p$ (probabilidad de aceptar la presa de tiempo largo), pero el efecto de la densidad de la presa de tiempo corto fue significativamente más fuerte.

{\bfseries Efecto de la variación de la duración de la recompensa}

Todas las aves mostraron un aumento en $p$ al presentarse un acceso desigual a la comida (más tiempo de acceso en la alternativa de mayor tiempo de manejo). La probabilidad de aceptar la presa de menor tiempo permaneció siempre en 1.

{\bfseries Efecto de imponer un intervalo de detención post-recompensa}

Cuando a la alternativa con menor tiempo de manejo le fue impuesto un tiempo de detención de 30 s, haciendo que su tiempo total de manejo fuese mayor, el valor promedio de $p$ incrementó ligeramente (aunque para dos aves, decrementó). La presa con el menor tiempo de manejo fue siempre aceptada salvo por un ensyo de una sola ave.

{\scshape\bfseries Discusión}

Hay resultados consistentes tanto con forrajeo como con condicionamiento operante. De las hipótesis presentadas al comienzo, se desacreditan las de estructura y la del ambiente físico. Resta la hipótesis paramétrica.

{\bfseries Resultados consistentes con la optimalidad}

La virtualmente universal aceptación de la presa con el menor tiempo de manejo pre-recompensa, el incremento en $p$ cuando el FI de la tecla central incrementó, el dominio de la presa de tiempo de manejo corto por encima de la otra al determinar $p$.

{\bfseries Desviaciones de la optimalidad}

La preferencia de las aves era estocástica en lugar de exclusiva, es decir, $p$ no se restringió a los valores de 0 y 1. Esto se ha visto tanto en forrajeo como en estudios de laboratorio ({\itshape e.g.,} ley de igualación).

Las aves mostraron un sesgo por el rechazo de la alternativa de mayor tiempo de manejo. Cuando el FI de la tecla central era de 7.5 con los dos tipos de recompensa equiprobables, el tiempo promedio a la siguiente recompensa sería el mismo ya sea que se acepte la alternativa con mayor tiempo de manejo o se rechace indefinidamente hasta llegar a una presentación de la otra alternativa. Así, se podría predecir que la probabilidad $p$ sería de 0.5, pero en la realidad fue consistentemente menor a 0.5.

Las palomas se vieron afectadas por la densidad del peor tipo de presa, cuando se asume que solo la densidad de el resto de presas debería ser importante.

La inmediatez dominó sobre la duración de la recompensa. Esto es consistente con otros resultados en situaciones operantes de elección, pero la consistencia con la literatura de forrajeo es más difícil de establecer, pues usualmente la preferencia por la presa más grande suele darse por hecha.

El último fallo de la optimalidad es el fallo de la detención post-recompensa en compensar por las diferencias en tiempos de manejo. No hay paralelo en la literatura de forrajeo.

{\bfseries Los orígenes de la conducta óptima}

Se necesita un solo marco de referencia que explique tanto la conducta óptima como la subóptima. Los fallos en optimalidad en este caso parecen caer en dos categorías: fallos debidos al muestreo y fallos debidos a la preferencia temporal.

En el primer grupo se colocaría la ocurrencia de preferencia estocástica en lugar de exclusiva, y el efecto de la densidad de la presa de mayor tiempo de manejo sobre $p$. Ambos se explican si se asume que las palomas monitorean conductas alternativas cuando ésto no es muy costoso.

La preferencia temporal o preferencia por la inmediatez describe el dominio de la demora sobre la duración de la recompensa, y la ineficacia de la detención post-reforzamiento. También podría explicar por qué las palomas muestran un sesgo para rechazar a la presa de mayor tiempo de manejo.

Cabe resaltar que tanto el monitoreo de otras alternativas como la preferencia por la inmediatez pueden resultar adaptativas. Puede ser que estas aparentes limitaciones de la optimalidad reflejen los límites de una aproximación de un solo atributo al forrajeo. 

\end{document}
