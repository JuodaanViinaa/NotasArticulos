\documentclass[a4paper,12pt]{article}
\usepackage[utf8]{inputenc}
\usepackage[T1]{fontenc}
\usepackage[spanish]{babel}
\usepackage{anysize}
\marginsize{25mm}{25mm}{25mm}{25mm}

\title{Choosing fast and simply: Construction of  preferences by starlings through parallel option valuation}
\author{Tiago Monteiro, Marco Vasconcelos, Alex Kacelnik}
\date{2020}

\begin{document}

{\scshape\bfseries \maketitle}

Se suele aceptar la idea de que hay dos mecanismos de decisión en los humanos: uno rápido, compartido con otras especies, y uno lento que involucra evaluación cognitiva. Igualmente, hay mucha difusión en la idea de que las preferencias no existen fuera del contexto de elección, que se construyen {\slshape on the spot} a través de una evaluación comparativa. Se propone que investigar la toma de decisiones en otras especies con referencia a hipótesis desarrolladas para humanos, y viceversa, puede ayudar a generar una teoría más integral de la toma de decisiones.

Los experimentos de los autores con starlings prueban que:
\begin{enumerate}
	\item Una jerarquización por {\slshape direct rating} (una respuesta graduada fuera de un contexto de elección, como el tiempo de reacción) predice la preferencia en elección.
	\item Las latencias son menores en elección, lo que se contrapone a la noción de que un proceso de comparación cognitiva ocurre en el momento de la elección.
	\item Las elecciones de los sujetos no son irracionales, sino que se pueden interpretar en términos de {\slshape profitability ranking}.
\end{enumerate}

Estos experimentos, al ser en no-humanos, evalúan la llamada {\slshape elección por experiencia}, en lugar de la {\slshape elección por descripción}, lo que indica que las irracionalidades de la elección en humanos podrían ser endémicas del sistema de elección peculiar humano.

El modelo de encuentros secuenciales propone lo que ya sabemos: la valoración de las alternativas es independiente. El encentro simultáneo de ellas dispara dos procesos paralelos de ``aumulación''. El primero de ellos en cruzar un cierto umbral se manifiesta conductualmente y censura al otro.

En su experimento evaluaron a starlings en un ambiente con 6 estímulos encontrados secuencialmente o en pares. Los estímulos tenían distinto {\slshape profitability ratio} (combinación particular de magnitud y demora, por ejemplo, 1 pellet/10 segundos, 2 pellets/20 segundos) y {\slshape profitability} (el resultado de esa división, 0.1 pellets/segundo en ambos casos). Los 6 estímulos resultaban en 15 combinaciones posibles, y además usaron ensayos de pico para medir la discriminación temporal.

Como se esperaría según SCM, una mayor {\slshape profitability} estaba asociada con latencias menores y mayor preferencia en eleción. Además, la fuerza de la preferencia era modulada tanto por la {\slshape profitability} de la opción en cuestión como la de la alternativa con que se comparaba. También se encontró evidencia de acortamiento en las latencias durante los ensayos de elección comparados con los ensayos secuenciales.

Se evaluó la noción de que los organismos prefieren las ganancias pequeñas e inmediatas sobre las grandes y demoradas. En las comparaciones en las cuáles se equiparó la {\slshape profitability} pero el {\slshape profitability ratio} fue distinto (e.g., 1 pellet/10s vs 2 pellets/20s), los animales mostraron la preferencia opuesta: preferían la recompensa más grande y demorada. Esto contradice ideas sobre la impulsividad.

En resumen, se mostró {\slshape invarianza del procedimiento} entre {\slshape direct rating} y elección cuando se usa la latencia como métrica de {\slshape direct rating}. Esto contrasta con los fallos de invarianza de procedimiento en humanos, aunque éstos son típicos más de experimentos basados en la descripción que en la experiencia (imposibles con no-humanos). No es posible saber si esas fallas implican diferencias entre especies en mecanismos de toma de decisiones o si son solo una particularidad de la elección por descripción.

Esas consideraciones podrían resolverse llevando el procedimiento a humanos. Si los resultados son similares, se podría deber a que los mismos mecanismos son relevantes en humanos, pero solo en aquellas situaciones que disparan el uso del llamado ``sistema 1'', el que es compartido con los animales no-humanos. Estos mecanismos podrían aun así ser irrelevantes en situaciones que invocan al ``sistema 2''.

\end{document}
