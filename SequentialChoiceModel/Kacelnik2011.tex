\documentclass[a4paper,12pt]{article}
\usepackage[utf8]{inputenc}
\usepackage[T1]{fontenc}
\usepackage[spanish]{babel}
\usepackage{anysize}
\marginsize{25mm}{25mm}{25mm}{25mm}
%\usepackage[colorlinks, citecolor=blue]{hyperref}
%\renewcommand{\refname}{Referencias}
%\usepackage[backend=biber,style=apa]{biblatex}
%\addbibresource{library.bib}

\title{Darwin's ``tug of war'' vs. starling's ``horse racing'': how adaptations for sequential encounters drive simultaneous choice}
\author{Alex Kacelnik, Marco Vasconcelos, Tiago Monteiro}
\date{2011}

\begin{document}
{\scshape\bfseries \maketitle}

Darwin y Franklin aparentemente tomaban decisiones basados en una ponderación de los pros y contras de cada una, en una clase de {\slshape tug of war}. Aunque es tentador pensar que algo similar sucede en los mecanismos cognitivos que guían la elección los animales y humanos, esa aseveración debe analizarse. Una extensión de ese mecanismo indicaría que el proceso de decidir se toma un tiempo que debería escalar con el número de alternativas, y que elegir una sola opicón debería se sumamente rápido.

Los autores proponen que en encuantros simultáneos se usan los mismos mecanismos que en encuentros secuenciales, de modo que no hay una comparación directa entre las alternativas, sino solo una competencia por cuál tendrá la menor latencia y será manifestada conductualmente. La latencia en este caso refleja la tendencia a rechazar una `presa' y continuar forrajeando. No se trataría de un tiempo de reacción, sino de la mezcla de tiempo de reacción con la motivación a perseguir. A través de la experiencia, los animales crean una `biblioteca' de valores subjetivos para los tipos distintos de presa, la cual genera una distribución de probabilidades de latencias que se muestrea en los encuentros posteriores.

Parte de la lógica del artículo es que los encuentros simultáneos son lo suficientemente raros para no haber producido su propio mecanismo de toma de decisiones que resulte en optimalidad, de modo que los organismos se valen del mecanismo de los encuentros secuenciales para lidiar con ellos. Así, cuando las opciones se presentan a la vez, cada una dispara un proceso independiente de muestreo de latencia.

Los modelos clásicos asumen un balance entre la rapidez y la precisión en la toma de decisiones. Al comienzo es óptimo tomar decisiones rápidas para ganar conocimiento sobre las alternativas. El tiempo dedicado a esa exploración inicial debería aaumentar con el número de alternativas. Y al ya conocer las consecuencias asociadas con lada alternativa, también debería haber un tiempo incrementado de procesamiento de la información cuando se presentan simultáneamente. Esa demora representa un costo que no tiene sentido pagar cuando hay solamente una alternativa presente.

Se asume que los animales van formándose una valoración de las alternativas según experimentan sus consecuencias. Esta valoración depende de su magnitud, demora, del estado del propio animal y de cómo la alternativa se compara con el resto del ambiente disponible. Así, el conflicto que dio forma a los mecanismos de decisión no estaría entre rapidez y precisión, sino entre aceptar cada alternativa y rechazarla a favor del resto del ambiente.

Ambos modelos (tug of war y sequential choice) asumen que, al encontrar una alternativa, un conjunto de neuronas dispara a una tasa $R_i$ donde i indica la opción. Esta tasa depende de la valoración que el sujeto hace de la alternativa y es función del aprendiazaje. $R_i$ se integra en una cantidad $S_i$ que representa la tendencia instantánea a responder en la alternativa. Al responder y desaparecer el estímulo, el valor de $S_i$ regresa a cero. La latencia entre el inicio del estímulo y la respuesta varía debido a lo estocástico de la respuesta de las neuronas, por lo que se construye una distribución de latencias para cada alternativa.\\*
Aunque ambos modelos coinciden en encuentros individuales, difieren en los simultáneos.

{\scshape Tug of war}

En este modelo, se computa la diferencia entre sus valores de $S_i$. El sujeto evalúa el atractivo relativo de cada alternativa, y cuando esta excede un cierto umbral, actúa. Dado que ambas señales compiten, se puede ver como si jalaran en direcciones opuestas. Así, si ambas alternativas fuesen igualmente valiosas, el sujeto no decidiría nunca y quedaría en el dilema de {\slshape Buridan's ass}.

El centro del argumento es que un proceso comparativo tendrá implícito un costo en tiempo de decisión. Esta noción se cumple en cierta medida con los humanos.

{\scshape Sequential choice model}

Este modelo no incluye comparación, sino censura mutua entre las alternativas. Cada alternativa alcanza un umbral independiente pasado el cual, el animal responde, todas las alternativas desaparecen, y su valor regresa a cero. Este modelo tiene como consecuencia el acortamiento de las latencias cuando se presentan elecciones simultáneas, pues las latencias más grandes de ambas distribuciones son censuradas con mayor probabilidad que las latencias más pequeñas, y ese acortamiento debería ser más severo para la opción menos preferida, pues se ve más duramente censurada.

Las predicciones que hace este modelo son: (a) la valoración depende del aprendizaje y la comparación de la alternativa con el ambiente, (b) habrá latencias menores dada la censura mutua, y (c) las elecciones simultáneas deberían poder predecirse con las latencias secuenciales.

Suponiendo una elección entre A y B con distribuciones de latencias $f_A$ y $f_B$, con distribuciones acumuladas $\Phi_A$ y $\Phi_B$. La probabilidad de elegir A es la probabilidad $P_A$ de que una muestra de $f_A$ sea menor que una muestra de $f_B$, y está dada por $$P_A=p(l_A<l_B)=\int_0^\infty f_A(x)\cdot[1-\Phi_B(x)]dx$$ donde $l_A$ y $l_B$ son muestras aleatorias de sus distribuciones y $x$ es un valor de latencia particular.

En pruebas empíricas, se ha encontrado que las latencias para una alternativa varían según qué tan redituable es la otra alternativa con la que se compara, incluso cuando se presenta de manera independiente: una misma alternativa en un contexto rico resultaba en una latencia mayor que en un contexto comparativamente pobre. Al ser una evaluación intra sujetos, se hace evidente que se trata de un efecto del aprendizaje, con lo que se cumple el principal supuesto del modelo (sacado de Shapiro, 2008).

En un experimento distinto (Freidin, 2009) aportó evidencia sobre el supuesto de que las latencias reflejan una tendencia a rechazar una alternativa a favor del resto del ambiente: se presentaron tres opciones, una de ellas (A) entregaba comida con demora de 1s; otra (B), con demoras de 4 a 24 segundos entre condiciones, y una más (R) les permitía rechazar la alternativa presente. Los sujetos rechazaban en mayor medida la alternativa B cuando ésta tenía una mayor demora, lo que correlacionaba con mayores latencias de respuesta.

Sin embargo, estas predicciones son también compatibles con el modelo de {\slshape tug of war}. Las predicciones contrastantes serían el acortamiento de latencias (especialmente de la alternativa menos preferida) y la predicción de la elección simultánea con base en los encuentros secuenciales. Ambas predicciones fueron apoyadas por Shapiro (2008). 

Sin embargo, el modelo podría no ser sensible a las fluctuaciones locales en la valoración de las alternativas. Para comprobar que el modelo fuera también eficaz para predecir elecciones particulares y no solamente proporciones de elección (es decir, predicciones locales además de globales) Vasconcelos el al. (2010) utilizaron las últimas cuatro latencias a cada alternativa encontrada individuamente como predictor de las preferencias simultáneas, lo que resultó en un 84\% de predicciones acertadas; y el porcentaje aumentó al utilizar más latencias como predictor.

{\scshape Conclusión}

La diferencia principal entre los modelos estaría en el costo temporal añadido por las operaciones que llevan a la elección. Se favorece a un modelo en el que las alternativs sean procesadas de forma independiente, más como una carrera que como una batalla.

Se ha propuesto que en elecciones distintas de la selección de presas, los animales podrían encontrarse a menudo con decisiones simultáneas, y que mecanismos evolucionados en tales contextos podrían extenderse al contexto del forrajeo con presas simultáneas, de modo que sí habría un mecanismo específico para tales situaciones. Esto podría ocurrir, por ejemplo, en la selección de pareja.

La relación entre las latencias secuenciales y la elección simultánea parece ser causal y no solamente correlacional, dado que se puede predecir la elección con la latencia pero no al revés. Los ensayos secuenciales contienen más información que los simultáneos.

Estos datos no son definitivos. Podría haber un protocolo distinto, u otra especie, que dé evidencia de una competencia directa entre las alternativas. Por ello, es necesario evaluar el modelo en más protocolos experimentales.

La latencia extendida cuando el ambiente es más rico podría reflejar una tendencia a observar el resto de las alternativas antes de tomar una decisión.


%\printbibliography

\end{document}
