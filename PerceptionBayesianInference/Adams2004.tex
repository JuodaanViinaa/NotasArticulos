\documentclass[a4paper,12pt]{article}
\usepackage[utf8]{inputenc}
\usepackage[T1]{fontenc}
\usepackage[spanish]{babel}
\usepackage{csquotes}
\usepackage{anysize}
\usepackage{graphicx}
\marginsize{25mm}{25mm}{25mm}{25mm}

\title{Experience can change the ``light from above'' prior}
\author{Wendy J. Adams \and Erich W. Graf \and Marc O. Ernst}
\date{2004}

\begin{document}
{\scshape\bfseries \maketitle}

La aproximación Bayesiana puede describir el desempeño en tareas perceptuales en las cuales la información de los estímulos se combina con presunciones a priori. Pero no es del todo claro si los priors visuales están ``pre-cableados'' o si se aprenden como respuesta a las estadísticas ambientales.

Se investiga la adaptabilidad del prior de ``{\itshape light from above}'' agregando información sobre la forma mediante realimentación táctil.Se prueba si el mismo prior se puede generalizar a otras situaciones o si es específico a su tarea.

Inicialmente cada sujeto juzgó estímulos de diferentes orientaciones como cóncavos o convexos. El pico del prior de posición de iluminación fue inferido con los datos y se encontró que estaba aproximadamente arriba de los observadores.

Estímulos visuales-hápticos de entrenamiento fueron consistentes con un rango de posiciones de la fuente de luz cuya media fue movida aproximadamente 30° desde el prior base para cada sujeto. Estímulos visuales en este rango se combinaron con realimentación táctil que indicaba que el estímulo era una protuberancia convexa. Otras orientaciones se combinaron con realimentación táctil cóncava. Así, estímulos antes juzgados como convexos ahora se sentían cóncavos y viceversa. Los observadores exploraban un conjunto de estímulos por un tiempo ilimitado antes de juzgar la forma de un estímulo solamente visual subsecuente.

La realimentación táctil removió la ambigüedad durante el entrenamiento. Después de éste, los sujetos juzgaron un conjunto de estímulos solo visuales idénticos a los de la línea base para inferir su prior post-entrenamiento. El grupo que entrenó con un desplazamiento de +30° resultó con un desplazamiento de +8.9°. El grupo que entrenó con -30° resultó con -13.1°. El efecto fue significativo.

Hay dos explicaciones para este efecto: los observadores pueden haber aprendido implícitamente que la posición promedio de la fuente de luz se había movido en la dirección entrenada, es decir, el prior de la posición de luz cambió. Si este es el caso se debería ver un efecto de transferencia en otra tarea que involucrase al mismo prior. La otra posibilidad es que los observadores pueden haber aprendido directamente la relación entre el patrón de iluminación y la forma, o pudieron adoptar una estrategia cognitiva para etiquetar a los objetos como cóncavos o convexos. En este caso no se esperaría ninguna transferencia. Para discriminar las posibilidades se realizó otro experimento que implica al mismo prior.

Los observadores juzgaban cuál de dos paneles grises era más claro. La orientación del estímulo y la luminancia relativa de los paneles cambiaba de ensayo a ensayo. No había información explícita sobre la iluminación en la escena, pero el punto de equiluminancia subjetiva de cada observador cambiaba con la orientación de una forma consistente con la presunción de que el estímulo se iluminaba desde arriba.

Para identificar efector de transferencia los observadores repitieron los juicios después de completar la tarea visual-háptica del experimento 1. El grupo entrenado con un desplazamiento de -30° obtuvo ahora un cambio de -17.6° en la dirección inferida de la luz. El grupo de +30° obtuvo 13.8°. Este efecto fue significativo. Por lo tanto los efectos del experimento 1 fueron resultado de un cambio en la posición asumida de la fuente de luz. El segundo experimento hace suponer que el sistema visual usa el mismo prior en distintas tareas.

Los humanos, a diferencia de los pollos, pueden modificar el prior de ``luz-desde-arriba''. Un breve entrenamiento háptico resultó en un cambio sustancial en la posición inferida de la luz. Se espera que este cambio desaparecería rápidamente cuando los sujetos regresen al mundo real.

En conclusión, parece ser que los priors se actualizan constantemente en un sistema adaptable que monitorea las estadísticas del ambiente.

\end{document}
