\documentclass[a4paper,12pt]{article}
\usepackage[utf8]{inputenc}
\usepackage[T1]{fontenc}
\usepackage[spanish]{babel}
\usepackage{csquotes}
\usepackage{anysize}
\marginsize{25mm}{25mm}{25mm}{25mm}

\title{Paradoxical choice in rats: Subjective valuation and mechanisms of choice}
\author{Andrés Ojeda, Robin A. Murphy, Alex Kacelnik}
\date{2018}

\begin{document}

{\scshape\bfseries \maketitle}

La significancia biológica de maximizar la tasa de ganancia de la comida es muy clara, pero no es el mismo caso para la ganancia de información, especialmente cuando esta no tiene una utilidad directa para el organismo. Aun así, la información post-elección es relevante teórica y empíricamente en los humanos y otros organismos. Por qué esta información inútil debería valorarse es una cuestión interesante.

En subcho, las aves escogen la información a pesar de que sea inútil y lleve implícito un costo en reforzamiento obtenido. Se ha propuesto que esto es señal de que sus mecanismos son maladaptativos, y se ha usado como evidencia en contra de la teoría de forrajeo óptimo, aunque ésta lidia con ambientes naturales y no artificiales.

Otra perspectiva indica que esta conducta se debe a una disparidad entre el ambiente natural y el de laboratorio, argumentando que la misma preferencia es óptima en ambientes naturales en los cuales la información de consecuencias negativas permite abandonar una búsqueda y comenzar otra nueva.

{\itshape Encontrar el mismo fenómeno en un mamífero indicaría que se trata de una adaptación a una característica común de todos los escenarios de forrajeo, en lugar de una limitación maladaptativa en la capacidad cognitiva de las aves}, aunque no se ha conseguido aún con ratas. Aquí se proponen modificaciones que pretenden resolver la información conflictiva que se ha encontrado (Chow, Martínez): 
\begin{enumerate}
	\item Se usan estímulos auditivos como señales, pues se adaptan bien a las capacidades de discriminación de las ratas.
	\item Se usan estímulos distintos para las cuatro contingencias posibles: dos para cada alternativa.
	\item Se prueba la discriminación antes de probar las preferencias.
	\item Se varía la magnitud de las recompensas asociadas con la opción informativa.
\end{enumerate}

Además, se busca probar si el modelo de elección secuencial (SCM) también aplica para las ratas. Para ello se registran las latencias de respuesta en ensayos forzados y de elección.

{\scshape\bfseries Procedimiento}

Se usaron dos palancas como operandos de elección. Las probabilidades de reforzamiento de la alternativa informativa fueron de 0.2, 0.4 y 0.5 en condiciones distintas. La probabilidad de la alternativa no informativa se mantuvo en 0.5. Las claves eran sonidos de 10s de duración. Entrenaron las respuestas a las palancas con contingencias pavlovianas.

Para analizar las latencias con respecto al SCM, formaron pares de latencias a ambas alternativas en ensayos forzados de acuerdo con el orden en el que se presentaron. Esos pares fueron usados para preferir la proporción de eleción, que se definía como la proporción de pares en la cual la opción informativa tenía una latencia menor que la no informativa. Después se hizo una prueba de permutaciones: hubo 672 sesiones, cada una con preferencias predichas y observadas. Esas preferencias se aleatorizaron en $10^6$ permutaciones, y para cada permutación se determinó el porcentaje de sesiones en el que la preferencia observada y predicha coincidían. La probabilidad de la proporción observada bajo la hipótesis nula fue calculada como la fracción de permutaciones con un mayor porcentaje de coincidencias que la coincidencia real (no aleatorizada).

Para probar el supuesto del acortamiento de las latencias en ensayos de elección dado por el SCM, se determinó la diferencia entre latencias de ensayos de eleccióny forzados en función del nivel de preferencia por la alternativa. Dado el sesgo en las latencias y que el tamaño de las muestras depende de la elección de los sujetos, se utilizó el índice de latencia, que toma valores cercanos a 0 cuando más latencias en ensayos de elección para una alternativa son menores que la latencia mediana en ensayos forzados de la misma alternativa; cercanos a 1 cuando más latencias en ensayos de elección son mayores que la mediana de latencias de ensayos forzados para la misma alternativa; y de 0.5 cuando los ensayos forzados y de elección tienen la misma latencia mediana. El modelo de elección secuencial predice valores menores a 0.5, mientras quela perspectiva tradicional de {\itshape tug of war} predice valores mayores a 0.5.

{\scshape\bfseries Resultados}

Las ratas discriminaban bien.

Las ratas preferían la alternativa informativa cuando su probaiblidad de reforzamiento era de 0.5 o 0.4, pero cuando era de 0.2, preferían la alternativa no informativa.

Las latencias de respuesta eran iguales o menores en los ensayos de elección comparados con los ensayos forzados. Cuando la preferencia era absoluta, las latencias eran idénticas, tal como se esperaba según el SCM. Cuando la preferencia era parcial, las latencias a ambas alternativas mostraron acortamiento durante los ensayos de elección. Además, el efecto de acortamiento era más fuerte para la alternativa menos preferida.

Las latencias en ensayos forzados fueron buenos predictores de la elección, aunque las preferencias predichas fueron menos extremas que las observadas. La preferencia predicha coincidió con las observada en un 88\% de las sesiones, lo cual está por encima del azar.

{\scshape\bfseries Discusión}

Aunque las ratas también prefieren la alternativa informativa en cierto grado, su preferencia no es absoluta, y se revierte cuando el costo es lo suficientemente grande. Aunque no queda claro si las ratas podían discriminar la diferencia entre 40\% y 50\% de probabilidad de reforzamiento.

La preferencia no puede explicarse por ausencia de discriminación.

La presencia de la preferencia por la información sugiere que no se llame al fenómeno elección subóptima, pues más que un fallo de las palomas es un mecanismo muy preservado filogenéticamente. Resta determinar qué diferencias entre el contexto natural y el artificial son responsables de los resultados.

Este estudio mostró, aún con cuatro estímulos (incluyendo al inhibidor condicionado) preferencia por la alternativa informativa (en cierto grado) por lo que ésta no se puede atribuir, como argumentaba Martínez, a la ausencia de un estímulo que señale la ausencia de reforzamiento. La diferencia podría estar en que Marinez le impuso a las ratas un costopor la información mayor del que están dispuestas a pagar (que en este experimento estuvo en 20\%). Otrasa posibles explicaciones podrían ser la generalización (pues en otros estudios las claves son diferenciables solo por su posición).

Los resultados llevan a los autores a concluir que el fenómeno existe en ratsa, aun si es un grado menor. Que la prefrencia por la información no sea lo bastante grande para sobreponerse a la pérdida del reforzamiento podría ser una auténtica diferencia entre especies. El argumento de Vasconcelos es que en las aves la señal de no reforzamiento no es computada en los cálculos de valor de la alternativa (lo que, en términos de aprendizaje asociativo, sería decir que la clave negativa no adquiere propiedades de inhibición condicionada). Esto lleva a predecir preferencia extrema por la información sin importar su probabilidad de reforzamiento, lo que ha sido confirmado hasta ahora en aves pero no en ratas.

{\scshape Mecanismo de elección: el modelo de elección secuencial}

La valoración de las alternativas es independiente y se puede evaluar en situaciones de no-elección. Gana el modelo SCM sobre el de {\itshape tug of war}.

La corroboración más directa de las predicciones del modelo SCM vienen de la observación de que las latencias se vieron acortadas solamente cuando había preferencias parciales y no cuando éstas eran exclusivas.

En resumen, dos hipótesis tomadas de la investigación con palomas son confirmadas en ratas: (1) las ratas sacrifican recompensas para reducir la incertidumbre, aunque esto no significa que la evitación de la incertidumbre sea un reforzador primario. El mecanismo podría estar relacionado con el descuento del valor de los tiempos muertos; y (2) el mecanismo que lleva a la devaluación de la preferencia en las situaciones de elección parece ser bien descrito por un modelo que da prioridad a los encuentros secuenciales sin postular algún mecanismo adaptado para las situaciones de elección.



\end{document}
