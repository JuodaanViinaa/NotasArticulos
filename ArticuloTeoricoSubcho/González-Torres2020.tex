\documentclass[a4paper,12pt]{article}
\usepackage[utf8]{inputenc}
\usepackage[T1]{fontenc}
\usepackage[spanish]{babel}
\usepackage{csquotes}
\usepackage{anysize}
\marginsize{25mm}{25mm}{25mm}{25mm}

\title{Suboptimal choice behavior by pigeons is eliminated when key-pecking behavior is repaced by treadle-pressing}
\author{Rodrigo González-Torres, Julio Flores,\\* Vladimir Orduña}
\date{2020}

\begin{document}

{\scshape\bfseries \maketitle}

Evaluating different species in the suboptimal choice procedure has given way to the proposition of possible mechanisms which drive suboptimality. One such mechanism is incentive salience attribution. Incentive salience is a property of some conditioned stimuli (CS) which, due to their contingent pairing with an unconditioned stimulus (US), acquire the ability to attract behavior towards them, to function as secondary reinforcers, and to arouse complex emotional and motivational states related to the receipt of the US (Robinson et al., 2018).

Incentive salience depends, generally speaking, on the biological relatedness between the CS and the US for a particular species, although there are still important individual differences among organisms of a single species.

The incentive salience hypothesis of suboptimal choice dictates that the differences found between species in the procedure are nothing more than a byproduct of differences in the incentive salience of the discriminative stimuli used for each species: lit keys are highly salient for pigeons but not for rats, while the stimuli used at first with rats had very little salience. Still, the use of levers, which are reported as having great incentive salience for rats, led to discrepant results.

It is possible that levers do not have a high enough incentive salience, thus, a different approach would be to reduce the incentive salience attributed by pigeons. This can be done by separating the discriminative stimulus from the response manipulandum, and substituting it for a response that does not resemble the natural consummatory response, since maladaptive pigeon behavior has been shown to be increased by the influence of pavlovian contingencies on pecking behavior. This influence is eliminated when the pecking response is substituted by treadle pressing.

Evidence indicates that key pecking is more sensitive to pavlovian contingencies, and less so to instrumental contingencies. As a wise old man said, “treadle pressing in pigeons is more comparable to the operant behavior shown by other species than is key pecking” (Hemmes, 1975; p 356).

It is possible, then, that the mechanism by which the treadle-pressing response diminished maladaptive behavior in the studies described above was a decrement in the incentive salience, which was favored by both the separation between the discriminative stimuli and the response manipulandum, and by the dissimilarity in topography between the operant response and the consummatory responses originally elicited by the US.

The suboptimal behavior of pigeons could be diminished by presenting ambient lights as discriminative stimuli, instead of the usual illuminated keys, and by replacing key pecking by treadle pressing as the choice response.

{\scshape\bfseries Experiment}

Pigeons were trained to treadle-press, then they were evaluated in the suboptimal choice procedure. The probabilities of reinforcement were .2 for the discriminative alternative, and .5 for the non-discriminative alternative. Pigeons ran until the data reached apparent stability, evaluated by visual inspection. Then, a reversal condition was ran.

Finally, pigeons had a run in a control condition in which they experienced the classical suboptimal choice procedure using lit keys as operants and discriminative stimuli, after which they also experienced a reversal condition.

{\scshape\bfseries Results}

Pigeons showed preference for the optimal alternative when treadle-pressing, and preference for the suboptimal alternative when key-pecking. In both cases a robust discrimination was found, and only two pigeons went against the general trend.

{\scshape\bfseries Discussion}

This study controlled for possible individual differences in the degree of suboptimality by evaluating the very same pigeons in both variants of the task, and demonstrating distinct preferences.

It would be desirable to counterbalance the order of the conditions. Unfortunately, once key-pecking is acquired, pigeons tend to peck the treadle.

These data support the hypothesis that the degree of incentive salience attributable to the discriminative stimuli is a strong determinant of preference for pigeons. Still, the same effect has not been demonstrated with rats.

Maybe levers do not elicit the same level of incentive salience attribution for rats as lit keys do for pigeons. This is supported by the finding that, while all pigeons develop sign-tracking behavior for keys, only 35\% of rats develop sign-tracking behavior for levers. Maybe the right stimulus for rats simply has not been found.

The relevant manipulation seems to be the substitution of the operant response and not the change in the stiumuli (from localized keys to ambient lights), since the literature indicates that using ambient stiumuli does not entirely remove the attribution of incentive salience from choice (Patterson and Winokur, 1973; Rashotte et al., 1977; Green and Schweitzer, 1980).

Incentive salience requires the transfer of the consummatory behaviors caused by the UC to the EC. Treadles, unlike key lights, do not allow this transfer due to their particular response topography (Timberlake and Grant, 1975).

A recent theoretical proposal is consistent with these results. Following the Behavior Systems Theory, it has been proposed that an interaction between the natural foraging sequence of an organism and the stimuli and responses available in the task is what determines optimal or suboptimal preference. Because key pecking is related to pigeons’ consummatory behavior, the stimuli that elicit it activate focal search for food, are attributed with incentive salience, and promote suboptimal preferences. Thus, treadle pressing and its associated ambient stimuli activate general search (instead of focal search), are not attributed with incentive salience, and therefore should not promote suboptimal choice. Still, this framework does not explain rats' optimal preference.

The authors attempt to explain the results using either the {\itshape associability decay model} ---by proposing that the manipulations could somehow alter the rate of decay of associability--- or the {\itshape temporal information theoretic approach} ---by speculating about a relation between incentive salience and the degree of competition between temporal information and primary reinforcement.

These results highlight the need of including the value of the conditioner inhibitor in models of suboptimal choice, since the early models, created when only data from pigeons was available, assume a null or very small contribution of the $S^-$ stimulus to preference.

\end{document}
