\documentclass[a4paper,12pt]{article}
\usepackage[utf8]{inputenc}
\usepackage[T1]{fontenc}
\usepackage[spanish]{babel}
\usepackage{csquotes}
\usepackage{anysize}
\marginsize{25mm}{25mm}{25mm}{25mm}

\title{Rats engage in suboptimal choice when the delay to food is sufficiently long}
\author{Paul J. Cunningham, Timothy A. Shahan}
\date{2019}

\begin{document}

{\scshape\bfseries \maketitle}

Within this temporal information–theoretic framework, it is possible that the reason previous experiments failed to find suboptimal choice in rats is that the value of $Df/Ds$ was not large enough to increase $w$.

A large majority of previous experiments exploring suboptimal choice in rats have used a fixed-ratio (FR) 1 IL schedule with a 10-s TL duration. This is perhaps not surprising because these values are sufficient to produce suboptimal choice in pigeons. However, evidence from the intertemporal choice literature suggests that rats might require longer delays to food than pigeons before making maladaptive choices.

According to the temporal information–theoretic model, longer TL durations should increase the delay to food at the choice point and increase the value of $D_f/D_s$, thus increasing the weight given to relative temporal information and increasing the likelyhood of suboptimal choices.

The authors test the hypothesis that, given a sufficiently long TL, rats will show suboptimality in the suboptimal choice procedure.

The suboptimal choice procedure was followed as usual. The optimal alternative had a 50\% probability of non-signaled reinforcement, while the suboptimal alternative had a 20\% probability of delivering signaled reinforcement, in both cases. TL durations were varied across conditions ranging from 10s to 50s.

The results indicated that rats could discriminate stimuli predictive of food form stimuli that did not predict food (measured by feeder entries during the TLs).

Eight out of nine rats acquired preference for the suboptimal alternative when the TL duration was at least 30s (although there were unadressed history effects).

The temporal information–theoretic model was fit to the mean suboptimal choice proportions, resulting in a good fit, which is not surprising given the number of free parameters. The purpose of this was to obtain parameter estimates which could be compared with those obtained from pigeons.

Pigeon data suggest that (1) piegons have a relatively strong bias for using temporally informative signals to make decisions, (2) the weighting mechanism of pigeons appears to be hypersensitive to $D_f/D_s$, (3) pigeons appear to be hypersensitive to the relative temporal information conveyed by the TL stimuli, and (4) pigeons appear to be only slightly hypersensitive to relative reinforcement rate.

Rats also appear to be hypersensitive to relative temporal information. Howeve, they also appear to be slightly less sensitive than pigeons to relative temporal information. Finally, {\bfseries rats appear to be more sensitive to relative reinforcement rate than pigeons}.

Previous failures to find suboptimal choice in rats have given rise to numerous hypotheses about differences in decision-making processes between rats and pigeons. The most often-cited possi- bility is that rats, unlike pigeons, are sensitive to the conditioned- inhibitory properties of the stimulus never followed by food.
However, results from the present experiment are difficult to understand in terms of sensitivity to conditioned inhibition. If a 10-s S– is sufficient to discourage suboptimal choice due to its aversive properties, then a 50-s S– should be even more aversive because it signals a longer wait-time in which food is not forthcoming.

Another possible reason offered for between-species differences in suboptimal choice is related to between-species differences in incentive salience attributed to TL stimuli. If attribution of incentive salience to TL stimuli is required to generate suboptimal choice, then rats should only engage in suboptimal choice when lever-insertions (which do acquire incentive salience) are used as TL stimuli.

Results from the present experiment might also suggest that incentive salience is not necessary for suboptimal choice, given that rats in the present experiment readily made suboptimal choices despite the use of lights and tones as TL stimuli.

Trujano and Orduña (2015), unlike the present experiment, did not find suboptimal choice with a 30-s TL duration. Perhaps the reason for this difference is that the tone + light stimuli used in the present experiment was more salient than the small LED bulbs used in Trujano and Orduña and might therefore be more prone to serve as conditioned reinforcers when established as temporally informative stimuli. Thus, stimulus modality (and incentive salience) could interact with Pavlovian factors.

We can draw tentative conclusions about potential differences in suboptimal choice between rats and pigeons. First, rats appear to have a much stronger bias against using temporally informative signals to make decisions than pigeons (i.e., m is much higher in rats than pigeons). Second, the weighting mechanism in rats ap- pears to be less sensitive to variations $D_f/D_s$ than pigeons (i.e., $\beta$ is smaller in rats than pigeons). 

In sum, the temporal information–theoretic model of suboptimal choice suggests that the bits of temporal information conveyed by TL stimuli and the rate of reinforcement afforded by suboptimal and optimal alternatives compete to control choice. Suboptimal choice arises when relative temporal information contributes more heavily to choice than relative reinforcement rate. Results from the present experiment suggest that rats, like pigeons, are susceptible to make bad decisions when they are provided an opportunity to earn bits of (temporal) information even though doing so is not in their long-term best interest.

\end{document}
