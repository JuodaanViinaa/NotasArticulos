\documentclass[a4paper,12pt]{article}
\usepackage[utf8]{inputenc}
\usepackage[T1]{fontenc}
\usepackage[spanish]{babel}
\usepackage{csquotes}
\usepackage{anysize}
\marginsize{25mm}{25mm}{25mm}{25mm}

\title{Individual differences in incentive salience attribution are not related to suboptimal choice in rats}
\author{Paulina López, Rodrigo Alba, Vladimir Orduña}
\date{2018}

\begin{document}

{\scshape\bfseries \maketitle}

One explanation for the species difference in suboptimal choice is incentive salience attribution. Although there's evidence against that notion, it is still worthy of further evaluation. To that effect, this study aims to assess the suboptimality of rats categorized as either {\itshape sign-strackers} or {\itshape goal trackers}. The hypothesis is that {\itshape sign trackers} will show greater preference for the suboptimal alternative than goal-trackers.

{\scshape\bfseries The experiment}

45 rats were evaluated using the PCA index. The top and bottom 8 were selected as sign-trackers and goal-trackers. Those rats were then placed in the probability variant of the suboptimal choice procedure, with probabilities of reinforcement of $p={.}5$ for the discriminative alternative, and $p={.}75$ for the non-discriminative alternative.

Both groups turned out to be optimal, and no significant differences were found between their choice proportions.

Their lever presses and feeder entries were analyzed in order to determine their degree of discrimination, and a robust discrimination was found.

The PCA index was calculated again at the end of the suboptimal choice procedure, and although the values had changed, there were still statistically significant differences between the groups, which means that their categorization as either sign-trackers or goal-trackers remained.

{\scshape\bfseries Discussion}

Although sign-trackers showed a higher level of pavlovian conditioned approach, they still chose optimally, much like uncategorized rats. Since PCA has been taken as evidence of a stronger capacity of attribution of incentive salience, these results argue against the relationship between this variable and suboptimal choice. This is relevant for the discussion about the discrepancy between Chow's and Martínez's results: it could be argued that, by chance, the sign-tracker sub-population was over-represented in Chow's study, and under-represented in Martínez's. But this experiment rules out that possibility.

The explanation for this difference could lie in the effect of the conditioned inhibitor. The inhibitor could have overshadowed the effect of incentive salience attribution.

Another possibility is a difference based in the number of training sessions: while Chow reported suboptimality at the 14 session mark, Martínez and this study went for as long as 40 sessions. Chow's subjects could have still been learning the contingencies when their experience was cut short.

A study by Beckman and Chow showed that rats prefer an alternative in which a lever announces the delivery of a reinforcer over another alternative in which a tone announces the same delivery, and they hold this preference in spite of the probability of reinforcement for the lever option being reduced to 50\%. While this shows that animals prefer stimuli with incentive salience, a key difference with suboptimal choice is the fact that in the latter, the same stimulus modality is used for both alternatives, which could cancel the effect of incentive salience. This notion should be explored with asymmetrical manipulations.

For pigeons, the effect of the conditioned inhibitor dissipates early in training, right as they start to choose suboptimally. Meanwhile, the conditioned inhibition effect persists in rats, which continue to behave optimally. Sign-trackers have been reported to be influenced to a greater degree by aversive consequences than goal-trackers. This could compensate the greater incentive salience attributed by them, supporting again optimal behavior.

\end{document}
