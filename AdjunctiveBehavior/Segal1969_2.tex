\documentclass[a4paper,12pt]{article}
\usepackage[utf8]{inputenc}
\usepackage[T1]{fontenc}
\usepackage[spanish]{babel}
\usepackage{csquotes}
\usepackage{anysize}
\usepackage{graphicx}
\marginsize{25mm}{25mm}{25mm}{25mm}

\title{Transformation of polydipsic drinking into operant drinking: A paradigm?}
\author{Evalyn F. Segal}
\date{1969}

\begin{document}
{\scshape\bfseries \maketitle}

Se describen los resultados de una única rata especial.

La rata corrió en una caja con una palanca, y dispensador de comida y agua. En las primeras 7 sesiones la rata recibió comida gratis cada 30, 60 y 90 segundos. Las lamidas subieron de 1 a 68 indicando desarrollo de polidipsia.

En la sesión 8 se moldeó a la rata para presionar la palanca, y por las siguientes 15 sesiones fue expuesta a programas de intervalo fijo de 1, 1.5, 2, y 3 minutos. La rata comenzó a mostrar el patrón normal de polidipsia (bebida breve tras la entrega de cada pellet). Cada bebida era seguida por presiones aceleradas a la palanca hasta la siguiente entrega de comida.

A mediados de la sesión 21 comenzó a cambiar el patrón de bebida: las lamidas comenzaron a concentrarse hacia el final del intervalo en lugar del inicio. Es decir, hubo una mezcla entre la bebida polidípsica que ocurría después de la entrega anterior y bebida operante que ocurría antes de la siguiente entrega.

El resto del método y resultados fue hecho caprichosamente y me da flojera resumirlo.

Se ha comparado a la polidipsia con la conducta de desplazamiento estudiada por lo etólogos. Estas conductas no parecen reflejas dado que involucran comportamiento esquelético altamente complejo, pero no parecen operantes dado que son evocadas y no están bajo el control de las contingencias de reforzamiento en el momento en que aparecen. Pero parece que esta conducta sí es susceptible a las contingencias operantes.

Es posible que haya un espectro entre lo claramente operante y lo claramente reflejo, y que conductas en puntos intermedios muestren niveles distintos de fuerza en su relación con los estímulos que los evocan y las contingencias que los refuerzan.


\end{document}
