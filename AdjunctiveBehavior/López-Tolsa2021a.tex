\documentclass[a4paper,12pt]{article}
\usepackage[utf8]{inputenc}
\usepackage[T1]{fontenc}
\usepackage[spanish]{babel}
\usepackage{csquotes}
\usepackage{anysize}
\usepackage{graphicx}
\marginsize{25mm}{25mm}{25mm}{25mm}

\title{Behavioral history effects on the maintenance of schedule-induced drinking in rats}
\author{Gabriela E. López-Tolsa \and Juan Ardoy \and Ricardo Pellón}
\date{2021}

\begin{document}
{\scshape\bfseries \maketitle}

La historia determina fuertemente la conducta. Una contingencia ya no presente puede aun así afectar a la conducta. La conducta inducida por el programa puede ser relevante para estos efectos de historia.

Cierta evidencia indica que la adquisición de bebida inducida depende de que la bebida sea parte de una historia conductual previa, especialmente cuando existe disponibilidad de conductas inducidas alternativas. Además, el nivel de polidipsia ya establecida puede bajar si cambian la condiciones experimentales, lo que hace preguntarse si la historia conductual puede producir cambios en el mantenimiento de la bebida inducida.

Hay dos hipótesis propuestas para estos hallazgos que, aunque basadas en mecanismos distintos, comparten la premisa de que la distribución de conductas en los intervalos entre reforzamiento depende de la ocurrencia de otras conductas. Pellón y Killeen sugieren que las distribuciones de conductas en el intervalo se moldean mutuamente dependiendo de la proximidad de cada conducta al reforzador. Baum propone que las conductas se distribuyen en un intervalo específico, y dependiendo de la inducción producida por eventos filogenéticamente importantes (PIEs) algunas incrementan llevando al decremento de otras.

Considerando esto, es posible que cuando un patrón previo de conducta se ha desarrollado, cambios en la situación pueden producir cambios en los patrones conductuales, aunque estos podrían no ser inmediatos dado que tendría que desarrollarse un nuevo patrón para integrar la nueva conducta.

Este estudio pretende investigar la contribución de la historia conductual sin la posibilidad de realizar bebida inducida en su mantenimiento subsecuente en animales que ya la habían desarrollado antes de su interrupción.

{\scshape\bfseries Method}

Se utilizaron 24 ratas con experiencia previa en FI con agua disponible. Las cajas tenían comedero, dos palancas (solo una activa), y dispensador de agua con sensor de respuestas.

Se trató de un diseño intra-sujetos con tres fases: línea base (A1), tratamiento (B), y regreso a línea base (A2).

Las ratas fueron re-expuestas a FI 15, 30 y 60 por 17 sesiones con agua disponible. Las lamidas no tenían consecuencia programada. Solo se analizaron las últimas 10 sesiones de A1. Los animales fueron asignados en pares según su tasa de respuesta en presiones y lamidas, y luego asignados a grupo experimental y control. En las 30 sesiones de B las ratas experimentales pasaron por el mismo FI pero sin agua, mientras que las controles permanecieron en sus jaulas sin entrenamiento. Después las condiciones iniciales fueron reinstauradas por 10 sesiones (el agua se regresó).

{\scshape\bfseries Results}

Ambos grupos mostraron tasas de respuesta similares en A1. En A2, las ratas del grupo EXP15 tuvieron tasas de lamida menores que las ratas control, especialmente en la primera sesión, pero diferencias menores se mantuvieron toda la fase. EXP30 y CON30 también mostraron diferencias entre sí que tendieron a disminuir. Las ratas de FI60 tuvieron diferencias leves en la primera sesión, pero estas desaparecieron rápidamente. Se calculó la diferencia en tasa de lamidas entre la última sesión de A1 y la primera de A2, y se realizaron pruebas $t$ para muestras independientes para comparar los grupos. Se encontraron diferencias solo entre los grupos de FI15 y FI30.

Las presiones a la palanca fueron similares en A1 para Fi15 y 30, pero para FI60 las ratas CON tuvieron tasas más altas. En la fase B las ratas EXP15 mostraron más presiones que en A1 y se mantuvieron durante la fase. EXP30 también mostró tasas más altas que en A1, pero su tasa disminuyó lentamente hacia el final de la fase B. EXP60 no mostró tasas más altas en B que en A1. Las diferencias entre el final de A1 y el inicio de B solo fueron significativas para EXP30.

Al inicio de A2, Exp15 y EXP30 mostraron tasa más altas de presión que en A1, pero bajaron hacia tasas similares a las de la línea base. Los demás grupos mostraron tasas similares de presión en A1 y A2. Se calculó la diferencia entre las tasas de respuesta al final de A1 y al inicio de A2 y se realizaron pruebas $t$. Se encontraron diferencias significativas solo entre los grupos de FI15 y 30.

{\scshape\bfseries Discussion}

Se intentó investigar el efecto de interpolar una historia de no-oportunidad de beber en el mantenimiento de bebida inducida ya desarrollada. Los datos indican que una vez desarrollada la bebida inducida, los sujetos con acceso al programa pero sin agua mostraron tendencia a disminuir su nivel de consumo de agua comparados con los controles. Todas las ratas bebieron de nuevo cuando el agua se hizo disponible de nuevo, aunque en niveles distintos correspondientes con sus historias conductuales, por lo que el consumo diferencial no puede atribuirse al mero paso del tiempo.

Se ha sugerido que los efectos de la historia en la adquisición son transitorios. Aquí se interrumpió la bebida durante el mantenimiento y no en la adquisición, y se observaron resultados similares en el sentido de tasas de lamida transitoriamente disminuidas.

Los hallazgos son consistentes con las hipótesis de Pellón y Killeen y Baum. Los sujetos del grupo experimental tuvieron que desarrollar un nuevo patrón conductual durante la fase B, y al volver a A2 debieron desarrollar otro más para reintegrar a la bebida inducida, mientras que los sujetos control no tuvieron esa necesidad.

Los cambios en presiones a palanca son consistentes con esas hipótesis: cuando ya no se puede realizar la bebida inducida, se espera incremento en otras conductas, pero solo cuando los intervalos entre reforzamiento son lo bastante cortos para que la bebida y la presión ocurran al mismo tiempo.

El nivel de bebida inducida fue el mismo para ambos grupos durante A1, pero en A2 los sujetos experimentales bebieron menos y los control más que en la línea base. Este incremento del grupo control puede atribuirse a un proceso por el cual la bebida inducida se adquiere más aun después de un tiempo sin la oportunidad de beber. Pero este efecto parece ser rápido, pues rápidamente se estabilizaron los niveles de bebida.

Estos hallazgos son consistentes con la hipótesis de que patrones conductuales se desarrollan en los intervalos entre reforzamiento de modo que la distribución de conducta depende de la ocurrencia de otras conductas en esos intervalos. Si el mecanismo es reforzamiento (como dice Pellón y Killeen) o inducción (como dice Baum) aun está por determinarse.


\end{document}
