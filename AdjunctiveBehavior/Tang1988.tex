\documentclass[a4paper,12pt]{article}
\usepackage[utf8]{inputenc}
\usepackage[T1]{fontenc}
\usepackage[spanish]{babel}
\usepackage{csquotes}
\usepackage{anysize}
\usepackage{graphicx}
\marginsize{25mm}{25mm}{25mm}{25mm}

\title{Prior exposure reduces the acquisition of schedule induced polydipsia}
\author{Maisy Tang \and Shelly L. Williams \and John L. Falk}
\date{1988}

\begin{document}
{\scshape\bfseries \maketitle}

Las ratas en programas intermitentes con rangos particulares de tasas de entrega desarrollan rápidamente polidipsia inducida por el programa. La polidipsia se mantiene durante meses mediante sesiones diarias y es dependiente solo de la disponibilidad de comida (y no de las necesidades de agua del animal).

Se ha preguntado si al adaptar a los animales al programa generador antes de dar acceso a agua se puede retrasar o impedir la adquisición de la polidipsia. También, dado que la sobre-indulgencia inducida por el programa en soluciones de 5\% etanol puede ser prevenida parcialmente por una historia de bebida inducida con otros tipos de fluidos, quizá la historia de no-disponibilidad de agua podría también atenuar la tasa de desarrollo y magnitud final de polidipsia de etanol al 5\%.

{\scshape\bfseries Method}

La cámara experimental contenía una palanca, un receptáculo de pellets y un ducto de entrega de agua (opcional).

El programa de entrega de pellets era de FI 1 minuto con 120 entregas. Los animales fueron divididos en dos grupos: uno de {\itshape schedule history} y uno de {\itshape home-cage history}. El grupo {\itshape schedule history} recibió una media de 128 días de sesiones en el programa sin acceso a agua. Había agua disponible siempre en las cajas habitación. El grupo {\itshape home-cage history} recibió sesiones diarias de FI 1 minuto por dos días, y luego solo fue mantenido en su peso dentro de su caja habitación durante una media de 109 días recibiendo una o dos sesiones de recuerdo al mes, y luego dos sesiones más antes de la siguiente fase experimental. Después de este período ambos grupos recibieron sesiones FI 1 minuto con agua disponible durante la sesión por una media de 57 días. En los últimos 10 días una o dos de las sesiones fueron sustituidas por sesiones de sondeo con un programa FR 1 con 120 reforzadores en el cual el animal permanecía dentro de la cámara por un período igual a su longitud usual de sesión FI 1 minuto. Estas sesiones permitían comparar la cantidad de bebida ocurrida en ellas con la bebida ocurrida en FI 1 minuto, que es conocido por inducir sobre-bebida.

En la última fase del experimento el programa FI 1 minuto permaneció en efecto pero se cambió el agua por una solución de 5\% etanol por 25-34 días. En las últimas 10 sesiones de nuevo se realizaron sesiones de sondeo.

{\scshape\bfseries Results}

El grupo {\itshape schedule history} muestra una adquisición retrasada de la polidipsia con relación al grupo {\itshape home-cage history}, además de una merma en el nivel máximo. Cuando el grupo {\itshape home cage} tardó alrededor de 8 días en llegar a su nivel máximo de 73 ml, el grupo {\itshape schedule} tardó 4-5 semanas en llegar a un nivel menor de 45.1 ml. Sus bebidas medias en las sesiones de sondeo fueron de 18 y 13.3 ml, respectivamente.

La bebida de etanol y las presiones a las palancas fueron equivalentes entre grupos, aunque en ambos casos hubo un aumento en el consumo comparado con sus sesiones de sondeo.

{\scshape\bfseries Discussion}

Los grupos adquirieron niveles diferentes de bebida a tasas diferentes, pero ambos llegaron a niveles de polidipsia. La tasa del grupo de {\itshape home-cage} es similar a la tasa reportada cuando no hay ninguna historia particular, por lo que ese grupo parece tener una respuesta polidípsica normal ante el programa FI 1 minuto.

Ha habido otros experimentos en los que se busca un efecto sobre la polidipsia de la imposición de otros tipos de historia. Permitir a las ratas comer un solo gran alimento diario no tuvo un efecto de retraso en la polidipsia. Tampoco exposición de 7 días a un programa sin acceso a agua. Ratas que recibían comida en sesiones diarias de FT 90 s por dos semanas sí mostraron una adquisición más lenta de la polidipsia en los 15 días siguientes comparadas con un grupo control, aunque ambos tuvieron niveles comparables de polidipsia al final.

Estos estudios y el presente indican que si la historia de exposición al programa generador sin acceso a agua es lo suficientemente larga se menguará la adquisición de polidipsia.

La no disponibilidad de agua puede influir en la adquisición de polidipsia al alterar la probabilidad de bebida. Esto mismo puede lograrse cargando a los animales con agua antes de las sesiones, aunque se ha mostrado que aun cargándolas con el 6\% de su peso la disminución de la polidipsia es poca. Otro estudio mostró que ratas que fueron cargadas con agua durante 5 sesiones consecutivas no adquirieron polidipsia, pero investigación posterior mostró que esto solo retrasa la adquisición.

En general, la sobrecarga de agua solo aminora la polidipsia con cargas muy grandes o cuando la función renal de los sujetos se ve artificialmente alterada de modo que ingerir más agua ocasione intoxicación.

Al inducir bebida pre-programa mediante una solución de sacarina o al inyectar agua directamente en la boca después de la ingestión de cada pellet la polidipsia puede reducirse o incluso eliminarse.

Esta evidencia indica que cuando la probabilidad de beber en los programas es reducida (haciendo menos disponible el agua o sobrecargando a los animales) se puede retrasar la adquisición y aminorar el nivel final de polidipsia. La eficacia de estas condiciones históricas puede depender de su ocurrencia bajo el control de estímulos del programa generador dado que simplemente evitar la bebida prandial (con los alimentos) o el acto de beber (haciendo que los animales obtengan sus líquidos de lechuga) no impide el desarrollo de polidipsia. 

Aunque las ingestas de los grupos fueron ampliamente distintas con agua, con el alcohol sus ingestas convergieron. Los niveles de bebida fueron similares a aquellos obtenidos mediante condiciones no precedidas por una historia particular.


\end{document}
