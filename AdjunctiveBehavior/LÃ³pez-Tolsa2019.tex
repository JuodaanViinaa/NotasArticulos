\documentclass[a4paper,12pt]{article}
\usepackage[utf8]{inputenc}
\usepackage[T1]{fontenc}
\usepackage[spanish]{babel}
\usepackage{csquotes}
\usepackage{anysize}
\usepackage{graphicx}
\marginsize{25mm}{25mm}{25mm}{25mm}

\title{Conductas adjuntivas: entre la inducción y el reforzamiento}
\author{Gabriela E. López-Tolsa}
\date{2019}

\begin{document}
{\scshape\bfseries \maketitle}

Los elementos básicos para entender el comportamiento de los organismos serían la conducta, los eventos que ocurren en el ambiente y la relación entre ambos.

El condicionamiento operante destaca como el procedimiento mediante el cual se adquieren nuevas conductas en función de sus consecuencias. Se consideraba que era necesaria una relación de dependencia entre la conducta y la consecuencia llamada contingencia para que ocurriera un incremento en la probabilidad de ocurrencia, pero fenómenos como la superstición sugieren que la contingencia no es indispensable, sino que es un modo de establecer una relación temporal consistente entre conducta y consecuencias.

Las conductas adjuntivas además retan al concepto de reforzamiento y su relación con el mantenimiento de conductas. Pueden ser una clave para entender el papel de la inducción y el reforzamiento en la conducta.

Conductas adjuntivas son las que se desarrollan en el intervalo entre reforzadores sin relación explícita entre su ocurrencia y la entrega de reforzador.

El primer fenómeno reportado fue la bebida inducida por el programa (BIP), nombrada así dado que no parecía ser reforzada por la entrega de la comida. La etiqueta de conductas adjuntivas engloba a la BIP y otros fenómenos similares, como la carrera inducida (en rueda de ejercicio), entradas al comedero, lamidas en un tuvo de aire, roer madera, acicalamiento, fumar, y otros.

Estas conductas (1) ocurren en tasas que sobrepasan su línea base, (2) se adquieren en varias sesiones, (3) en estado estable ocurren en la primera parte del intervalo entre reforzadores. (4) Aunque no tienen contingencia con la entrega del reforzador, sí necesitan que ocurra para mantenerse y (5) dependen de su magnitud y frecuencia, además, (6) requieren que el organismo esté privado del estímulo usado como reforzador y se desarrollan ante programas de tiempo y de razón, por lo que (7) no requieren el desarrollo de otra conducta, solo la intermitencia del programa.

Se han observado casos en que la conducta adjuntiva ocurre en momentos distintos del inmediato posterior a la entrega del reforzador, {\itshape e.g.,} ha llegado a ocurrir de forma sostenida o a ocurrir al final del intervalo, sustituyendo a la conducta meta. También es común que se desarrolle más de una conducta adjuntiva en ambientes con disponibilidad de múltiples actividades.

La primera hipótesis para explicar estas conductas presumía que son fundamentalmente distintas de las operantes dadas estas características. Staddon (1977) las consideró como aquellas que ocurren cuando hay una baja probabilidad de reforzamiento.

Otra hipótesis señala que son parte de sistemas conductuales específicos a la especie que son elicitados por la entrega periódica del reforzador. Los organismos muestran secuencias conductuales adaptativas dependientes de la situación que se relacionan con el sistema al que pertenece el tipo de reforzador. Por ejemplo, ocurrirán conductas relacionadas con la ingesta cuando el reforzador es comida.

También se ha sugerido que son un mecanismo para disminuir el estado emocional aversivo ocurrido en los programas de reforzamiento intermitente. El hallazgo que motivó esta hipótesis fue que el nivel de tolerancia a la frustración afecta al volumen de BIP desarrollado. Así, la conducta es mantenida por reforzamiento negativo y es operante. Aunque esos resultados no han podido replicarse.

Las hipótesis prevalecientes actualmente son las de reforzamiento (Killeen y Pellón) e inducción (Baum).

La hipótesis del reforzamiento accidental prosperó hasta que el grupo de Pellón las categorizó como operantes fortalecidas por la entrega del reforzador. Se ha evaluado el efecto de las consecuencias al establecer demoras contingentes a la conducta adjuntiva y se ha observado disminución o desaparición de dichas conductas.

Killeen y Pellón desarrollaron el modelo de {\itshape rastros competitivos} para explicar el desarrollo de conductas adjuntivas y operantes. Proponen que clases de conductas tienen distintas ventanas temporales para ser reforzadas. Las ventanas son los rastros exponenciales de proximidad que determinan el tiempo que puede pasar entre la conducta y el reforzador para que ésta sea reforzada. Las conductas adjuntivas y operantes tendrían gradients de demora diferentes, lo que determinaría el orden de distribución en el intervalo, y así varias conductas serían mantenidas por un solo reforzador.

Recientemente se ha propuesto que las conductas forman parte de un patrón conductual reforzado en su totalidad y repetido en cada intervalo.

Baum, en cambio, reta al concepto tradicional de reforzador como fortalecedor de la conducta y lo busca sustituir con el término de inducción. Inducción implicaría que eventos en el ambiente incrementan la ocurrencia de una conducta o el tiempo dedicado a ella. La presencia de un evento inductor incrementa la tasa de ocurrencia de una conducta, pero no se trata de una relación uno a uno.

Baum plantea que todas las conductas de un organismo se distribuyen en el tiempo, de modo que al aumentar unas otras disminuyen. Los eventos filogenéticamente importantes (PIE) inducen incrementos en unas y disminuciones en otras. Baum redefine también el concepto de contingencia en términos de probabilidad e indica que para que un PIE sea efectivo debe haber correlación entre él y la conducta.

Baum y Davidson diseñaron un modelo que asume que las conductas compiten jerárquicamente para distribuirse en el intervalo entre reforzadores, pero la jerarquía no queda siempre clara dado que los picotazos en palomas reemplazan a conductas relacionadas con comida, y las presiones a palanca en ratas son reemplazadas por el mismo tipo de conductas. Baum propone una visión en la cual toda situación es una elección entre la realización de diversas conductas.

Baum dice que su visión es distinta dado que no ve a la conducta como respuestas discretas adquiridas al ser fortalecidas por un reforzador inmediato, pero Pellón no limita su definición de reforzamiento a la conducta inmediatamente anterior, sino que asume que puede haber un efecto de forma demorada, o que los patrones conductuales se refuerzan como un todo.

Las conductas adjuntivas presentan una oportunidad para determinar la influencia relativa de inducción y reforzamiento para resolver si se trata de mecanismos contradictorios o complementarios.

Un experimento evaluó el efecto del establecimiento de contingencias en la BIP: hacer que cada lamida en el dispensador de agua acortara la demora hasta el siguiente reforzador incrementó el volumen de BIP, lo que indica que aunque la inducción es suficiente para establecer la BIP, establecer una contingencia fortalece la conducta.

Ambas hipótesis concuerdan en que hay competencia entre conductas, y que la preponderancia de una u otra depende del efecto fortalecedor del reforzador (Pellón) o de la inducción del PIE (Baum). Algunas conductas parte de los patrones dados por la evolución son seleccionadas por contingencias o por proximidad temporal y mantenidas a través del tiempo.

Los patrones conductuales son estables si el ambiente lo es, aunque si el ambiente introduce o niega la posibilidad de realizar alguna conducta, el patrón conductual se adapta para incorporar el cambio, lo que es consistente con la competición de respuestas planteada por ambas hipótesis.

Una diferencia entre las hipótesis está en su nivel de análisis. La inducción tiene un nivel molar en el que no se consideran ensayos específicos, sino actividades y condiciones extendidas en el tiempo, lo que es útil para estudiar conductas en estado estable, pero no permite determinar si aquello que se observa es resultado de un artefacto dados datos promedio o es un reflejo exacto de lo que ocurre.

Puede haber una postura media. Cowie indica que los reforzadores son estímulos que anticipan la entrega futura de más reforzadores. Así, las consecuencias y su control de estímulos pueden causar incremento o disminución en la conducta. Sin embargo, se ha observado que aunque la conducta puede ser inducida por el PIE anterior, si el reforzador futuro no ocurre, la probabilidad de repetir el patrón conductual disminuirá. Es decir, aunque el PIE induzca la conducta eso no implica que su presencia fortalezca o haya fortalecido la ocurrencia de dicha conducta.

López-Tolsa estudió la distribución de conducta en ensayos de pico no reforzados y observó que los patrones conductuales solo se reinician en ensayos precedidos por la entrega de reforzador, lo que indica que los patrones son repetidos por dos procesos: el reforzador anterior induce el inicio de un patrón, y el patrón es mantenido por el siguiente reforzador.

No quedan claros los mecanismos por los cuales la conducta operante se desarrolla y mantiene. Estudiar las conductas adjuntivas como ejemplo de operantes permitirá apreciar ambos efectos (inducción y reforzamiento). 


\end{document}
