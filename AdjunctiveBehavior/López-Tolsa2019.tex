\documentclass[a4paper,12pt]{article}
\usepackage[utf8]{inputenc}
\usepackage[T1]{fontenc}
\usepackage[spanish]{babel}
\usepackage{csquotes}
\usepackage{anysize}
\usepackage{graphicx}
\marginsize{25mm}{25mm}{25mm}{25mm}

\title{Conductas adjuntivas: entre la inducción y el reforzamiento}
\author{Gabriela E. López-Tolsa}
\date{2019}

\begin{document}
{\scshape\bfseries \maketitle}

Los elementos básicos para entender el comportamiento de los organismos serían la conducta, los eventos que ocurren en el ambiente y la relación entre ambos.

El condicionamiento operante destaca como el procedimiento mediante el cual se adquieren nuevas conductas en función de sus consecuencias. Se consideraba que era necesaria una relación de dependencia entre la conducta y la consecuencia llamada contingencia para que ocurriera un incremento en la probabilidad de ocurrencia, pero fenómenos como la superstición sugieren que la contingencia no es indispensable, sino que es un modo de establecer una relación temporal consistente entre conducta y consecuencias.

Las conductas adjuntivas además retan al concepto de reforzamiento y su relación con el mantenimiento de conductas. Pueden ser una clave para entender el papel de la inducción y el reforzamiento en la conducta.

Conductas adjuntivas son las que se desarrollan en el intervalo entre reforzadores sin relación explícita entre la ocurrencia y la entrega de reforzador.


\end{document}
