\documentclass[a4paper,12pt]{article}
\usepackage[utf8]{inputenc}
\usepackage[T1]{fontenc}
\usepackage[spanish]{babel}
\usepackage{csquotes}
\usepackage{anysize}
\usepackage{graphicx}
\marginsize{25mm}{25mm}{25mm}{25mm}

\title{The ``superstition'' experiment: a reexamination of its implications for the principles of adaptive behavior}
\author{J. E. R. Staddon \and Virginia L. Simmelhag}
\date{1971}

\begin{document}
{\scshape\bfseries \maketitle}

El experimento de superstición moldeó la visión teórica de Skinner de la conducta operante como el fortalecimiento de conducta emitida (impredecible) mediante la acción automática de los reforzadores. 

Las conductas inducidas plantean dudas sobre la visión de Skinner: estas conductas se desarrollan confiablemente  a pesar de que no son contiguas a la entrega de comida. Quizá algunas de las conductas que Skinner atribuyó al reforzamiento accidental de emisión espontánea tengan más bien factores causales comunes con estas ``actividades mediadoras''.

Por otro lado, la emergencia de conductas que parecen ser más Pavlovianas que operantes también plantea dudas en la concepción de Skinner: conductas ``instintivas'' parecen interferir con conductas operantes ya bien establecidas, como en el caso de Breland y Breland.

De modo similar, el procedimiento de automoldeamiento muestra una conducta que es mantenida incluso si su emisión previene la entrega de comida en una ocasión particular. Todos estos procedimientos muestran conductas relacionadas con la comida que ocurren en anticipación de la entrega de la misma y que son incompatibles con la ley del efecto.

La superstición de Skinner, como el automoldeamiento, es operacionalmente idéntica al condicionamiento Pavloviano. Quizá exposición prolongada a esa condición también tendrá como resultado la emergencia de conducta relacionada con la comida en anticipación a ésta. Quizá la superstición incluye actividades anticipatorias que ocurren antes de la comida y mediadoras que ocurren justo después.

Este experimento prueba estas ideas. Compara los efectos de intervalos fijos y variables en la conducta supersticiosa y compara programas de intervalo fijo dependientes e independientes de la respuesta. También se registran con más detalle el tiempo y el tipo de las conductas supersticiosas. Aunque se enfatiza el comportamiento en estado estable, se presentan algunos datos del desarrollo de la conducta supersticiosa. Se pretende mostrar que hace falta una revisión de la interpretación original de Skinner de la conducta supersticiosa.

{\scshape\bfseries Method}

Se utilizaron seis palomas en cajas operantes con las teclas cubiertas con cartón (excepto por la condición dependiente de respuesta).

Se utilizaron tres programas: tiempo fijo 12 segundos, tiempo variable 8 segundos, e intervalo fijo 12 segundos.




\end{document}
