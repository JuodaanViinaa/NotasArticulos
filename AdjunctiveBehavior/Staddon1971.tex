\documentclass[a4paper,12pt]{article}
\usepackage[utf8]{inputenc}
\usepackage[T1]{fontenc}
\usepackage[spanish]{babel}
\usepackage{csquotes}
\usepackage{anysize}
\usepackage{graphicx}
\marginsize{25mm}{25mm}{25mm}{25mm}

\title{The ``superstition'' experiment: a reexamination of its implications for the principles of adaptive behavior}
\author{J. E. R. Staddon \and Virginia L. Simmelhag}
\date{1971}

\begin{document}
{\scshape\bfseries \maketitle}

El experimento de superstición moldeó la visión teórica de Skinner de la conducta operante como el fortalecimiento de conducta emitida (impredecible) mediante la acción automática de los reforzadores. 

Las conductas inducidas plantean dudas sobre la visión de Skinner: estas conductas se desarrollan confiablemente  a pesar de que no son contiguas a la entrega de comida. Quizá algunas de las conductas que Skinner atribuyó al reforzamiento accidental de emisión espontánea tengan más bien factores causales comunes con estas ``actividades mediadoras''.

Por otro lado, la emergencia de conductas que parecen ser más Pavlovianas que operantes también plantea dudas en la concepción de Skinner: conductas ``instintivas'' parecen interferir con conductas operantes ya bien establecidas, como en el caso de Breland y Breland.

De modo similar, el procedimiento de automoldeamiento muestra una conducta que es mantenida incluso si su emisión previene la entrega de comida en una ocasión particular. Todos estos procedimientos muestran conductas relacionadas con la comida que ocurren en anticipación de la entrega de la misma y que son incompatibles con la ley del efecto.

La superstición de Skinner, como el automoldeamiento, es operacionalmente idéntica al condicionamiento Pavloviano. Quizá exposición prolongada a esa condición también tendrá como resultado la emergencia de conducta relacionada con la comida en anticipación a ésta. Quizá la superstición incluye actividades anticipatorias que ocurren antes de la comida y mediadoras que ocurren justo después.

Este experimento prueba estas ideas. Compara los efectos de intervalos fijos y variables en la conducta supersticiosa y compara programas de intervalo fijo dependientes e independientes de la respuesta. También se registran con más detalle el tiempo y el tipo de las conductas supersticiosas. Aunque se enfatiza el comportamiento en estado estable, se presentan algunos datos del desarrollo de la conducta supersticiosa. Se pretende mostrar que hace falta una revisión de la interpretación original de Skinner de la conducta supersticiosa.

{\scshape\bfseries Method}

Se utilizaron seis palomas en cajas operantes con las teclas cubiertas con cartón (excepto por la condición dependiente de respuesta).

Se utilizaron tres programas: tiempo fijo 12 segundos, tiempo variable 8 segundos, e intervalo fijo 12 segundos.

Las palomas fueron habituadas. Cuatro palomas corrieron en los dos procedimientos independientes de respuesta (fijo o variable). Dos de esas cuatro palomas después pasaron por el procedimiento dependiente de respuesta.

Las dos aves ingenuas, tras la habituación, pasaron por una sesión con el comedero disponible continuamente por diez minutos, y después directamente al procedimiento dependiente de respuesta. Esto tenía el propósito de evitar cualquier condicionamiento ``supersticioso''.

La descripción de las conductas se realizó mediante observación y presión de botones de manera discreta ({\itshape e.g.,} picoteo) o continua ({\itshape e.g.,} mirar hacia un lado de la caja).

{\scshape\bfseries Results}

Interesan las conductas y frecuencias de las conductas varios puntos tras la entrega de comida. Emergió un patrón sistemático de conducta post-comida:

La conducta en estado estable en ambos procedimientos independientes de respuesta cayó en dos categorías: la {\itshape respuesta terminal}, es decir, la conducta que ocurría consistentemente 6-8 segundos después de la entrega de comida en FI, y 2 segundos después en VI; y {\itshape actividades que precedían a la respuesta terminal en el intervalo}, probablemente indistinguibles de la ``conducta mediadora''. Se les etiquetará como ``actividades interinas'' ({\itshape interim activities}). Rara vez eran contiguas con la comida estas actividades.

Todas las aves muestran una clara división entre las actividades terminales y las interinas (según dicen los autores, aunque yo no lo veo tan claro).

Los datos no muestran evidencia de cambios sustanciales en el patrón de actividades terminales o interinas que sea atribuible a la imposición de un requisito de respuesta.

Acerca de la estructura secuencial de la conducta se encontró que (a) cada ave mostraba solo una pequeña cantidad de secuencias (tres o cuatro), (b) la secuencia era rígida, de modo que aunque algunas conductas a veces no ocurrían, nunca ocurrían fuera de su secuencia, y (c) la variabilidad de las secuencias era máxima al comienzo del intervalo y mínima al final. Evidencia muy marciana indica que el tiempo post-comida era el factor más importante que controlaba el inicio y el fin de cada actividad en la secuencia.




\end{document}
