\documentclass[a4paper,12pt]{article}
\usepackage[utf8]{inputenc}
\usepackage[T1]{fontenc}
\usepackage[spanish]{babel}
\usepackage{csquotes}
\usepackage{anysize}
\usepackage{graphicx}
\marginsize{25mm}{25mm}{25mm}{25mm}

\title{The interaction of psychogenic polydipsia with wheel running in rats}
\author{Evalyn F. Segal}
\date{1969}

\begin{document}
{\scshape\bfseries \maketitle}

Las ratas beben en exceso cuando se les entrega comida de forma intermitente. Una explicación es que beber solo ocurre para pasar el tiempo mientras se entrega el siguiente reforzador. A favor de esta hipótesis se ha encontrado que la introducción de una rueda de ejercicio induce carrera en un patrón similar al de la polidipsia. Este experimento pretende determinar si la carrera sustituye a la polidipsia o la complementa.

Se permitió a las ratas acceso a una rueda de ejercicio, un dispensador de agua, una palanca inactiva y comida entregada cada 90 segundos.

Las ratas tendían a beber durante aproximadamente 30 segundos después de una entrega de comida. Después de ello tendían a correr. Tanto las lamidas como las vueltas de la rueda incrementaron a niveles estables durante el experimento.

Se registró el porcentaje de vueltas y lamidas que ocurrieron en los primeros 30 segundos del intervalo de 90 segundos. Con el paso de las sesiones incrementó el número de lamidas que ocurrió al inicio del intervalo, y decrementó el número de vueltas para tres de cuatro ratas. Aunque parece que hay primacía de la bebida sobre la carrera los datos no indican si hay competencia, es decir, si la posibilidad de beber disminuyó la carrera o la posibilidad de correr disminuyó la bebida.

Una condición intermedia prohibió el acceso de las ratas a la rueda de ejercicio. Tras su reinstauración, las vueltas superaron a las lamidas temporalmente, pero el patrón previo se recuperó eventualmente, lo que indica que su preponderancia temporal se debió a su relativa novedad.

Las ratas pasaron por una sesión en la que solo se permitió la bebida y se restringió el acceso a la rueda. En ella el número de lamidas incrementó en gran medida, lo que indica un grado de competencia entre las dos conductas.

Al permitir exclusivamente la carrera, la distribución temporal de las vueltas se asemejó mucho a la de las lamidas: concentrada en el inicio del intervalo.

Parece ser que prevenir la bebida no incrementa necesariamente la carrera, pero prevenir la carrera sí incrementa la bebida. Eso no encaja con una explicación de ``algo que hacer''. Quizá ambas respuestas representan un {\itshape arousal} emocional generado por la frustración de esperar por la comida. La primacía de la bebida puede representar la interacción de la sed con el {\itshape arousal}.


\end{document}
