\documentclass[a4paper,12pt]{article}
\usepackage[utf8]{inputenc}
\usepackage[T1]{fontenc}
\usepackage[spanish]{babel}
\usepackage{csquotes}
\usepackage{anysize}
\usepackage{graphicx}
\marginsize{25mm}{25mm}{25mm}{25mm}

\title{The effects of schedule history and the opportunity for adjunctive responding on behavior during a fixed-interval schedule of reinforcement}
\author{Lisa M. Johnson \and Warren K. Bickel \and Stephen T. Higgins \and Edward K. Morris}
\date{1991}

\begin{document}
{\scshape\bfseries \maketitle}

Los patrones de respuestas son sensibles a la historia conductual además de a los programas presentes actualmente. Por ejemplo, entrenamiento con tiempos entre respuesta (IRT) mayores a cierto punto $t$ pueden llevar a tasas bajas de respuesta en programas FI posteriores.

Un posible mecanismo por el que la historia tiene estos efectos es la conducta inducida por el programa. Esta conducta se asocia con tasas bajas de respuesta en programas con IRT menor a $t$, y programas con IRT $> t$ tienen más probabilidades de producir conducta inducida por el programa que los programas de FR. Así, la conducta adjuntiva puede ser un mecanismo por el que la historia ejerce sus efectos en los patrones de respuesta futuros.

Este estudio investiga el papel de la historia y la disponibilidad de respuestas adjuntivas en la ejecución en programas FI en ratas. Se dio a las ratas una historia de respuestas en un programa con IRT $> 11s$ o en FR 40 con agua disponible durante la sesión. Después se impuso un programa FI 15s y se manipuló la disponibilidad de agua para determinar si la bebida inducida era el medio por el que la historia influye en la presión a las palancas. La polidipsia se eligió debido a que se genera fácil y confiablemente en ratas y es estable en el tiempo, además de que se sabe más de ella que de otras conductas.

{\scshape\bfseries Method}

Se usaron cuatro ratas. La caja experimental tenía una palanca, una charola de comida con una luz roja encima y un dispensador de agua. La ingesta de agua se determinó pesando el contenedor antes y después de las sesiones.

Dos sujetos pasaron inicialmente por un programa IRT $> 11s$ y dos más en FR 1. En los de FR el requisito de respuesta se incrementó gradualmente hasta 40. En todo momento hubo acceso libre a agua. Estas condiciones constituyeron las historias para el estudio y se mantuvieron hasta encontrar estabilidad.

{\itshape Manipulation I}. Tras alcanzar estabilidad los programas fueron cambiados a FI 15s. Al encontrar de nuevo estabilidad los sujetos recibieron en orden mixto 18 sesiones ``water in'' y ``water out'' mientras seguí en efecto el programa FI. En las sesiones water-in la botella estaba llena; en la water-out, vacía. Después de esta manipulación y antes de la siguiente los sujetos regresaron a FI 15s con la botella llena.

{\itshape Manipulation II}. Los sujetos on historia IRT $> t$ volvieron a IRT $> 11s$, y los de historia de FR volvieron a FR 40. Al volver a encontrar estabilidad se volvió a poner en efecto el programa FI 15s. Estas sesiones tenían 20 mL de agua disponible. Después de alcanzar estabilidad se cambió la cantidad de agua disponible por sesión a 0, 2.5, 5, 10 y 20 mL, de forma mixta. Cada nivel de agua se administró tres veces.

{\scshape\bfseries Results}

Los sujetos con la historia de IRT $> 11s$ desarrollaron polidipsia y permanecieron bebiendo cantidades grandes al pasar a FI 15s. Los sujetos con historia de FR 40 no desarrollaron polidipsia.

Las tasas de reforzamiento fueron aproximadamente iguales, aunque los sujetos con la historia de IRT $> 11$ tuvieron una tasa levemente inferior.

{\itshape Manipulation I}. Para los sujetos de historia IRT $> 11s$ las respuestas durante las sesiones water-in fueron similares a la línea base (4.6 y 5.3), pero en las sesiones water-out fueron mucho mayores (57.8 y 47.4). En contraste, para los sujetos de historia FR no hubo diferencias entre las dos condiciones: la tasa de respuestas fue alta y estable.

{\itshape Manipulation II}. La manipulación de la cantidad de agua no produjo ningún efecto en las respuestas del grupo de historia FR, pero sí lo hizo en el grupo de historia IRT $> 11s$: las tasas de respuesta estaban inversamente relacionadas con la cantidad de agua disponible.

{\scshape\bfseries Discussion}

Se demuestra que las historias de IRT $> 11s$ y de FR afectan la conducta en un programa de FI subsecuente. Tras la historia de IRT $> 11s$ la cantidad de agua disponible produjo grandes cambios en los patrones y tasas de respuesta en FI, pero lo mismo no ocurrió tras la historia de FR. Esto concuerda con hallazgos previos con animales humanos y no humanos, y permite señalar un mecanismo --la conducta inducida por el programa-- mediante el cual la historia influye en la conducta.

Los hallazgos concuerdan también con un reporte previo en el que se agregó un requisito de respuesta tandem para tasas altas y bajas de respuesta en FI bajo los mismos parámetros de este estudio. En ese caso se retrasó el desarrollo de polidipsia en una rata la cual mostró tasas de de respuesta en FI mayores que las ratas que ya habían desarrollado polidipsia. Pero cuando la polidipsia se desarrolló en esa rata, sus tasas de respuesta fueron comparables con las del resto de sujetos. Ambos estudios indican que la bebida inducida puede ser una forma en que la historia afecta a los sujetos con historia de IRT $> t$. Pero el hecho de que la disponibilidad de agua no afectara la ejecución de las ratas con historia FR indica que hay otros factores envueltos en la interacción de la historia con las respuestas subsecuentes.

Estos resultados muestran que la historia puede interactuar con variables que controlan la ejecución actual. En estos casos la historia puede afectar a la conducta de formas similares a las interacciones conducta-conducta observadas en otros arreglos experimentales. Investigación futura debería examinar la relación entre los efectos de la historia y las interacciones conducta-conducta.

Aquí los efectos de la historia fueron robustos y estables, en contraste con otros estudios en los que son transitorios. La diferencia puede estar en que (1) en este estudio los animales permanecieron en su condición histórica hasta encontrar estabilidad (58 a 102 sesiones), y (2) en que en otro estudio se administró {\itshape d-anfetamina}, que pudo haber modulado los efectos de las historias. Esto sugeriría un estudio sobre el efecto de las interacciones entre drogas e historia.

Este estudio indica que los determinantes de la conducta son múltiples e interdependientes. Las generalizaciones sobre los efectos de los programas pueden ser injustificadas si se ignora la historia. Y aunque la historia es importante, su influencia es modulada por las condiciones actuales. Y aun cuando las condiciones previas y actuales son constantes, la conducta puede modularse por otros aspectos del ambiente (como la disponibilidad de una respuesta adjuntiva).

La conducta es una función del programa actual, de la historia, y de otros factores adicionales.


\end{document}
