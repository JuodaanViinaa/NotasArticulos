\documentclass[a4paper,12pt]{article}
\usepackage[utf8]{inputenc}
\usepackage[T1]{fontenc}
\usepackage[spanish]{babel}
\usepackage{csquotes}
\usepackage{anysize}
\usepackage{graphicx}
\marginsize{25mm}{25mm}{25mm}{25mm}

\title{``Superstition'' in the pigeon}
\author{B. Skinner}
\date{1948}

\begin{document}
{\scshape\bfseries \maketitle}

Ocurre condicionamiento incluso si no lo pretendemos siempre que un {\itshape drive} se encuentra con reforzamiento.

Se describen ejemplos de conducta supersticiosa en el laboratorio. Palomas en un programa de tiempo fijo 15 segundos desarrollaron respuestas estereotipadas.

Las respuestas en general involucraban la orientación hacia un aspecto particular de la caja operante, lo que es indicativo de que las palomas respondían ante un aspecto del ambiente y no simplemente ejecutaban una serie de movimientos.

La conducta supersticiosa aparece con más probabilidad con intervalos entre reforzamiento breves debido a que intervalos más largos dan mayor oportunidad para variabilidad en la conducta. Además, en intervalos más largos hay una mayor cantidad de respuestas emitidas no reforzadas.

Cuando una respuesta supersticiosa fue llevada a la extinción, la reinstauración de la entrega no contingente de comida hizo resurgir a la conducta.

Los animales se comportan como si hubiese una relación causal entre su comportamiento y la presentación de comida, lo que es similar al comportamiento supersticioso humano.

Sin embargo, sí existe una cierta relación entre la entrega de reforzamiento y la conducta: si se entrega comida cada 15 segundos, se reforzarán las conductas que suelen ocurrir 15 segundos después de una entrega de comida. Del mismo modo en los humanos la conducta emitida y el posible reforzamiento tienen una determinante común ({\itshape e.g.,} un jugador de bolos que se mueve para manipular el camino de la bola: su conducta y el resultado final son determinados en el momento del lanzamiento).

La conducta parece aparecer por las relaciones accidentales entre la conducta y la presentación de un estímulo.


\end{document}
