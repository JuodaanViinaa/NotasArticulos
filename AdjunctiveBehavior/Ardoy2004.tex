\documentclass[a4paper,12pt]{article}
\usepackage[utf8]{inputenc}
\usepackage[T1]{fontenc}
\usepackage[spanish]{babel}
\usepackage{csquotes}
\usepackage{anysize}
\usepackage{graphicx}
\marginsize{25mm}{25mm}{25mm}{25mm}

\title{Effects of withholding the opportunity to press the operant lever on the maintenance of schedule induced drinking in rats}
\author{Juan Ardoy \and Ricardo Pellón}
\date{2004}

\begin{document}
{\scshape\bfseries \maketitle}

Se ha propuesto que la conducta adjuntiva ocurre cuando se interrumpe una actividad consumatoria en un animal motivado (como al permitir que un animal hambriento solo coma pequeñas cantidades intermitentemente). En esa situación los animales beberán en exceso si tienen la oportunidad. Pero la forma en que se desarrolla sugiere que es resultado de un proceso de aprendizaje. Una hipótesis inicial indica que es una respuesta condicionada pavloviana. Se ha sugerido que la polidipsia se restringe a períodos que involucran una baja probabilidad de reforzamiento y que su origen no se encuentra en la interrupción de la conducta consumatoria {\itshape per se}. Se ha enfatizado la influencia de factores asociativos en el desarrollo de la conducta adjuntiva, es decir, que las señales de reforzamiento y no-reforzamiento tienen un efecto en la polidipsia. Esta interpretación es compatible con un modelo en el cual las actividades interinas y terminales se localizan en períodos de baja y alta probabilidad de reforzamiento, respectivamente. 

En ciertos estudios en los que el agua no está disponible inmediatamente después de la comida se ha encontrado que los animales tienden a beber la misma cantidad durante las partes del intervalo entre ensayos en las que el agua sí está disponible, y la polidipsia también se ha logrado inducir si se pone un costo en respuestas a la obtención de agua.

Una interpretación distinta indica que la polidipsia podría ser operante: una conducta supersticiosa reforzada accidentalmente por la entrega de comida. Sin embargo, la bebida ocurre {\itshape después} del reforzador y también ocurre en programas de FR en los que es difícil establecer una relación enter la bebida y la presentación de comida. Aun así, evidencia señala similitudes funcionales entre la conducta adjuntiva y la operante con respecto a su susceptibilidad a las consecuencias ambientales: la bebida se puede castigar con una demora de 10s a la presentación de comida contingente a las lamidas, y se puede incrementar o disminuir la bebida adjuntiva si va seguida de presentación u omisión de comida.

El grado de polidipsia también es dependiente del nivel de privación de comida, de la magnitud del reforzador, y la naturaleza de la comida presentada, pero no depende del nivel de sed ni de la administración artificial de agua antes de la sesión. Así, parece depender en gran medida de las características motivacionales asociadas con el reforzador, pero esto no señala el mecanismo de adquisición, sino que solo indica similitudes con la conducta operante.

Este experimento busca explorar el papel en la polidipsia de los eventos que la siguen. Animales fueron expuestos a un programa FI 1 minuto y desarrollaron patrones de polidipsia y respuestas operantes a la palanca. La palanca fue después retirada y el programa cambió a FT 1 minuto. Si la conducta adjuntiva es controlada por la presentación previa del reforzador no se esperaría ningún cambio. Si es determinada por las acciones que siguen a la conducta, entonces deberán aparecer cambios. 

{\scshape\bfseries Method}

La caja operante tenía un dispensador de pellets rodeado a ambos lados por palancas retráctiles de las cuales solo se utilizó la izquierda. Se suministraba agua por una apertura en la pared derecha de la caja. El dispensador de agua tenía un sensor que registraba las lamidas.

En cada sesión se registró la duración de sesión, la cantidad de reforzadores obtenidos, de presiones de palanca, de lamidas, y de agua consumida. Esto permitió computar la tasa de presión de palanca y de lamidas.

Se condujeron 39 sesiones seguido de las cuales los sujetos fueron divididos en dos grupos. En la sesión 40, que aún era idéntica a las anteriores, se registró además la cantidad de presiones y de lamidas durante cada uno de los intervalos entre reforzadores.

En la última fase del experimento se retrajo la palanca de respuesta para el grupo experimental y se continuó entregando comida cada minuto. Para el grupo control la palanca permaneció presente pero no tenía consecuencias asociadas y se entregaba comida a la misma tasa. 

{\scshape\bfseries Results}

Ambos grupos desarrollaron tasas altas de presión y de lamidas, lo que muestra que el programa FI 60 es efectivo para mantener patrones de conducta operante y adjuntiva. No hubo diferencias entre los grupos en lamidas por minuto, pero la interacción grupo $\times$ sesión fue significativa, lo que indica que las ratas del grupo experimental adquirieron la bebida inducida antes que el grupo control, aunque los niveles de lamidas fueron idénticos.

{\scshape\bfseries Discussion}

Parece ser que la bebida adjuntiva fue temporalmente reducida cuando la presentación de comida cambió a FT 1 minuto y se evitó que los sujetos presionaran la palanca. Dado que en el grupo control no se presentó ese fenómeno el cambio se puede atribuir a la presentación de la palanca.

Parece indicarse que en animales con patrones estables de conducta adjuntiva y operante es suficiente retirar la oportunidad de responder para reducir transitoriamente la bebida adjuntiva. Estos resultados pueden tener importantes implicaciones para las teorías actuales de conducta adjuntiva.

Se ha sugerido que la presentación de comida actúa como un inhibidor condicionado al señalar un período en el que otra entrega de comida es altamente improbable. Así, la bebida adjuntiva estaría limitada a períodos de baja probabilidad de reforzamiento. La bebida sería una forma de escapar de estos períodos de no reforzamiento y puede reforzarse negativamente al mantener al animal lejos de su propia ejecución en el programa intermitente.

Los resultados presentes son más difíciles de explicar. En ambos grupos de animales la comida retuvo su valor como señal inhibidora y aun así se encontraron diferencias entre ellos, por lo que la explicación debe estar en lo ocurrido antes y no después de la conducta de beber.

Una alternativa tentativa es que la contingencia de reforzamiento puede afectar patrones amplios de conducta. La comida puede no solo reforzar la presión a la palanca, sino reorganizar el repertorio conductual del organismo. El animal puede aprender a emitir una secuencia de conductas que consiste en beber, presionar la palanca y finalmente aproximarse a la bandeja de comida. Un impedimento en el cumplimiento de cualquier eslabón afectaría a las conductas previas transitoriamente hasta que el patrón entero entero se reorganice. Quizá por esto hubo cambios cuando se impidió presionar a la palanca.

Las ratas beben después de la comida, pero la bebida puede estar mantenida en parte por la presentación del siguiente reforzador.

Los patrones característicos que generan los programas de reforzamiento no son solo resultado de los programas {\itshape per se}, sino que dependen también de la historia conductual. La bebida inducida por el programa puede ser también una función de la historia conductual y puede depender críticamente de si la conducta adjuntiva cabe en ella.

Este análisis asume que la conducta puede inducirse y mantenerse en ausencia de reforzamiento explícito. Esta aproximación enfatiza que la conducta natural de los organismos puede modificarse por experiencia, y sugiere una forma en que esto puede ocurrir en el caso de la conducta adjuntiva.


\end{document}
