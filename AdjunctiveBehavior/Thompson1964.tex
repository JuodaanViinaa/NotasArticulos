\documentclass[a4paper,12pt]{article}
\usepackage[utf8]{inputenc}
\usepackage[T1]{fontenc}
\usepackage[spanish]{babel}
\usepackage{csquotes}
\usepackage{anysize}
\usepackage{graphicx}
\marginsize{25mm}{25mm}{25mm}{25mm}

\title{Escape from SD associated with fixed ratio reinforcement}
\author{Donald M. Thompson}
\date{1964}

\begin{document}
{\scshape\bfseries \maketitle}

Aparentemente los programas con razones fijas muy elevadas son aversivos. Por ejemplo, Azrin logró que palomas se mantuvieran tiempos progresivamente más elevados alejadas de un programa de razón fija cuanto más alta era la razón.

El experimento tenía la forma siguiente: tras establecer la luz apagada como $S^{\delta}$ mediante entrenamiento preliminar se pasó a una condición en la cual una palanca estaba asociada con un programa de razón creciente. La otra palanca era de {\itshape timeout} y presionarla iniciaba un tiempo fuera de 30 segundos. Después de ello se pasó a un programa de razón descendente. Los saltos eran de 25 en 25 respuestas como requisito.

Se utilizó la razón entre número de escapes y número de escapes más número de reforzadores producidos como una medida de la aversión. Ese índice representa la proporción de reforzadores negativos ({\itshape i.e.,} escapes) del total de reforzadores obtenidos. Un valor más allá de 0.5 indica que al menos hubo un escape por cada reforzador adquirido.

La aversión al programa incrementó en función del requisito de respuesta como resultado de un incremento en el número de escapes y una disminución en el número de reforzadores positivos adquiridos.

Los datos fueron muy similares en los programas ascendentes y descendentes, de modo que sus datos fueron agregados.

Se encontró por observación casual que algunos sujetos pasaban 5 segundos bebiendo agua después de la entrega de reforzamiento sin importar el requisito del programa.

Los resultados indican que aunque los programas de FR son positivamente reforzantes, poseen a la vez propiedades aversivas que refuerzan la conducta que remueve el control de estímulos de FR. La cantidad de tiempo pasada en {\itshape timeout} incrementa junto con el requisito de respuesta de FR.

Se concluye que el procedimiento de {\itshape timeout} autoimpuesto es una buena medida de las propiedades aversivas concominantes de programas de reforzamiento positivo. Una implicación importante es que cualquier estímulo asociado con reforzamiento positivo puede poseer a la vez propiedades aversivas además de sus propiedades discriminativas.


\end{document}
