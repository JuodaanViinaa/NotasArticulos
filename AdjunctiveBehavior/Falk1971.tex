\documentclass[a4paper,12pt]{article}
\usepackage[utf8]{inputenc}
\usepackage[T1]{fontenc}
\usepackage[spanish]{babel}
\usepackage{csquotes}
\usepackage{anysize}
\usepackage{graphicx}
\marginsize{25mm}{25mm}{25mm}{25mm}

\title{The nature and Determinants of Adjunctive Behavior}
\author{John L. Falk}
\date{1971}

\begin{document}
{\scshape\bfseries \maketitle}

La polidipsia ha sido observada en ratas, monos rhesus, chimpancés, y palomas.

Es desconcertante dado que la privación de comida suele llevar a un decremento en la ingesta de comida, no a un incremento. Además es desadaptativo para un animal privado de comida gastar recursos en calentar agua y además llevarse al borde de la intoxicación por agua.

La polidipsia, dice una publicación previa, no es resultado del reforzamiento circunstancial de la bebida mediante el acceso a comida. Tampoco sirve a la estimación temporal ni es una respuesta incondicionada. A diferencia de las conductas reflejas la polidipsia tarda algunas sesiones en desarrollarse. En resumen: no hay una explicación aceptada para la polidipsia.

Cuando una manipulación simple produce control conductual poderoso se puede presumir que puede ser ampliamente generalizada.

Correspondencias entre la conducta adjuntiva y las actividades de desplazamiento ({\itshape displacement activities}) llevan a suponer que se trata de instancias de un fenómeno superior y posiblemente un nuevo tipo de conducta. Aunque debe tenerse cuidado al sugerir ``nuevo s tipos de conducta'' de que se trate de divisiones útiles que sugieran nuevas líneas de investigación.

Las conductas de agresión y de escape ocurren como adjuntas condiciones aversivas, como la presentación de choques eléctricos o de programas de extinción. El escape, al igual que la agresión, también se puede inducir mediante programas específicos. Parece ser que el origen común de estas conductas está en las propiedades aversivas de los parámetros de entrega de comida.

La conducta de {\itshape pica} también puede ser inducida (en macacos rhesus), al igual que las lamidas al aire en ratas.

Sin embargo, aunque estas conductas parecen surgir en condiciones similares, es muy distinto argumentar que forman una clase completa de conductas.

Parece ser que la variable determinante para la polidipsia no es el tiempo entre episodios de comida {\itshape per se}, sino la tasa de conducta consumatoria determinada por este parámetro temporal.

Parece ser que la polidipsia es indirectamente dependiente del tiempo entre entregas de comida. Este tiempo tiene un cierto {\itshape rango efectivo} en el cual la efectividad es máxima. Si las conductas adjuntivas son una clase completa de conducta se esperaría que otras conductas presenten funciones y rangos efectivos similares. Funciones similares han sido encontradas para la agresión inducida. Sin embargo los datos de escape inducido no parecen apoyar la misma noción hasta ahora, aunque tampoco se han hecho experimentos que busquen explorar específicamente el rango efectivo.

Se esperaría que cambiar el nivel de privación, lo que modificaría la ``fuerza'' de las respuestas de alimentación, module la polidipsia. Se ha encontrado que incrementar el peso de los animales tiene disminuye la cantidad de agua consumida a pesar de que las respuestas de presión en la palanca se mantengan, es decir, el mantenimiento de la ingesta y las respuestas no es suficiente para mantener la polidipsia. Si hay una clase separada de conductas adjuntivas se esperaría que manipulaciones comparables en el peso tuviesen efectos similares. Un efecto así se ha encontrado en las lamidas al aire inducidas, y en palomas saciadas se ha encontrado disminución del ataque inducido.

Una característica común para todas las conductas inducidas (bebida, ataque, escape, pica) es que ocurren de manera excesiva.

La mayor parte de las conductas adjuntivas ocurren justo después de la entrega de comida salvo por algunos casos de ataque y escape, que ocurren antes del inicio de razones fijas grandes. Sin embargo hay algunos casos particulares: programas de intervalo fijo de 3 minutos aproximadamente resultan en polidipsia distribuida en ráfagas breves en lugar de una sola ráfaga post reforzamiento. Un caso particular mostró cómo la conducta de polidipsia, al ser evocada por un programa de FI 4 min, pasó a ser controlada por una contingencia operante y adoptó un patrón de festón.

La conducta inducida puede además funcionar como reforzador para conducta operante (polidipsia, ataque inducido).

La conducta inducida parece ser resultado de las oportunidades del ambiente y hay cierto grado de sustitución entre ellas, por ejemplo, la provisión de una rueda de ejercicio concurrente con agua lleva a conducta excesiva en ambos operandos.

Algunas explicaciones propuestas son la decepción, frustración o confusión de los animales en un programa intermitente. Sin embargo esas explicaciones no se sostienen dado que es difícil argumentar que tras meses en una situación experimental lo animales continúan inocentemente decepcionados y se consuelan bebiendo. La conducta inducida es el resultado del control de los programas, no un estado de transición concerniente a ese control.

La conducta adjuntiva no es elicitada como una conducta respondiente novedosa, sino que probablemente ya es parte del repertorio del animal y las condiciones la exacerban.

Parece ser que lo que determina la ocurrencia de la conducta inducida no es la recencia del último pellet, sino la baja probabilidad de reforzamiento que correlaciona con ella.

Las actividades de desplazamiento son acciones que ocurren fuera del contexto en que serían más adaptativas, por ejemplo, acicalamiento que ocurre interrumpiendo la conducta de agresión. Se puede importar la definición hacia las conductas adjuntivas: respuestas que son usualmente función de variables distintas de aquellas que dominan la situación presente. Se ha descrito que las actividades de desplazamiento ocurren cuando un animal involucrado en una actividad consumatoria es prevenido de continuar con ella. Esto sería congruente con los resultados de conducta inducida: se previene la alimentación continua de un animal mediante un programa de reforzamiento.

La posibilidad de que una conducta adjuntiva sustituya a otra es un hallazgo empírico, pero no una necesidad teórica. Es decir, colocar a un organismo en un ambiente distinto no garantiza que, impidiéndole concretar una conducta consumatoria, emergerá una conducta adjuntiva distinta.

Por otro lado, hay explicaciones que buscan darle una función adaptativa a la conducta adjuntiva. Por ejemplo, como reguladora de la temperatura o del contenido de agua en el estómago. Habiendo fallado eso, se ha probado también lo contrario: sugerir que se trata de reacciones maladaptativas ante la frustración. Según esta visión, toda la conducta es adaptativa o maladaptativa, y la conducta adjuntiva no parece ser adaptativa.

La conducta adjuntiva es conducta mantenida con alta probabilidad por estímulos cuyas propiedades reforzantes en la situación se derivan principalmente como función de parámetros del programa que gobiernan la disponibilidad de otra clase de reforzadores.




\end{document}
