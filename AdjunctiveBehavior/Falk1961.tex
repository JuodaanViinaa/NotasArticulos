\documentclass[a4paper,12pt]{article}
\usepackage[utf8]{inputenc}
\usepackage[T1]{fontenc}
\usepackage[spanish]{babel}
\usepackage{csquotes}
\usepackage{anysize}
\usepackage{graphicx}
\marginsize{25mm}{25mm}{25mm}{25mm}

\title{Production of polytdipsia in normal rats by an intermittent food schedule}
\author{J. Falk}
\date{1961}

\begin{document}
{\scshape\bfseries \maketitle}

Ratas en un programa de intervalo variable 1 minuto comenzaron a beber agua a una tasa exagerada con un patrón identificable: tras ganar un pellet aparece un {\itshape burst} rápido de lamidas seguido de un regreso a presiones a la palanca. Las ratas beben tanto que no obtienen tantos pellets como podrían, y lo hacen a pesar de no estar privadas de agua.

Las ratas beben durante la sesión aproximadamente 3.43 veces más de lo que bebían en línea base durante 24 horas, lo que es más impresionante si se considera que usualmente las ratas privadas de comida beben menos.

El efecto se desarrolla en una o dos sesiones en todos los animales.

Métodos usuales para inducir bebida incluyen introducir el agua artificialmente al estómago o hacer las lamidas contingentes a la evitación de un choque. Con este arreglo se produce polidipsia sin los métodos traumáticos usuales.


\end{document}
