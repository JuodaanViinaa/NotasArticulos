\documentclass[a4paper,12pt]{article}
\usepackage[utf8]{inputenc}
\usepackage[T1]{fontenc}
\usepackage[spanish]{babel}
\usepackage{csquotes}
\usepackage{anysize}
\usepackage{graphicx}
\marginsize{25mm}{25mm}{25mm}{25mm}

\title{Schedule-induced behavior}
\author{Ricardo Pellón \and Gabriela E. López-Tolsa \and Valeria E. Gutiérrez-Ferre \and Esmeranda Fuentes-Verdugo}
\date{2020}

\begin{document}
{\scshape\bfseries \maketitle}

La conducta adjuntiva ocurre en el intervalo entre reforzamiento en ausencia de contingencias de reforzamiento explícitas.

Las conductas inducias tienden a organizarse en el tiempo de manera secuencial si es que hay más de una disponible a la vez.

La polidipsia es afectada por parámetros de la entrega de comida (magnitud, densidad, calidad), pero no por los parámetros de el agua misma ni por la sed. Además, no se observa polidipsia si el reforzador es alimento húmedo, y el uso de agua como reforzador no induce comida.

Se han propuesto al menos dos mecanismos para las conductas inducidas por el programa: reforzamiento e inducción.

{\scshape Reforzamiento} implica que clases distintas de conductas tienen distintas ventanas temporales en las cuales pueden ser reforzadas. {\scshape Inducción} implica que ciertos eventos, conocidos como Eventos Filogenéticamente Importantes, incrementa la ocurrencia de la conducta o el tiempo pasado involucrándose en ciertas conductas. Cuando estos eventos ocurren consistentemente cerca de alguna conducta sobreviene un incremento en la conducta cuando el evento ocurre.

También existen implicaciones neuronales: se han implicado a las vías dopaminérgicas mesolímbicas (enfocadas en el incentivo) y al eje hipotálamo-pituitario-adrenal (que propondría a las conductas adjuntivas como respuestas ante situaciones estresantes). Además, se ha propuesto a las conductas inducidas como un modelo de compulsividad y se ha sugerido que pueden tener características en común con la conducta compulsiva en humanos.


\end{document}
