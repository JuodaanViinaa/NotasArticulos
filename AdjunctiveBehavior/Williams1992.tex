\documentclass[a4paper,12pt]{article}
\usepackage[utf8]{inputenc}
\usepackage[T1]{fontenc}
\usepackage[spanish]{babel}
\usepackage{csquotes}
\usepackage{anysize}
\usepackage{graphicx}
\marginsize{25mm}{25mm}{25mm}{25mm}

\title{Prior exposure to a running wheel and scheduled food attenuates polydipsia aquisition}
\author{Shelly L. Williams \and Maisy Tang \and John L. Falk}
\date{1992}

\begin{document}
{\scshape\bfseries \maketitle}

La polidipsia ocurre en programas intermitentes en rangos particulares. La limitación de la comida o el refuerzo accidental de beber mediante la entrega de comida no parecen ser motivos de la sobre-bebida.

Se ha encontrado que la exposición a un programa generador sin agua disponible retrasa o evita la emergencia de polidipsia.

Este estudio pretende determinar si la provisión de una rueda de ejercicio junto con el programa FI 1 minuto sería una historia aun más efectiva para interferir con el desarrollo posterior de polidipsia.

{\scshape\bfseries Method}

Se utilizó una cámara operante con una palanca, un receptáculo de pellets, acceso opcional a agua y acceso opcional a una rueda de ejercicio en el panel contrario al agua.

El programa de entrega de pellets era de IF 1 minuto con 120 recompensas por sesión. Hubo dos grupos: uno de ellos tuvo 10 sesiones de entrenamiento en FI 1 minuto seguidas de 140 sesiones con acceso a una rueda de ejercicio a la par del programa, seguidas de 2-3 meses de sesiones idénticas más acceso a agua, seguidas de un mes de sesiones idénticas sustituyendo el agua con etanol al 5\%. En las últimas 10 sesiones tanto de agua como de alcohol se introdujo una sesión de sondeen la que el programa pasó a FR 1. Estas sesiones permitían comparar la bebida normal con la bebida inducida por el programa. El segundo grupo corrió de forma idéntica salvo porque al introducir el agua la rueda de ejercicio fue removida permanentemente.

{\scshape\bfseries Results}

Se compararon los resultados con los de Tang (1988). El grupo que permaneció simplemente en su caja habitación mostró el patrón usual de adquisición de polidipsia; el de exposición a FI 1 minuto sin acceso a agua mostró un retraso en la adquisición y un nivel final menor de polidipsia; ambos grupos con rueda mostraron interferencia con la adquisición de polidipsia mayor que la del grupo de {\itshape schedule history}. No hubo diferencias entre ambos grupos con rueda.

La ingesta de alcohol fue idéntica en todos los grupos salvo en el que la rueda permaneció. En él, la ingesta fue menor.

La comparación con las sesiones de sondeo mostró que a pesar del retraso en la adquisición sí ocurrió polidipsia.

{\scshape\bfseries Discussion}

La exposición prolongada a condiciones que usualmente inducen polidipsia pero en ausencia de la oportunidad de beber resulta en la atenuación de la tasa de desarrollo y nivel final de polidipsia una vez que el agua se hace disponible. La atenuación es aun más pronunciada cuando la condición histórica incluye acceso a una rueda de ejercicio, y hubo poca diferencia entre los grupos de ruedas. Estudios previos han reportado poca interferencia con la polidipsia por acceso concurrente a una rueda de ejercicio.

Aunque hubo una atenuación en la polidipsia al incluir una rueda, el consumo de agua y de etanol permaneció en niveles polidípsicos.


\end{document}
