\documentclass[a4paper,12pt]{article}
\usepackage[utf8]{inputenc}
\usepackage[T1]{fontenc}
\usepackage[spanish]{babel}
\usepackage{csquotes}
\usepackage{anysize}
\usepackage{graphicx}
\marginsize{25mm}{25mm}{25mm}{25mm}

\title{An inverse relationship between baseline fixed-interval response rate and the effects of a tandem response requirement}
\author{Warren K. Bickel \and Stephen T. Higgins \and Kimberly Kirby \and Lisa M. Johnson}
\date{1988}

\begin{document}
{\scshape\bfseries \maketitle}

Las condiciones presentes al momento del refuerzo son importantes determinantes de la tasa de respuesta. Por ejemplo, programas de FI refuerzan tiempos entre respuesta (IRT) largos. La contribución del reforzamiento preferencial de IRT largos a la tasa de respuesta puede determinarse cambiando los requisitos en el momento del reforzamiento, por ejemplo, mediante la imposición de un requisito de respuesta tandem de razón fija (FR), dado que los FR refuerzan IRT cortos. Si las tasas de respuesta incrementan con el requisito de respuesta tandem, eso indica que las condiciones que operan al momento de la respuesta contribuyen a la tasa de respuesta con ese programa. Sin embargo, cuando se han agregado programas tandem FR a programas FI se han encontrado resultados inconsistentes (incrementos grandes o solo modestos en las tasas de respuesta, decrementos o ningún cambio).

Las variables que llevan a estos resultados inconsistentes no se han determinado. Los estudios previos han usado FR 9 a 12, y diferentes valores de FI, lo que indica que en el FI puede estar la causa. Otro posible culpable puede ser la tasa de respuestas de línea base en FI antes de la adición del requisito de FR. Es posible que aquellos animales con tasas bajas en línea base tuvieran un incremento significativo al incluir el programa tandem, y aquellas con tasas altas de inicio no tuvieran ningún incremento, tuvieran un incremento modesto, o un decremento.

Para investigar si el efecto de agregar un requisito de FR tandem a un programa FI es una función de la tasa base de respuesta se necesitan producir tasas altas y bajas con un mismo FI. Para ello, en este experimento se dará a los sujetos una historia conductual de FR o IRT $> ts$, que han mostrado producir tasas relativamente altas y bajas de ejecución den FI.

En este estudio se varió la historia para investigar la influencia de la tasa de respuesta en línea base sobre el efecto de agregar un requisito FR tandem.

{\scshape\bfseries Method}

Se usaron seis ratas. Las cámaras experimentales tenían una palanca, dispensador, charola de comida. La ingesta de agua se registró midiendo el volumen de agua en el dispensador antes y después de la sesión.

Los sujetos fueron entrenado inicialmente en FR 1. La mitad de ellos después incrementó progresivamente a FR 40 y la otra mitad pasó a IRT $> 11s$. FR 40 e IRT $> 11s$ constituyeron las historias conductuales.

Los sujetos tenían acceso completo a agua en todas las sesiones. Tras llegar a un criterio de estabilidad ambos programas fueron cambiados a FI 15s. Una vez llegados a estabilidad se agregó un requisito tandem FR de modo que al completar la décima respuesta después de que el FI se cumpliera se entregaría un pellet ({\itshape i.e.,} FI 15s FR 9). Después de llegar a estabilidad se realizó una replicación intra sujetos. En ella, los sujetos regresaron a FI 15s hasta alcanzar estabilidad y entonces volvieron al programa tandem FI-FR.

{\scshape\bfseries Results}

Las ratas que provenían de historias FR 40 tenían tasas altas de respuesta al llegar a FI, y al introducir el requisito tandem hubo cambios pequeños en sus tasas de respuesta (-4.1\% y +9\%). Su segunda vuelta produjo resultados similares. Su patrón de respuestas general fue de respuestas constantes con pocas pausas.

Las ratas de historia IRT $> 11s$ tenían tasas relativamente bajas de respuesta al pasar a FI. La imposición del programa tandem resultó en incrementos relativamente grandes en las tasas (de 1.4 a 3.5 veces). En la replicación el incremento fue incluso mayor. El patrón general de respuestas fue de pausas largas seguidas de un número pequeño de respuestas.

Una rata del grupo IRT $> 11s$ tuvo resultados distintos: sus tasas de respuesta bajaron en 2.6\%. Pero en su segunda exposición al tandem mostró un incremento similar al de las demás ratas.

Con el programa tandem bajó la tasa de reforzamiento e incrementó el consumo de agua con relación a la condición FI para las ratas con historia IRT $> 11s$ pero no para las ratas con historia FR.

{\itshape Rate dependency}

Para determinar si los efectos del requisito tandem se relacionaban inversamente con las tasas de respuesta en la condición FI se graficó la proporción de cambio en la tasa de respuesta de la línea base FI al tandem FI-FR en función de la tasa de respuesta de la línea base. Se encontró que según incrementaba la tasa de respuesta en FI el incremento en la tasa de respuesta en el programa tandem era progresivamente menor.

{\scshape\bfseries Discussion}

El experimento pretendía determinar si los efectos de añadir un requisito de respuesta tandem dependían de la tasa de respuesta en la línea base de FI. Eso se logró con distintas historias conductuales. Se encontró que las magnitudes de los cambios en las tasas de respuesta estaban inversamente relacionados a la tasa de respuesta de la línea base.

Pero las tasas de respuesta de la condición tandem no eran necesariamente dependientes de la historia. Por ejemplo, un sujeto tuvo inicialmente una tasa de respuesta en la condición FI más alta que la de otros sujetos en su misma historia (IRT $> 11s$), y en su primera exposición al tandem hubo un cambio muy pequeño. Esto sugiere que la historia no fue el único determinante de la relación inversa y que este efecto puede depender más de las tasas de respuestas en sí mismas y no de los efectos que determinan esas tasas.

En estudios previos además de la tasa de respuesta se variaba la tasa de reforzamiento. En este, la última se mantuvo relativamente constante, de modo que no se le pueden atribuir las discrepancias. 

La relación inversa entre la tasa base de respuesta en FI y los efectos de añadir un programa tandem son similares al fenómeno en farmacología de la dependencia de tasa ({\itshape rate dependency}). Dependencia de tasa es el fenómeno en el que los efectos de una droga sobre la tasa de respuesta están inversamente relacionados con las tasas de respuesta pre-droga. Esto sugiere que el principio de la dependencia de tasa puede ser importante ene el análisis de los efectos de variables no farmacológicas en la conducta controlada por programas.

No se pudieron separar las contribuciones de la tasa de respuesta y del consumo de agua. Es posible que los resultados se deban a los factores que resultaron en las tasas altas y bajas (el consumo de agua) y no a las tasas en sí mismas. La polidipsia puede ser un mecanismo mediante el cual la historia ejerce sus efectos en la conducta operante.

Se mostró que cambiar las condiciones que operan en el momento del reforzamiento mediante un requisito de respuesta tandem puede influir en la ejecución en FI cuando inicialmente se muestran tasas bajas de respuesta. Se encontró también evidencia que indica que otras conductas de los sujetos puede contribuir a las tasas de respuesta obtenidas en programas FI. Por último, se encontró que las tasas de respuesta en programas FI interactuaron con el requisito tandem de modo tal que la ejecución de tasa baja incrementó marcadamente, mientras que la de tasa alta permaneció relativamente sin efectos.

Las variables que operan aquí también lo hacen cuando no se estudian explícitamente, de modo que podrían tener implicaciones en estudios previos.


\end{document}
