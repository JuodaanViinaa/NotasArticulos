\documentclass[a4paper,12pt]{article}
\usepackage[utf8]{inputenc}
\usepackage[T1]{fontenc}
\usepackage[spanish]{babel}
\usepackage{csquotes}
\usepackage{anysize}
\usepackage{graphicx}
\marginsize{25mm}{25mm}{25mm}{25mm}

\title{Selection by consequences}
\author{B. Skinner}
\date{1981}

\begin{document}
{\scshape\bfseries \maketitle}

Todo en los organismos vivos es seleccionado por sus consecuencias. La reproducción es la primera y principal consecuencia. El condicionamiento operante es una segunda consecuencia. Requiere de susceptibilidad de los organismos al reforzamiento y la existencia de conductas inespecíficas.

La selección natural y el condicionamiento operante pueden trabajar en sinergia redundante, pero el condicionamiento es capaz de sustituir a la selección natural: si el único mecanismo es la selección natural, el acto de comer o la conducta sexual no necesitan ser reforzantes, sino solo incrementar el éxito reproductivo. Pero el condicionamiento permite la aparición de nuevos tipos de conductas que solo llevan eventualmente a alimentación, o de comportamiento sexual que no necesariamente termina en reproducción. Estas conductas no necesariamente serán adaptativas.

De ese modo probablemente la conducta social humana se hizo más robusta en el momento en que las vocalizaciones pasaron a estar bajo control operante y fuimos capaces de coordinar mejor y aprender de las experiencias de otros.

Un tercer tipo de selección es la selección cultural. Esta depende de los efectos que el comportamiento individual tenga sobre el grupo y no de las consecuencias individuales mismas.

Así, la conducta humana es producto de:
\begin{itemize}
        \item Contingencias de supervivencia responsables de la selección natural.
        \item Contingencias de reforzamiento responsables de los repertorios conductuales de los miembros de la especie.
        \item Contingencias especiales mantenidas por un entorno social.
\end{itemize}

De los tres niveles solo el segundo ocurre en una escala temporal que permita análisis momento a momento. En los tres niveles la ausencia de cambios en tiempos largos es explicable mediante la ausencia de variaciones seleccionables. Solo el segundo nivel, también, tiene el problema de definir qué implica la transmisión de aquello que se selecciona. En los otros niveles implica transmitirlo a individuos de generaciones posteriores, pero en el segundo implica solo su conservación en el individuo en cuestión.

Las explicaciones de selección por consecuencias son aun controversiales porque reemplazan a otras explicaciones muy queridas: la selección natural reemplaza a un creador, el condicionamiento reemplaza a la noción de voluntad, la selección social reemplaza al origen de la cultura como un contrato social.

La selección por consecuencias funciona con base en las consecuencias pasadas, es decir, un animal no tiene ojos para poder ver, sino porque animales pasados que resultaban tener ojos podían ver mejor.

La selección por consecuencias no necesita de ``esencias'' especiales para dar explicación de los fenómenos. Ni vida, ni mente, ni {\itshape zeitgeist}. Solamente consecuencias.

La selección por consecuencias, la ``replicación con error'' que la hace posible, solo son posibles en entidades vivas o en cosas fabricadas por entidades vivas. Así, la selección por consecuencias es lo que define a la vida.

Como alternativa a la selección por consecuencias se utiliza la metáfora del ``almacenamiento'', según la cual los entes almacenan la información que les será útil, y la recuperan en el futuro. Como ejemplo están los genes y la memoria. Quizá un uso válido esté en el tercer nivel, en el cual hay un almacén en forma de documentos y objetos permanentes.

Otra alternativa propuesta es la organización. Si bien, los entes biológicos y sociales son altamente organizados, no hay un principio de organización que justifique que así lo sean.

Una más es la idea del desarrollo. Aunque Skinner la critica sin mucho sustento.

A la estructura también se le suele achacar la fuerza causal detrás de la existencia de entidades. Cuando esto sucede se descuida a la selección.

Resistencia especial contra la selección por consecuencias viene de la ausencia de un agente iniciador: en lugar de que un agente se adapte a un entorno, el entorno moldea el comportamiento del agente. Esto pone en riesgo a la propia humanidad: enfrentándose a nuevas presiones, ¿debemos esperar que la selección tome su curso, o podemos tomar medidas y trascenderla? La respuesta es que aun si intentamos trascenderla no podemos escapar de la selección. Podemos actuar, pero aun así debemos esperar a que ocurran consecuencias sobre nuestras actividades. Nos vemos a nosotros mismos como agentes iniciadores solo por ignorancia de nuestra historia genética y ambiental.


\end{document}
