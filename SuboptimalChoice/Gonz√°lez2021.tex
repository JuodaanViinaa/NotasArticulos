\documentclass[a4paper,12pt]{article}
\usepackage[utf8]{inputenc}
\usepackage[T1]{fontenc}
\usepackage[spanish]{babel}
\spanishdecimal{.}
\usepackage{csquotes}
\usepackage{anysize}
\usepackage{graphicx}
\marginsize{25mm}{25mm}{25mm}{25mm}

\title{The role of inhibition in the suboptimal choice task}
\author{Valeria V. González \and Aaron P. Blaisdell}
\date{2021}

\begin{document}
{\scshape\bfseries \maketitle}

En elección subóptima las palomas prefieren la alternativa con señales predictivas a pesar de tener una menor densidad de reforzamiento.
El procedimiento tiene su base en el paradigma de respuestas de observación, en el cual los animales están dispuestos a responder para obtener estímulos que informan sobre el programa vigente, a pesar de no poder cambiarlo.

Una interpretación viene de la teoría de información, según la cual la información positiva y negativa deberían ser igualmente valiosas.
La información positiva es directamente valiosa, pero la negativa lo es porque permite redirigir los recursos a otro lugar (Vasconcelos).
Pero experimentos más recientes han mostrado que las buenas y malas noticias no son iguales, en tanto que los animales prefieren las buenas que las malas noticias.
Pero los experimentos no indican si los animales aprenden que las malas noticias son señales de ausencia de reforzamiento, o si aprenden a ignorar la señal.
Evidencia indica que los humanos prefieren las malas noticias que la ausencia de noticias, y las ratas prefieren un choque inescapable señalado que uno no señalado.

Se ha investigado poco la {\itshape preferencia} por la información separada de las respuestas de observación.
Un procedimiento en el que se hizo usó un laberinto en E en el que ambos brazos proveían la misma probabilidad de comida, pero uno informaba anticipadamente mediante el color de las paredes y otro no.
Se encontró que las ratas prefieren la información anticipada a pesar de que esta no cambie el resultado final.
Más tarde, el grupo de Zentall encontró que esta preferencia por la información sucede aun si implica menor acceso a comida.

Varias explicaciones presumen que el $S^{-}$ es ignorado y no tiene efecto en la elección ({\itshape e.g.,} el reinforcement rate model), lo que contrasta con la evidencia que indica que los animales prefieren un mal resultado señalado que uno sin señalar.
Del mismo modo, la explicación de la información temporal de Cunningham y Shahan supone que el $S^{-}$, al no señalar ninguna relación temporal con un reforzador, es ignorado.
Por otro lado, la hipótesis $\Delta-\Sigma$ supone que los animales deben prestar atención a todas las probabilidades, incluida la de $S^{-}$.
En resumen, no es si $S^{-}$ afecta la elección; y si lo hace, no es claro si se debe a un proceso perceptual, atencional o de aprendizaje.

Se propone que el $S^{-}$ sí contribuye a la elección subóptima al volverse un inhibidor condicionado.
Laude, utilizando el procedimiento de elección subóptima de magnitudes, encontró que se desarrolla inhibición condicionada, pero esta se pierde con entrenamiento extendido.
No se provee una explicación para este fenómeno, pero una posibilidad es que el $S^{-}$ inicialmente redujera las respuestas debido a inhibición externa, la cual menguó mientras el $S^{-}$ se hizo más familiar.

El uso del procedimiento de magnitudes implica que todas las teclas son informativas.
Es incierto si el mismo mecanismo causa la elección subóptima en ambos procedimientos.
Por otro lado, $S^{-}$ era una línea vertical, mientras que los demás estímulos eran colores.
Se ha mostrado que los colores son más salientes que las formas, por lo que usar al estímulo menos saliente como $S^{-}$ puede haber contribuido a que este estímulo perdiera control sobre la conducta.
Además, solo se utilizó una prueba de $S^{+}S^{-}$ para determinar la inhibición.
Se requieren de más pruebas (por ejemplo, con estímulos novedosos) para descartar la inhibición externa.

Fortes incrementó la probabilidad y longitud del $S^{-}$, manipulaciones que deberían incrementar la inhibición condicionada, pero no hubo cambios en la preferencia.
Trujano, por otro lado, encontró que la inhibición en ratas incrementa con el entrenamiento, pero no encontró elección subóptima y concluyó que la diferencia en resultados está relacionada con una diferencia en el impacto de los inhibidores condicionados.


\end{document}
