\documentclass[a4paper,12pt]{article}
\usepackage[utf8]{inputenc}
\usepackage[T1]{fontenc}
\usepackage[spanish]{babel}
\spanishdecimal{.}
\usepackage{csquotes}
\usepackage{anysize}
\usepackage{graphicx}
\marginsize{25mm}{25mm}{25mm}{25mm}

\title{Rats' performance in a suboptimal choice procedure implemented in a natural foraging analogue}
\author{Fernanda González-Barriga \and Vladimir Orduña}
\date{2024}

\begin{document}
{\scshape\bfseries \maketitle}

Elección subóptima se separa de teorías de maximización y se ha visto como modelo de gambling.

Se ha mostrado que las palomas son subóptimas y las ratas óptimas.
Entre las explicaciones propuestas están diferencias en saliencia incentiva, la sensibilidad a información temporal (con TLs largos se llegó a encontrar suboptimalidad en ratas, pero el resultado no se pudo replicar ni por el mismo laboratorio), diferencias en los modos de búsqueda activados para cada especie (general con luces o tonos, que predice conducta óptima; focal con palancas, que predice subóptima).

Timberlake dice que el modo focal de búsqueda se activa por claves que predicen el tiempo o lugar de presentación de comida, y que el modo se caracteriza por conducta más enfocada en la procuración de comida.
Esto, sumado a la facilidad de las ratas para navegar laberintos, sugiere que algunos estímulos de los laberintos activan el modo focal.
Esto señala la posibilidad de estudiar las preferencias de ratas en un entorno que involucre esas señales ecológicamente válidas: locomoción y alternativas definidas espacialmente.

Se evalúan ratas en un laberinto que incorpora estímulos y respuestas relacionados con el forrajeo natural para analizar su impacto en las preferencias en elección subóptima.

\section{Experiment 1}

\subsection{Method}

Ocho ratas Wistar, en pares, con restricción al 85\% del peso libre.

Se utilizó un laberinto con puertas de guillotina con detector infrarrojo.
Tenía tres espacios:
\begin{enumerate}
    \item Espacio de elección, con la puerta A en la parte trasera.
	Dos puertas en la parte frontal (B y C).
    \item Espacio de resultados, con puertas D, E, F y G, dividido en dos partes por una pared.
	Espacio de reforzamiento, en donde se entregaba comida desde dos receptáculos de pellet.
\end{enumerate}

Los espacios estaban conectados por túneles de PVC.
Además, después del espacio de reforzamiento había un túnel que regresaba a los animales al espacio de elección.

Para el preentrenamiento se utilizó una versión simplificada del laberinto.

\subsection{Procedure}

Los sujetos fueron habituados y entrenados a comedero y a los túneles.
Si durante el preentrenamiento los sujetos mostraban preferencia por las puertas B o C, la alternativa discriminativa era asignada a esa puerta.
Si no había preferencia, la asignación era contrabalanceada.

Se pasó al procedimiento de elección subóptima.
Durante las primeras 10 sesiones solo se presentaron ensayos forzados, divididos en ocho bloques de ocho ensayos.
De los cuatro ensayos discriminativos, dos eran positivos y dos negativos.
De los cuatro ensayos no discriminativos, dos eran ND1 y dos ND2.
Cada ensayo comenzó cuando los sujetos cruzaron la puerta A, momento en el que se abría la puerta B o C.
Tras cruzar la nueva puerta se abría ahora D, E, F o G con probabilidad .5.
D era la puerta discriminativa reforzada; E, no reforzada, y F y G eran no discriminativas con probabilidad de reforzamiento de .75.
5 segundos después de entregar comida, u 8 segundos después de llegar a la caja meta cuando no había reforzamiento, se abría la puerta final.

En las sesiones 11, 12 y 13 se agregaron dos ensayos de elección a cada bloque para un total de 16 ensayos de elección por sesión.

Después se procedió a una reversión de las alternativas discriminativa y no discriminativa para evaluar un sesgo de posición.
En las sesiones 14 a 23 solo hubo ensayos forzados, mientras que en 24, 25 y 26 hubo ensayos de elección.

{\noindent\scshape\bfseries Data analysis}

Como índice de preferencia en ensayos forzados se analizaron las latencias para cruzar las puertas discriminativa y no discriminativa en ensayos forzados.
Se calculo la mediana de cada sujeto y se realizó un {\scshape Anova} con los datos de las últimas cinco sesiones.

En ensayos libres se analizó la proporción de elección por la alternativa discriminativa.
Las medias de las tres sesiones de prueba de cada fase se analizaron con pruebas t de una muestra contra 0.5.

Como índice de discriminación se analizó la latencia para cruzar cada puerta de resultado (positivo, negativo, ND1 y ND2), y el tiempo de viaje para cada tipo de resultado.
Se calculo la mediana de cada resultado y se analizaron las últimas 5 sesiones con {\scshape Anova}.

\subsection{Results}

Se observó una latencia menor para entrar a la alternativa no discriminativa en los ensayos forzados.
En los ensayos de elección se encontró preferencia por la alternativa no discriminativa, con proporción de elección media de 0.033.

Un {\scshape Anova} mostró que el tipo de resultado tuvo un efecto significativo en la latencia para entrar en cada puerta, lo que indica discriminación.
Hubo una diferencia significativa entre resultados positivo y negativo, pero no entre ND1 y ND2.

Para los tiempos de viaje, se observó que el tiempo fue mayor para el resultado negativo que para los demás.

\subsection{Discussion}

Evalúa la generalidad de subcho en un contexto más ecológicamente válido en el que las alternativas se definen exclusivamente por su localización espacial, y se requiere esfuerzo alto para llegar a cada resultado.

Las ratas prefieren la alternativa discriminativa, y discriminan bien los túneles.

Hubo una rápida adquisición de la preferencia subóptima, especialmente durante la reversión.
Además, la preferencia fue más marcada que en cualquier otro estudio.

Se pueden hacer otras manipulaciones, como alterar las probabilidades de reforzamiento o la longitud de TL (mediante la longitud de los túneles).

En el segundo experimento se evaluaron probabilidades de .2 y .5, y se utilizaron túneles más largos.
También se implementó una economía cerrada, y en otra condición se permitió a los animales escapar del resultado.

\section{Experiment 2}

\subsection{Method}

Seis ratas Wistar en cajas individuales al 85\% de su peso libre.
El peso se mantuvo artificialmente durante el pre entrenamiento; después, se mantuvo mediante el acceso a la comida durante la sesión entre 80 y 92\%.

En el laberinto se integraron dos puertas de escape en el espacio de resultados, y se extendieron los túneles de 0.5 a 1.5 metros.
Había agua disponible libremente durante la sesión.

El preentrenamiento fue igual al anterior.

El entrenamiento difirió en que había ensayos de elección desde el inicio del entrenamiento y durante todo el experimento, se entregaban dos pellets como reforzador, y los ensayos se presentaron en bloques de 12, de los cuales cinco eran discriminativos, cinco no discriminativos, y dos de elección.
En los ensayos discriminativos uno presentaba el túnel positivo y cuatro el negativo; en los no discriminativos sucedía igual, pero cualquier túnel tenía probabilidad de reforzamiento de .5.
La condición terminaba cuando los sujetos concluían 20 bloques de 60 ensayos.
Después se implementó una reversión.
Finalmente, después de la reversión se implementó el escape, en el cual se presentaba una puerta disponible en el espacio de resultados y que llevaba de vuelta al espacio de elección.

{\noindent\scshape\bfseries Data analysis}

La preferencia se evaluó de la misma manera que en el experimento anterior.
La discriminación se analizó igual también al inicio, pero al introducir el escape se comenzó a evaluar como la proporción de ensayos escapados para cada resultado.

La latencia para entrar a la puerta no discriminativa fue menor que para la puerta discriminativa.
La proporción de elección mostró la misma tendencia.

A partir del quinto bloque de sesiones la latencia para el resultado negativo fue mayor que para los demás.
Lo mismo sucedió con el tiempo de cruce.

Sobre el escape, se observó una alta proporción de escapes durante el resultado negativo, y ningún escape en los otros resultados.

\subsection{Discussion}

Se replicaron los resultados del experimento 1 con probabilidades y longitudes diferentes, además de la economía cerrada.
También se replicó un alto nivel de discriminación.

Como evidencia adicional de discriminación está la proporción de escapes.

Finalmente, la economía cerrada no generó diferencias con respecto al experimento 1.
Las latencias para entrar a las alternativas fueron en general mayores para la economía cerrada, pero la diferencia entre alternativas se mantuvo.

\section{General discussion}

Se replica elección subóptima con alternativas diferenciadas por su ubicación espacial, con respuestas de locomoción para elegir, y con locomoción necesaria para acceder a la consecuencia final.
Aunque los índices de discriminación fueron distintos de los experimentos usuales, se encontró buena discriminación.

En economía cerrada no se encontraron diferencias relevantes aunque se ha observado que el tipo de economía tiene un efecto en la determinación del valor de un reforzador y en la relación entre un programa de reforzamiento y la tasa de respuestas que mantiene.
Aquí la economía cerrada tenía el único propósito de simular más cercanamente al forrajeo natural.

El procedimiento se parece a procedimientos que han buscado simular al forrajeo natural en modelos de dieta óptima.
El procedimiento prototípico de encuentros sucesivos tiene como características principales la simulación de diferentes fases de un episodio de forrajeo (búsqueda, elección, manejo y consumo).
Después de cumplir el programa que representa a la fase de búsqueda, la fase de elección se presenta cuando una de dos ``presas'' se encuentra.
Aquí, los organismos pueden aceptar o rechazar.
Así, los organismos nunca eligen entre dos resultados, sino que eligen aceptar o rechazar cada uno.

En este procedimiento la elección y el manejo están separados, los sujetos encuentran los resultados secuencialmente y no simultáneamente (en la mayoría de las sesiones del experimento 1), y los sujetos podían rechazar el resultado que se presentaba.
Esto, unido a la discusión de Timberlake de cómo el desempeño de ratas en laberintos se beneficia de la similitud entre los laberintos y sus entornos naturales, fortalece la analogía con el forrajeo natural.

En subcho el escape es un aspecto novedoso y relevante del procedimiento.
Modela la situación de forrajeo natural en la que, se propone, se basa la preferencia de las aves por la alternativa discriminativa.
Solo un estudio ha implementado escape en ratas y no fue exitoso dado que no escapaban consistentemente.

En este estudio los animales no fueron entrenados a escapar.
Es posible que la disposición de las ratas a escapar se relaciones con el uso de una topografía relevante (alejarse del resultado negativo) como respuesta de escape.
Además, la rapidez de adquisición y especificidad de la emisión de la respuesta indica que el resultado negativo puede ser aversivo, lo que podría incrementar la preferencia por la alternativa sin componente aversivo (no discriminativa).
Aunque esa interpretación sugeriría que al implementar el escape la preferencia por la alternativa discriminativa debería incrementar y no sucedió.
Quizá esto se deba a que el ahorro de tiempo y esfuerzo al escapar no fue suficiente para revertir la preferencia.

Se ha reportado que en estado estable el estímulo negativo no tiene propiedades de inhibición condicionada, lo que se ha tomado como base para su preferencia subóptima.
Sin embargo, investigación posterior reportó que la preferencia subóptima se encontró aunque había inhibición condicionada y que la fuerza de la inhibición correlacionaba con la preferencia subóptima.
Esto sugiere que las propiedades de inhibición condicionada por sí mismas no explican el desarrollo de preferencia subóptima, y sugiere la necesidad de analizar otras variables.
Una posibilidad es una diferencia en la aversión al estímulo negativo entre ratas y palomas.
Esa diferencia explicaría, además de la preferencia, la distinta proporción de escapes entre este estudio y el de Fortes con palomas.

Alargar los túneles pretendía incidir en el reporte de Cunningham y Shahan sobre cómo alargar los TL lleva a elección subóptima.
No se encontró un efecto, aunque es posible que la longitud simplemente no fuese suficiente.

En resumen se replicó elección subóptima en laberintos.


\end{document}
