\documentclass[a4paper,12pt]{article}
\usepackage[utf8]{inputenc}
\usepackage[T1]{fontenc}
\usepackage[spanish]{babel}
\usepackage{csquotes}
\usepackage{anysize}
\usepackage{graphicx}
\marginsize{25mm}{25mm}{25mm}{25mm}

\title{Meliorating the Suboptimal-Choice Argument}
\author{Marilia Pinheiro de Carvalho\and Cristina Santos\and Cristina Soares\and Armando Machado}
\date{2019}

\begin{document}
{\scshape\bfseries \maketitle}

En ``{\itshape What suboptimal choice tells us about the control of behavior}'' Zentall muestra que en condiciones simples los animales parecen comportarse de forma irracional haciendo elecciones subóptimas. Después introduce algunas hipótesis para explicarlo: heurísticos. Después menciona lo que la elección subóptima puede decir sobre el control de la conducta.

La parte dos contiene hipótesis comprobables, pero también otras conceptualmente confusas e inconsistentes; y la parte tres es demasiado vaga para ser útil.

{\scshape\bfseries Part 1. Suboptimal choice as a subset of surprising research findings}

Hay muchos reportes de conducta aparentemente irracional, como el caso del {\itshape instinctive drift} reportado por los Breland.

La novedad de los casos de Zentall es que implican elección explícita. En las tareas de {\itshape less-is-more} y recompensa efímera los animales eligen las alternativas que les dan menos comida; y en la {\itshape midsession reversal task} los animales eligen la opción incorrecta a pesar de que las claves locales son aparentemente fáciles de usar.

Cuando los hallazgos nos sorprenden, dictan una agenda para investigación empírica y teórica subsecuente. Empíricamente se intenta medir la confiabilidad y generalidad de los hallazgos. Teóricamente se intenta revisar las concepciones y teorías previas para reconciliarlas con los hallazgos nuevos. Por esto último es que Zentall clama que la elección subóptima puede ayudarnos a entender mejor el proceso de aprendizaje.

{\scshape\bfseries Part 2. Hypothetical processes of suboptimal choice}

Las hipótesis que presenta Zentall para ``guardar las apariencias'' son la contribución más importante:

{\scshape The less-is-better effect}

Macacos y chimpancés son indiferentes entre una opción mixta (comida preferida + no preferida) y una única (solo la comida preferida). Esto se explicó diciendo que los primates valoraban solamente la comida más preferida de cada opción e ignoraban la menos preferida. Pero una década después se mostró que los chimpancés preferían la comida única sobre la mixta, el efecto {\itshape less is better}.

Investigación posterior mostró que esto no se debe a un sesgo por cantidades menores de comida, ni porque la comida mixta tuviera valor positivo y negativo, o porque la comida menos preferida no tuviera ningún valor.

La hipótesis actual es que los dos items tienen valor positivo, y que los animales promedian sus valores. Zentall dice que este heurístico de promediar aplica cuando los animales escogen entre comidas de distintas calidades.

Aunque ingeniosa, la hipótesis tiene problemas: condiciones experimentales idénticas han llevado a resultados distintos y no sé sabe que determina cuál resultado será obtenido; no todos los sujetos muestran el efecto; cuando la opción mixta incluye las dos comidas altamente preferidas, y la opción singular contiene solo una de ellas, los animales prefieren la mixta, pero el grado de la preferencia varía con el contenido de la opción singular; cuando se elige entre 20g de la comida preferida junto con 5g de la misma, contra 20g de la comida preferida, los chimpancés son indiferentes o prefieren la opción singular. Zentall presenta solo los resultados consistentes con el heurístico de promedio, pero no presenta los inconsistentes.

Otros autores han propuesto un efecto de aversión a la comida fragmentada, como un mecanismo para proteger contra comida descartada por alguien más. Si esto fuese cierto, el efecto ya no sería subóptimo.

{\scshape The ephemeral reward task}

Palomas eligen entre teclas roja y verde. Picar rojo da comida y termina el ensayo, pero picar verde da comida y hace que R permanezca presente, de modo que también se puede obtener su comida. La preferencia por verde da el doble de comida.

Zentall propone la hipótesis de reforzamiento diferencial: Cuando se escoge verde, tanto verde como rojo se asocian con recompensa: verde con la primera y rojo con la segunda; cuando se escoge rojo solo ese color se asocia con recompensa. Mientras se acumulan los efectos de los ensayos rojo gana más fuerza asociativa que verde y por lo tanto es preferido.

Sin embargo especies distintas muestran preferencias distintas. Y para explicar estas diferencias Zentall utiliza la impulsividad. Animales impulsivos tienden a preferir rojo, pero podemos reducir su impulsividad colocando una demora entre la elección y la comida. Zentall argumenta que insertar la demora puede alentar a las palomas a asociar el segundo reforzador con la elección inicial de la alternativa B.

Pero la hipótesis es confusa dado que no está bien elaborada. No explica cómo la impulsividad previene las asociaciones, ni en su forma coloquial ni en la teórica (preferencia por recompensas pequeñas e inmediatas). Además, incrementar la demora entre eventos hace a la asociación más difícil de aprender. Sin embargo, la hipótesis asume lo contrario sin justificación. 

Con demoras insertadas palomas y ratas tienen preferencias óptimas. Zentall argumenta que la demora disminuye la impulsividad como en las respuestas de compromiso. Sin embargo, para extender la explicación de respuestas de compromiso a la tarea de recompensa efímera tendría que mostrarse que la opción verde es (S1) menos valorada que la opción roja cuando el consumo es inminente, y (S2) más valorada cuando el consumo está demorado.

Una interpretación de la tarea que no se basa en la impulsividad comienza con la hipótesis de Zentall de reforzamiento diferencial y le añade la idea de que las palomas y ratas fallan en asociar verde con la segunda recompensa dada la interferencia de los eventos intervinientes, incluida la primera recompensa y la respuesta en la tecla. Así, la demora es efectiva porque incrementa la exposición a verde y fortalece su huella de memoria, reduciendo los efectos de interferencia. 

{\scshape The midsession reversal task}

Dados dos estímulos S1 y S2, la elección de S1 se refuerza durante los primeros 40 ensayos, y la de S2 durante los últimos 40. Las palomas muestran errores anticipativos y perseverantes. Lo sorprendente es que las palomas confían en el timing en lugar de en las claves locales.

La primera explicación de Zentall fue que las palomas tienen dificultades recordando sus elecciones anteriores y resultados con intervalos entre ensayos relativamente largos. Dada esta dificultad, recurren al tiempo para determinar cuándo cambiar de una tecla a otra. 

Las dificultades de memoria pueden limitarse a las discriminaciones visuales y espaciales solamente según la evidencia, y podrían explicar el número relativamente pequeño de errores anticipatorios y perseverantes, pero solo al costo de no poder explicar el gran número de elecciones correctas en los ensayos restantes.

Además, el timing requiere memoria. Modelos representativos de timing tienen componentes de memoria.

La segunda explicación de Zentall dice que las palomas usan timing dada la simetría y excesivo número de claves confiables en la tarea. Las claves de reforzamiento pueden competir entre sí en maneras complejas.

Así, el timing se justifica no por limitación de memoria sino por competencia entre las claves. Pero esta explicación debe elaborarse más para acomodar los siguientes hallazgos: cuando el ITI se duplica, las palomas comienzan a escoger S2 alrededor del primer cuarto de la sesión, lo que corresponde aproximadamente al momento de la reversión original. Esto apoya la idea de que el tiempo desde el inicio de la sesión es la clave principal para pasar de S1 a S2, y es incompatible con la hipótesis de competencia de claves. Pero cuando ITI se reduce a la mitad, las palomas cambian poco después de la reversión y no al final de la sesión. Esto sugiere que el timing no es la única variable implicada. Cómo interactúan el timing y las claves locales permanece poco claro.

De acuerdo con la hipótesis de competencia de claves, cuando hay menos claves disponibles y S1 es la única fuente confiable de información, las palomas se adhieren a la realimentación de S1 y no cometen errores de anticipación ni perseverantes. Esta explicación parece predecir que cuando S2 es la única fuente confiable de información el desempeño debería mejorar también, lo que no es apoyado por los datos.

Una explicación alternativa es que la diferencia en probabilidades de reforzamiento entre S1 y S2 sesga el estimado de las palomas del momento de la reversión.

{\scshape\bfseries Part 3. Beyond the heuristic value of heuristics}

Zentall concluye que la elección subóptima puede explicarse en términos de heurísticos evolucionados que funcionan bien en la naturaleza pero a veces fallan en el laboratorio. Pero esta conclusión es anticipada porque la evidencia es, en el mejor de los casos, dudosa. 

En ocasiones la conducta aparentemente subóptima es en realidad conducta óptima cuando se revela su verdadera función biológica (como en el caso de la aversión a la comida fragmentada). Otros casos de conducta subóptima quizá solo señalan las limitaciones de nuestras teorías actuales. Zentall da un ejemplo cuando discute elecciones que parecen irracionales hasta considerar el horizonte de tiempo de los animales: en ocasiones la recompensa de alta probabilidad no provee suficientes nutrientes para sobrevivir el día. Al considerar esto la etiqueta de ``subóptimo'' se desecha.

Sin claridad conceptual sobre definiciones y usos y sin argumentos de palusibilidad, los heurísticos permanecen solo como historias. Pero aun si tienen esto, se necesita coordinar a los heurísticos con procesos conductuales conocidos. 

¿Qué nos dice en realidad la elección subóptima sobre el control de la conducta? Nos dice que no entendemos el fenómeno de la elección, que nuestras teorías están equivocadas o incompletas. No sabemos si el efecto de {\itshape less-is-better} se extiende a mezclas de tres o más items, o si el efecto de la recompensa efímera ocurre cuando las opciones son espacialmente diferentes.

Se necesita elaborar nuevas hipótesis sobre los procesos conductuales de la elección subóptima, clarificarlas y probarlas. Para ello es crucial conocimiento de la filogenia del animal. Sin embargo ese conocimiento es difícil de obtener.

Sobre los heurísticos, hay que quitarlos del domino de la ciencia privada donde ``todo se vale'' al dominio de la ciencia pública donde la coherencia y conceptual y sensibilidad empírica gobiernan. Y para ello hay que definirlos con claridad, identificar las condiciones que los activan, y coordinarlos con procesos conductuales conocidos. De lo contrario se multiplicarán y abarcarán mucho sin hacernos avanzar.

\end{document}
