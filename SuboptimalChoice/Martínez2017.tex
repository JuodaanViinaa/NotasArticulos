\documentclass[a4paper,12pt]{article}
\usepackage[utf8]{inputenc}
\usepackage[T1]{fontenc}
\usepackage[spanish]{babel}
\usepackage{csquotes}
\usepackage{anysize}
\marginsize{25mm}{25mm}{25mm}{25mm}

\title{Incentive salience attribution is not the sole determinant of suboptimal choice in rats: conditioned inhibition matters}
\author{Montserrat Martínez, Rodrigo Alba,\\*William Rodríguez, Vladimir Orduña}
\date{2017}

\begin{document}

{\scshape\bfseries \maketitle}

One possibility for explaining this inter-species difference is related to the impact of the stimulus that predicts non-reinforcement, which should acquire properties as conditioned inhibitor: while it has been shown via summation tests that conditioned inhibition dissipates very early in training for pigeons (Laude et al., 2014), it does not dissipate for rats (Trujano et al., 2016).

Recently, it has been proposed that rats’ and pigeons’ differential sensitivity to conditioned inhibitors is not a complete explanation of the inter-species difference in suboptimal behavior, and that this difference could be better understood by analyzing the incentive salience of the discriminative stimuli employed.

A careful analysis of the procedure indicates that the suboptimality found in the study by Chow et al. (2017) could have been promoted not only by the incentive salience of the discriminative stimuli, but also by the absence of a conditioned inhibitor. Given the strong conditioned inhibition effect shown by rats in suboptimal choice procedures, it seems plausible that the presence of a conditioned inhibitor would counteract the effect of the predictor of reinforcement, even when it has high incentive salience.

{\scshape\bfseries This experiment}

This study evaluates rats in the suboptimal choice task using levers as TL stimuli. In contrast to Chow et al., here, a lever signals non-reinforcement for the discriminative alternative. Two levers are also used as stimuli for the non-discriminative alternative. Nosepoke entries are the choice response. 

Under these conditions, rats behaved optimally and showed a robust discrimination.

The number of lever presses on the levers that were reliable predictors of reinforcement and of non-reinforcement was analized. Discrimination was also shown via the feeder entries.

{\scshape\bfseries Discussion}

For a complete assessment of these interpretations, the present experiment integrated within the same procedure both variables – the presence of a conditioned inhibitor, and discriminative stimuli with high incentive salience – with the goal of assessing their impact on choice behavior. The main result indicated that in a procedure that involves the presence of a conditioned inhibitor, rats behaved optimally, choosing the option associated with the highest probability of reinforcement, even though there was an alternative that presented discriminative stimuli with high incentive salience.

It is worth mentioning that the optimality found is not dependent on the specific probabilities of reinforcement associated with each alternative – .50 vs .75 in the discriminative and non-discriminative, respectively – given that previous research has also found it when these probabilities are .20 vs .50 (Trujano and Orduña 2015).

This optimal choice behavior stands in contrast to the report by Chow et al. (2017). The high number of lever presses observed in both studies suggests that the levers acquired incentive salience. The fact that even under these circumstances our subjects did not show suboptimal behavior, suggests that the difference in results from both studies could have arisen because of an important difference between the procedures: the conditioned inhibitor.

The optimal behavior displayed by our subjects suggests that the presence of a lever with conditioned inhibition properties counteracted the influence of the lever associated with reinforcement and diminished the value of the discriminative alternative.

In our study, the similarity between these stimuli allowed an initial generalization of the expectancy of reinforcement, and during the first 5 sessions (see Figs. 2 and 3), subjects responded on the lever that predicted non-reinforcement at the same degree than on the lever that predicted reinforcement. This expectancy of reinforcement, together with its omission, are two necessary conditions for the establishment of conditioned inhibition. The blackout employed by Chow et al. (2017) was clearly associated with the omission of reinforcement, and therefore met the second condition. However, given that in no moment of the training was there an expectancy of reinforcement during the blackout, the first condition was not satisfied, which could explain why conditioned inhibition did not emerge.

The presence of a conditioned inhibitor is irrelevant in studies with pigeons. However, the following evidence suggests that the presence or absence of conditioned inhibitors makes a difference for rats: a) when the discriminative stimuli were levers, the 2-stimuli procedure generated suboptimal choice (Chow et al., 2017), while the 4-stimuli procedure generated optimal behavior (the present results); b) when the discriminative stimuli were lights, and the probability of reinforcement was the same for the discriminative and the non-discriminative alternatives, subjects preferred the discriminative alternative in the 2-stimuli procedure (Chow et al., 2017), but were indifferent in the 4-stimuli procedure.

For situations in which conditioned inhibitors are absent, the incentive salience of the discriminative stimulus should play a relevant role. In contrast, in situations in which there is a conditioned inhibitor, incentive salience would play a less critical role, and be constrained by the impact of the conditioned inhibitor.


\end{document}
