\documentclass[a4paper,12pt]{article}
\usepackage[utf8]{inputenc}
\usepackage[T1]{fontenc}
\usepackage[spanish]{babel}
\usepackage{csquotes}
\usepackage{anysize}
\usepackage{graphicx}
% \usepackage{hyperref}
\marginsize{25mm}{25mm}{25mm}{25mm}

\title{Pharmacological evidence that 5-HT\textsubscript{2C} receptor blockade selectively improves decision making when rewards are paired with audiovisual cues in a rat gambling task}
\author{Wendy K. Adams \and Chris Barkus \and Jacqueline-Marie N. Ferland \and Trevor Sharp \and Catharine A. Winstanley}
\date{2017}

\begin{document}
{\scshape\bfseries \maketitle}

Claves pareadas con recompensas incrementan motivación y modulan conducta dirigida a metas; sensibilidad a ellas podría incrementar vulnerabilidad a adicciones. Quizá las máquinas de apuestas que explotan esta vulnerabilidad son especialmente malignas.

Agregar sonidos y luces a la rat gambling task incrementa elección riesgosa, de modo que quizá impacten los procesos de toma de decisiones y no solamente el involucramiento en la tarea. Dada la posible sinergia entre elección de riesgo, sensibilidad a las claves, y vulnerabilidad a la adicción, entender los mecanismos que subyacen la elección riesgosa guiada por claves puede llevar a desarrollar tratamientos.

La dopamina señaliza las propiedades motivacionales de estímulos condicionados y las adictivas de muchas drogas. La elección riesgosa dirigida por claves es sensible a la modulación por ligandos del receptor D\textsubscript{3}. Por tanto, los mecanismos neurobiológicos que subyacen la elección riesgosa dirigida por claves podrían traslaparse con aquellos que regulan conductas adictivas.

La serotonina también está implicada en procesos de adicción. Es uno de los principales reguladores de inhibición conductual, en particular los receptores 5-HT\textsubscript{2A} y 5-HT\textsubscript{2C}. Antagonistas de 5-HT\textsubscript{2A} y agonistas de 5-HT\textsubscript{2C} disminuyen la impulsividad motora y atenúan el reestablecimiento re búsqueda de drogas en modelos de recaída. Parece que los receptores tienen acciones opuestas en cuanto a impulsividad motora.

Manipulaciones serotonérgicas pueden modular conductas dirigidas por claves. Se hipotetiza que la adición de claves a rGT puede reclutar señalización de 5-HT\textsubscript{2A/C} en el proceso de elección. Se predice que el antagonismo de 5-HT\textsubscript{2A} y agonismo de 5-HT\textsubscript{2C} disminuirá la elección riesgosa dirigida por claves, y antagonismo de 5-HT\textsubscript{2C} hará lo contrario.

\section{Methods and materials}

\subsection{Subjects}

37 ratas Long-Evans macho.

\subsection{Behavioral apparatus}

Cajas Med con cinco nosepokes de un lado, y comedero con luz general en el otro.

\subsection{Behavioral training}

Los sujetos pasaron por la versión con o sin claves de la rGT, que es muy similar a Iowa Gambling Task. Cada nosepoke está asociado con magnitud, probabilidad y castigo diferentes. La estrategia óptima es elegir las alternativas con menor recompensa pero menor castigo asociado. La tarea con claves es igual a la normal salvo por que se presenta una clave audiovisual de 2s al mismo tiempo que se entrega la comida. La saliencia y complejidad de la clave incrementa con el tamaño de la recompensa ganada, y se ha encontrado que los roedores encuentran tales claves apetitivas.

Cada ensayo iniciaba con una entrada en el comedero encendido, lo que apagaba su luz e iniciaba un ITI de 5s. No responder durante el ITI llevaba al encendido de los nosepokes 1, 2, 4 y 5. Responder en ellos llevaba a reforzamiento o a castigo. En reforzamiento se entregaba la comida de inmediato. En castigo parpadeaba la luz del nosepoke elegido durante el tiempo fuera programado. Tras completar cualquier ensayo iniciaba el siguiente. Una respuesta en los nosepokes durante el ITI llevaba a un castigo de 5 segundos registrado como respuesta prematura en el que se encendía la luz general. No responder tras diez segundos de presencia de los nosepokes se contaba como omisión. Las respuestas prematuras sirven como índice de impulsividad motora.

El entrenamiento continuó durante 44 sesiones.

\subsection{Effects of systemic M100907, SB 242,084 and Ro-60-0175 on performance of the cued and uncued rGT}

Se administraron las drogas en el siguiente orden:
\begin{enumerate}
    \item Ro-60-0175
    \item M100907
    \item SB 242,084
\end{enumerate}
Todas se administraron 15 minutos antes de las pruebas conductuales. Las drogas se daban en un ciclo de 4 días: primero una sesión de línea base, luego administración de droga o vehículo antes de la prueba, luego prueba sin droga, y finalmente un día de descanso.

\subsection{Data analyses}

Se analizaron las siguientes variables: puntuación (suma de elecciones ventajosas menos la suma de elecciones desventajosas), porcentaje de elección de cada opción, porcentaje de respuestas prematuras, porcentaje de omisiones, ensayos completados, y latencias de elección y de recolección de recompensa. Se aplicó una transformación de arco seno a los datos expresados en porcentaje para limitar la influencia de un techo artificialmente impuesto. Antes de cambiar de drogas se verificaba un patron de respuesta estable mediante {\scshape Anova} de medidas repetidas de las últimas tres sesiones.

Estos análisis determinaban el nivel basal de preferencia de riesgo.

Los efectos de las drogas se analizaron mediante {\scshape Anova} de medidas repetidas con dosis y, cuando aplicara, elección como factores entre sujetos. Si se encontraba un efecto principal significativo de dosis o una interacción dosis $\times$ elección con $p < 0.05$, {\scshape Anovas} adicionales comparando las dosis con el vehículo se realizaban y se comparaban los valores con los valores de salina mediante pruebas t simples.

\section{Results}

\subsection{Effects of 5-HT\textsubscript{2A} and 5-HT\textsubscript{2C} receptor ligands on choice}

La elección riesgosa fue mayor en la versión con claves tras la administración del vehículo. Ni el agonista de 5-HT\textsubscript{2C} ni el antagonista de 5-HT\textsubscript{2A} tuvieron efectos significativos en la elección. En contraste el antagonista de 5-HT\textsubscript{2C} incrementó significativamente y de forma dependiente de la dosis la puntuación de los animales en la versión con claves de rGT, pero no en la versión sin claves. Un análisis reveló que el efecto en la versión con claves resultó principalmente de un incremento en la selección de P2, la alternativa más ventajosa.

\subsection{Effects of 5-HT\textsubscript{2A} and 5-HT\textsubscript{2C} receptor ligands on premature responding}

Los niveles de respuestas prematuras no difirieron entre las versiones con y sin clave de rGT.

M100907 no atenuó las respuestas prematuras en la versión sin claves, pero produjo reducciones dramáticas en esta medida en la versión con claves. Aunque SB 242,084 no alteró los patrones de elección en la versión sin claves sí incrementó significativamente las respuestas prematuras. Un incremento similar se dio en la versión con claves.

\subsection{Effects of 5-HT\textsubscript{2A} and 5-HT\textsubscript{2C} receptor ligands on other task variables}

La elección era más rápida en la versión sin claves, pero el tiempo de recolección fue igual entre versiones. Ro-60-0175 y M100907 incrementaron la latencia de elección en ambas versiones. SB 242,084 incrementó la latencia de elección y el tiempo de recolección en ambas variantes.

Las omisiones fueron bajas en ambas variantes y las drogas no las alteraron. Menos ensayos fueron completados en la versión con claves.

\section{Discussion}

Se muestra que la adición de claves a la rGT recluta diferencialmente neurotransmisión serotonérgica en el proceso de toma de decisiones. La elección fue sensible a la modulación del antagonista del receptor 5-HT\textsubscript{2C}. Contrario a la hipótesis, la droga disminuyó la capacidad de las claves para potenciar la elección de las alternativas riesgosas. La mejora en la decisión fue concurrente con un incremento en las elecciones prematuras. La misma droga además incrementó las respuestas prematuras en la versión sin claves.

Los datos apoyan la hipótesis de que la acción impulsiva y la toma de decisiones riesgosas pueden disociarse farmacológicamente, además de indicar que el bloqueo del receptor 5-HT\textsubscript{2C} tiene simultáneamente efectos benéficos y perjudiciales en los procesos cognitivos asociados con la vulnerabilidad a las adicciones.

En la versión con claves se encontraron mayores niveles basales de elección riesgosa comparada con la versión normal, además de menor número de ensayos completados.

La latencia de elección fue mayor en la versión con claves, contrario al reporte anterior. Se requerirán estudios adicionales para determinar si la toma de decisiones genuinamente es más lenta cuando las ganancias son pareadas con claves audiovisuales, o si el efecto observado entre tareas es solo una anomalía estadística indicativa de una variación poblacional en la cepa Long-Evans.

Aunque no se afectó la elección en la versión sin claves por ninguna droga, reportes previos indican que la 5-HT puede mediar la toma de decisiones en la tarea. Puede ser que la elección en la tarea sea sensible a disminuciones globales en 5-HT pero no a actividad disminuida en los subtipos 5-HT\textsubscript{2} probados aquí. Varios de los más de 14 subtipos de receptores han sido implicados en otras formas de toma de decisiones. Ligandos para receptores 5-HT\textsubscript{2B}, 5-HT\textsubscript{6}, y 5-HT\textsubscript{7} deben explorarse con las versiones con y sin clave de rGT. La mejora en la elección en la versión con  claves tras la administración de un antagonista de 5-HT\textsubscript{2C} apoya investigación adicional sobre los efectos potenciales de distintos compuestos serotonérgicos en este tipo de toma de decisiones.

Sobre la especificidad de los efectos conductuales observados, aunque SB 242,084 decrementó el número de ensayo s completados en la versión sin claves, no lo alteró en la versión con claves. Ninguna droga alteró los ensayos omitidos. Cualquier cambio en la impulsividad motriz o elección riesgosa no puede confundirse con alteraciones globales en la disposición para involucrarse en la tarea. R-60-0175 y M100907 incrementaron la latencia, y R-60-0175 incrementó la latencia para recoger la recompensa en ambas versiones de la tarea. Si la reducción en impulsividad motora refleja un estilo cognitivo más deliberado o menor excitación anticipatoria en relación a la entrega de recompensa no queda claro.

No es claro por qué es necesaria la presencia de claves concurrentes a la recompensa para que SB 242,084 acelere la latencia de recolección de recompensa y module los patrones de elección, y el mecanismo cognitivo que subyace a la capacidad de tales claves para potenciar la elección de riesgo aun está por determinarse. Esto seguramente se apoyará de evidencia de psicofarmacología. Las ratas no muestran sesgo hacia claves más salientes en una prueba de preferencia simple, por lo que asociar la recompensa con una clave compleja no es suficiente para amplificar su valor. No es claro si las claves se volvieron reforzadores condicionados, pero es posible. Un reporte reciente sugiere que ratas con mayores niveles de respuesta ante el reforzamiento condicionado tienen mejor desempeño en la rGT con claves. Así, animales que pueden resolver la estrategia óptima mejor en rGT con claves podrían tener mayor motivación incentiva para las claves ambientales asociadas con recompensas. Impulsar esta motivación incentiva podría mejorar la toma de decisiones cuando las recompensas están pareadas con estímulos condicionados.

Entender los mecanismos cognitivos por los cuales los estímulos condicionados influyen en la toma de decisiones es un área de interés creciente dada la evidencia de que tales claves modulan el involucramiento en gambling. La evidencia sugiere que máquinas con claves audiovisuales evocan un estado de inmersión en jugadores asociado con el estrechamiento de la atención.

Evidencia indica que receptores 5-HT\textsubscript{2C} pueden ser hipersensibles en TOC, y esto puede contribuir a involucramiento compulsivo en ciertas tareas. Otros datos sugieren que el antagonismo de 5-HT\textsubscript{2C} puede exacerbar conductas compulsivas. Determinar si los agonistas o antagonistas de 5-HT\textsubscript{2C} son beneficiosos para aminorar rasgos relacionados con la adicción puede depender del grado en que la conductas cognitivas dominan en cada individuo.


\end{document}
