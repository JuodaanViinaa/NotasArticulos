\documentclass[a4paper,12pt]{article}
\usepackage[utf8]{inputenc}
\usepackage[T1]{fontenc}
\usepackage[spanish]{babel}
\spanishdecimal{.}
\usepackage{csquotes}
\usepackage{anysize}
\usepackage{graphicx}
\marginsize{25mm}{25mm}{25mm}{25mm}

\title{Forced-exposure trials increase suboptimal choice}
\author{Margaret McDevitt \and Jeffrey M. Pisklak \and Roger M. Dunn \and Marcia L. Spetch}
\date{2022}

\begin{document}
{\scshape\bfseries \maketitle}

El fenómeno de elección subóptima ha recibido gran atención reciente.

El fenómeno señala la importancia de las señales informativas.
Dunn y Spetch encontraron preferencia por una alternativa de 50\% de reforzamiento sobre otra de 100\% siempre que la primera fuese informativa.

En los primeros reportes de elección subóptima hicieron falta los ensayos forzados, en los que se presenta únicamente una de las alternativas.
Estudios recientes con ensayos forzados han encontrado comportamiento subóptimo más extremo y menos variable, pero otras diferencias en los procedimientos dificultan determinar la influencia de los ensayos forzados en la reducción de variabilidad.

Se pueden comparar estudios con y sin inclusión de ensayos forzados.
Spetch et al (1990) no usaron ensayos forzados con probabilidades de 50 y 100\% y encontraron preferencia fuerte por la alternativa óptima.
En contraste, el grupo de Zentall usó dos ensayos forzados por cada uno de elección y encontró indiferencia o una preferencia pequeña por la alternativa subóptima bajo los mismos parámetros.
Belke y Spetch usaron un procedimiento asimétrico en el que las elecciones a la alternativa subóptima que llevaban a omisión siempre eran seguidas por ensayos forzados repetidos hasta que se entregaba comida, y encontraron una preferencia aun mayor por la alternativa subóptima.
Estas diferencias sugieren que la presencia y arreglo de ensayos forzados puede afectar el desarrollo de la preferencia.
Este estudio es la primera evaluación directa de los ensayos forzados.


Hay una paradoja en sugerir que los ensayos forzados promueven la elección subóptima, porque los ensayos forzados incrementan la oportunidad de aprender que la alternativa subóptima provee menos comida.
Baum ha sugerido que la conducta subóptima refleja entrenamiento insuficiente o un fallo para hacer discriminaciones difíciles.
Si fuese así, agregar ensayos forzados debería disminuir la preferencia subóptima.

Este estudio investiga el grado en que la inclusión de ensayos forzados influye en la preferencia.
Ambos experimentos presentaron probabilidades de reforzamiento de 20 y 50\%.
Este procedimiento produce fuerte preferencia subóptima cuando hay señales diferenciales en la alternativa subóptima, y una reversión a fuerte preferencia óptima cuando no hay señales.
El experimento 1 presentó distintos niveles de experiencia con ensayos forzados y de elección.
Un grupo experimentó solo ensayos de elección, mientras que otro experimentó una proporción 2:1 de ensayos forzados y de elección, y uno más experimentó solo ensayos forzados antes de la inclusión de ensayos de elección.
En el experimento 2 la mitad de los sujetos fue entrenada con 10\% de ensayos forzados, mientras que los demás entrenaron con 90\% de ensayos forzados.
Ambos grupos de ese experimento pasaron primero por una condición no señalada, seguida de una condición señalada.

\section{Experiment 1}

\subsection{Method}

Los sujetos fueron 12 palomas con experiencia en cadenas concurrentes y en discriminación simple.

Se utilizaron cajas operantes con tres teclas de respuesta, aunque solo se usaron las teclas laterales.

Las alternativas eran señaladas por teclas con un círculo negro, y diferían solamente en la posición.
La localización era constante pero el lado asociado con cada alternativa fue contrabalanceado entre palomas. 
Había un intervalo entre ensayos de 5 s, u los eslabones terminales eran de 10 s.

Las aves fueron separadas al azar en tres grupos.
Para el grupo {\itshape all choice}, todos los ensayos eran de elección.

Para el grupo {\itshape 67\% FE}, cada sesión era una combinación de ensayos forzados y de elección.
Cada bloque de tres ensayos tenía dos forzados y uno de elección, pero el orden era aleatorizado en cada bloque.

En el grupo {\itshape all FE} todos los ensayos eran forzados durante el entrenamiento.
Tras 25 sesiones de entrenamiento se presentaron sesiones de prueba con un ensayo de elección por cada dos forzados.

Se determinó estabilidad cuando en tres bloques consecutivos de tres sesiones (a) la media entre bloques se desviaba en 0.05 o menos, y (b) no había una tendencia hacia arriba o hacia abajo.

\subsection{Results}

La proporción de elección se calcula como la media de las nueve sesiones de estabilidad.

Las palomas del grupo {\itshape 67\% FE} desarrollaron preferencia consistente y extrema por la alternativa subóptima.
Cada paloma en {\itshape all FE} mostró preferencia similar y extrema por la alternativa subóptima y después, en el transcurso de las siguientes ocho sesiones, la preferencia bajó ligeramente, principalmente por una paloma.
En contraste, el grupo {\itshape all choice} empezó con una preferencia inicial por la alternativa óptima, pero mostró patrones individuales variables según avanzaron las sesiones.
La media se estabilizó cerca de la indiferencia, pero varió de preferencia exclusiva por la alternativa óptima a exclusiva por la subóptima.

\section{Experiment 2}

Se usaron 10 palomas nuevas y 2 del experimento anterior.
Se separaron al azar entre los grupos de 10\% y 90\%.

Antes del experimento cada ave recibió preentrenamiento de dos o tres sesiones en las que las picadas al estímulo usado en el experimento fueron reforzadas de acuerdo con programas FR1, FR20 y FR35.

El procedimiento consistió en la presentación de las alternativas óptima y subóptima, a veces individualmente y a veces simultáneamente.
La proporción de ensayos forzados difirió entre grupos (10\% vs 90\%).


\end{document}
