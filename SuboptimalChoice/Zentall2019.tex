\documentclass[a4paper,12pt]{article}
\usepackage[utf8]{inputenc}
\usepackage[T1]{fontenc}
\usepackage[spanish]{babel}
\usepackage{csquotes}
\usepackage{anysize}
\usepackage{graphicx}
\marginsize{25mm}{25mm}{25mm}{25mm}

\title{Differences in rats and pigeons suboptimal choice may depend on where those stimuli are in their behavior system}
\author{Thomas R. Zentall, Aaron P. Smith \and Joshua Beckmann}
\date{2019}

\begin{document}
{\scshape\bfseries \maketitle}

No se suele dar mucha importancia a las variables organísmicas en la conducta. Timberlake propuso analizar la conducta en el contexto más amplio de {\itshape sistemas conductuales} en términos del grado en que la conducta reforzada es consistente con la natural del organismo. El aprendizaje sería un proceso de ajuste en sistemas conductuales ya funcionales.

Esto indica que estímulos breves debería controlar conductas con resultas proximales, y estímulos más largos deberían controla conductas más generales como la exploración y la investigación.

Cuando las especies muestran conductas cualitativamente diferentes en respuesta a procedimientos similares probablemente la respuesta esté en la ecología natural de las especies, y no en mecanismo subyacentes de aprendizaje diferentes.

Así, se aplica una aproximación de sistemas conductuales a elección subóptima. Se propone que las diferencias entre especies vienen de diferencias ecológicas en las condiciones que llevan a obtener el efecto.

{\scshape\bfseries The suboptimal choice task with pigeons}

Las palomas eligen la alternativa subóptima, contradiciendo a la predicción de que maximizarán ganancias y minimizarán costos. Esto  no parece resultar de la incertidumbre de la alternativa óptima.

Parece ser que las palomas eligen la alternativa subóptima con base en el valor del reforzador condicionado y no en la probabilidad de reforzamiento.

Se ha propuesto que las señales de la alternativa subóptima proveen información y eso las hace valiosas. También que el estímulo positivo señala reforzamiento más pronto que los estímulos de la alternativa óptima. Y también que el contraste positivo es el responsable del efecto.

Si la probabilidad de reforzamiento no parece importar, eso parece sugerir que los estímulos que predicen la ausencia de reforzamiento no inhiben la elección. Quizá esto se deba a que en la naturaleza esos estímulos solo llevan a las palomas a buscar en otro sitio, cuando en el procedimiento esos estímulos no se pueden rechazar.

Según la teoría de sistemas conductuales un estímulo asociado con la ausencia de reforzamiento debería interferir con la búsqueda focal pero puede no afectar la elección en ensayos posteriores. Además, en la naturaleza aproximarse a un reforzador condicionado ({\itshape i.e.,} un árbol frutal) incrementaría la probabilidad de encontrar reforzadores ({\itshape i.r.,} fruta), pero en el procedimiento aproximarse a los reforzadores condicionados no altera la probabilidad de encontrar comida.

{\scshape\bfseries The suboptimal choice task with rats}

Dado que las ratas eligen la alternativa óptima, se ha hipotetizado que son sensibles a los inhibidores condicionados, lo que fue mostrado por pruebas de sumación. Esto llevó a proponer que las especies pueden ponderar diferencialmente los resultados riesgosos precedidos por claves correlacionadas. Pero antes de llegar a esa conclusión se debe examinar la posibilidad de que la diferencia esté en las condiciones de estímulo bajo las cuáles se evaluaron las especies (hipótesis de la inhibición condicionada).

Basados en la perspectiva de los sistemas conductuales se hipotetiza que las claves visuales representan niveles distintos de búsqueda para ratas y palomas. Las luces y los tonos activarían una secuencia de búsqueda general en ratas, y la aparición de la palanca activaría una secuencia más específica caracterizada por la captura de la presa. Para las palomas los estímulos visuales dispararían una búsqueda focalizada de este último tipo. Así, los estímulos asociados con la búsqueda focal desarrollan saliencia incentiva, y los demás evocan un modo de búsqueda general.

{\scshape\bfseries Incentive salience: Between-subject vs Between-stimulus effects}

Las diferencias individuales en tasa de {\itshape sign-tracking} no predicen la elección subóptima. Pero los hallazgos en elección subóptima corroboran investigación previa que sugiere que estímulos distintos afectan al grado de conducta de {\itshape sign-tracking}.

La diferencia en resultados entre luces y palancas en ratas sugiere que las palancas tocan una estrategia conductual distinta del repertorio de las ratas, quizá dado que es un estímulo táctil y móvil que hace ruido. Lo mismo para las palomas con estímulos visuales. El {\itshape sign-tracking} que muestran ambas especies con su respectivo estímulo podría reflejar la consistencia de ese estímulo con la conducta consumatoria de la especie.

Esto indicaría la dificultad de producir elección subóptima en palomas cuando los estímulos fuesen auditivos.

{\scshape\bfseries Incentive salience: Necessary but may not be sufficient}

Martínez utilizó palancas como estímulos e incluyó un inhibidor condicionado. A pesar de ello encontraron elección óptima. Una explicación es que pudo ocurrir algo de generalización entre las señales de ganancia y pérdida. Pero estos resultados indican que la saliencia incentiva podría no ser suficiente.

{\scshape\bfseries Concluding remarks}

La teoría de sistemas conductuales sugiere que distintos estímulos se asocian con aspectos distintos de la conducta relacionada con recompensas que son determinados por la recompensa misma y por la conducta de forrajeo específica de la especie. Así, la pregunta no es ``¿las ratas eligen de forma subóptima?'' sino ``¿en qué condiciones están predispuestas a hacerlo?'' y ``¿por qué existen estas diferencias?''.

La literatura existente parece sugerir que la suboptimalidad depende de estímulos que toquen conducta relacionada con recompensas específica de la especia que compita con contingencias operantes económicamente ventajosas.

Entender las condiciones de frontera para el efecto entre especies será cada vez más importante para establecer mecanismos comunes de toma de decisiones. La teoría de sistemas conductuales puede ayudar a esa meta.


\end{document}
