\documentclass[a4paper,12pt]{article}
\usepackage[utf8]{inputenc}
\usepackage[T1]{fontenc}
\usepackage[spanish]{babel}
\spanishdecimal{.}
\usepackage{csquotes}
\usepackage{anysize}
\usepackage{graphicx}
\marginsize{25mm}{25mm}{25mm}{25mm}

\title{Rats' performance in a suboptimal choice procedure implemented in a natural-foraging analogue}
\author{Fernanda González-Barriga \and Vladimir Orduña}
\date{2024}

\begin{document}
{\scshape\bfseries \maketitle}

Ratas óptimas, palomas subóptimas.
Una explicación propuesta es la diferencia en la saliencia incentiva de los estímulos discriminativos.
Chow siguió este argumento, pero no fue replicado en otros estudios.

Otra explicación se basa en una sensibilidad diferencial a la información temporal que dan los estímulos.
Cunningham y Shahan encontraron preferencia subóptima cuando los TL son de al menos 30 s, pero tampoco se pudieron replicar esos resultados ni en su propio laboratorio.
Además, Alba {\itshape et al} han encontrado preferencia óptima incluso con TL de 50 s.

Otro intento está en la diferencia en la relación de los organismos con los estímulos el laboratorio y los naturales: picar teclas podría activar un modo focal de búsqueda de comida, mientras que en las ratas las luces y tonos evocan un modo de búsqueda general, pero las palancas evocan uno focal.
Se ha encontrado preferencia óptima aun empleando luces como estímulos, así que parece justificado evaluar el impacto de otros tipos de estímulo que se presume se relacionan fuertemente con el modo focal en las ratas.


\end{document}
