\documentclass[a4paper,12pt]{article}
\usepackage[utf8]{inputenc}
\usepackage[T1]{fontenc}
\usepackage[spanish]{babel}
\usepackage{csquotes}
\usepackage{anysize}
\marginsize{25mm}{25mm}{25mm}{25mm}

\title{Evolved Psychological Mechanisms as Constrains on Optimization}
\author{Marco Vasconcelos, Alejandro Macías}
\date{2019}

\begin{document}
{\scshape\bfseries \maketitle}

Dado que los mecanismos conductuales estan sujetos a la selección natural, pueden dar la impresión de estar diseñados para maximizar la adecuación inclusiva ({\slshape inclusive fitness}). Malinterpretando esta premisa, algunos investigadores concluyen que toda (o casi toda) la conducta observada debería ser flexible y adaptable a las demandas ambientales. Dado que fue esculpida por la acción optimizante de la selección natural, se asume que la conducta misma debería ser óptima. Esto esta mal, porque quien optimiza no es el organismo y su conducta, sino la selección natural, que actuó en el pasado. Los mecanismos serán adaptativos solo en la medida en que las condiciones actuales sean similares a las condiciones que les dieron forma. Así, la conducta será óptima en el ambiente que un organismo encontraría típicamente en su nicho ecológico, y subóptima en otros ambientes.

Aunque en promedio los mecanismos conductuales tendrán resultados favorables (pues de lo contrario, serían eliminados por la selección natural), pueden ocasionalmente llevar a conducta subóptima cuando el ambiente es decididamente diferente. Hay muchas situaciones así, y más que mostrar evidencia contra la aproximación normativa, muestran oportunidades para identificar los mecanismos y entender su significancia adaptativa.

Zentall ha mostrado muchas situaciones así: {\slshape less-is-better effect, ephemeral choice task, midsession reversal task, suboptimal choice task}. Concluye que su origen son heurísticos evolucionados que se desempeñan bien en la naturaleza, pero que a veces fallan en condiciones de laboratorio.

{\slshape Heurísticos} son atajos utilizados para llegar a soluciones rápidas y eficientes (aunque no sean óptimas) a problemas prácticos. Esta noción ha sido explorada en varias disciplinas. En la ecología conductual le llaman reglas de dedo a esos heurísticos. La idea general es que la evolución dotará a los organismos con soluciones específicas para los problemas que se encuentran repetidamente a través de generaciones. Esto implica la existencia de una amplia colección de reglas de dedo, y un mecanismo que seleccione qué regla usar en cada situación. Por ejemplo, en el efecto de {\slshape less-is-better}, Zentall propone un heurístico que promedia el valor de las alternativas y reduce el valor de la opción con más comida, pero no queda claro en qué situaciones tal mecanismo sería funcional.

Se argumenta, al contrario, que dada la complejidad de los entornos, la selección natural debió dotar a los organismos con una serie de mecanismos multipropósito que se desempeñan bien en general, en lugar de una serie de reglas complejas para situaciones específicas.

Al encontrarse con preferencias aparentemente paradójicas, uno debería preguntarse cómo estas se podrían originar de mecanismos ya conocidos operando en circunstancias distintas a aquellas en las cuales se desarrollaron. En otras palabras, se reemplazan los heurísticos por procesos de aprendizaje. Solo al ser necesario los procesos conocidos deberían ser suplementados por nuevas proposiciones.

En conclusión, los mecanismos de aprendizaje son producto de la selección natural, y son a la vez adaptaciones y limitaciones a la optimización. Al entender esta naturaleza dual de los mecanisos, las preferencias subóptimas se vuelven herramientas para entenderlos y no para desacreditarlos.


\end{document}
