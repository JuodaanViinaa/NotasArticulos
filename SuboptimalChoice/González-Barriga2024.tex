\documentclass[a4paper,12pt]{article}
\usepackage[utf8]{inputenc}
\usepackage[T1]{fontenc}
\usepackage[spanish]{babel}
\spanishdecimal{.}
\usepackage{csquotes}
\usepackage{anysize}
\usepackage{graphicx}
\marginsize{25mm}{25mm}{25mm}{25mm}

\title{Incentive-salience attribution is attenuated in spontaneously hypertensive rats, an animal model of ADHD}
\author{Fernanda González-Barriga \and Vladimir Orduña}
\date{2024}

\begin{document}
{\scshape\bfseries \maketitle}

La capacidad de atribuir saliencia incentiva es un componente fundamental de la motivación.
El fenómeno de saliencia ocurre cuando un estímulo condicionado adquiere propiedades que reflejan un proceso motivacional: a) el estímulo captura la atención y evoca aproximación hacia sí, b) se convierte en un reforzador condicionado, y c) evoca estados motivacionales condicionados asociados con conducta de búsqueda de recompensas.

Niveles anormales de atribución de saliencia incentiva se asocian con desórdenes conductuales.
Se relaciona con índices de conducta adictiva en roedores, y ayuda a entender desórdenes psiquiátricos en humanos.

La motivación se asocia con el TDAH, y en él se han encontrado alteraciones en la sensibilidad a la recompensa.

Las ratas SHR son un modelo fundamental en la investigación del TDAH.
Sin embargo, sus procesos motivacionales son poco estudiados, comparados con el estudio de su sensibilidad a los parámetros de reforzamiento.

Se usa la {\itshape Pavlovian Conditioned Approach task} (PCA) para estudiar la atribución de saliencia incentiva en no-humanos.
En ella cada ensayo tiene un inicio impredecible y presenta una palanca iluminada por 8 s seguidos de la entrega de comida.
Dos respuestas pueden aparecer en los animales: a la palanca ({\itshape sign-tracking}) y al comedero ({\itshape goal-tracking}).
El {\itshape sign-tracking} se presume como indicio de 


\end{document}
