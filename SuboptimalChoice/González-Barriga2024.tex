\documentclass[a4paper,12pt]{article}
\usepackage[utf8]{inputenc}
\usepackage[T1]{fontenc}
\usepackage[spanish]{babel}
\spanishdecimal{.}
\usepackage{csquotes}
\usepackage{anysize}
\usepackage{graphicx}
\marginsize{25mm}{25mm}{25mm}{25mm}

\title{Incentive-salience attribution is attenuated in spontaneously hypertensive rats, an animal model of ADHD}
\author{Fernanda González-Barriga \and Vladimir Orduña}
\date{2024}

\begin{document}
{\scshape\bfseries \maketitle}

La capacidad de atribuir saliencia incentiva es un componente fundamental de la motivación.
El fenómeno de saliencia ocurre cuando un estímulo condicionado adquiere propiedades que reflejan un proceso motivacional: a) el estímulo captura la atención y evoca aproximación hacia sí, b) se convierte en un reforzador condicionado, y c) evoca estados motivacionales condicionados asociados con conducta de búsqueda de recompensas.

Niveles anormales de atribución de saliencia incentiva se asocian con desórdenes conductuales.
Se relaciona con índices de conducta adictiva en roedores, y ayuda a entender desórdenes psiquiátricos en humanos.

La motivación se asocia con el TDAH, y en él se han encontrado alteraciones en la sensibilidad a la recompensa.

Las ratas SHR son un modelo fundamental en la investigación del TDAH.
Sin embargo, sus procesos motivacionales son poco estudiados, comparados con el estudio de su sensibilidad a los parámetros de reforzamiento.

Se usa la {\itshape Pavlovian Conditioned Approach task} (PCA) para estudiar la atribución de saliencia incentiva en no-humanos.
En ella cada ensayo tiene un inicio impredecible y presenta una palanca iluminada por 8 s seguidos de la entrega de comida.
Dos respuestas pueden aparecer en los animales: a la palanca ({\itshape sign-tracking}) y al comedero ({\itshape goal-tracking}).
El {\itshape sign-tracking} se presume como indicio de atribución de saliencia incentiva a la palanca.

Al comparar a ratas SHR con Sprague-Dawley se encontraron menores niveles de atribución en SHR.

En este estudio se comparan ratas SHR con Wistar por ser comunes.

\section{Method}

\subsection{Subjects}

45 Wistar y 45 SHR ingenuas, de dos meses.

\subsection{Apparatus}

Diez cajas operantes estándar con dos palancas retráctiles iluminables y un comedero.


\subsection{Procedure}

Después de habituación y entrenamiento al comedero, inició la PCA task.
Se presentaban 25 ensayos por sesión que consistían en 8 segundos de presentación de una palanca (izquierda para la mitad de sujetos, derecha para la otra mitad) iluminada seguida de la entrega de un pellet en el comedero.
El ITI era variable de 30 a 150 s.
Los sujetos corrieron 12 sesiones en días consecutivos.

El desempeño se evaluó con el índice PCA, que se calcula con el número de respuestas, la probabilidad de respuesta, y la latencia a palanca y comedero.
1) Se considera la diferencia entre el número de presiones a la palanca y el número de entradas al comedero, dividida entre la suma de ambas respuestas.
2) Se calcula la diferencia entre probabilidades de presionar la palanca y entrar al comedero.
Y 3) se calcula la diferencia entre la latencia media a presionar la palanca y a entrar al comedero.
Si no hay respuesta en un ensayo la latencia se toma como 8 s.
El índice PCA es el promedio de 1, 2, y 3, y va de -1 a 1.
Un valor negativo indica {\itshape goal-tracking}; uno positivo, {\itshape sign-tracking}.

\section{Results}

Los índices mostraron desviaciones de normalidad y heterogeneidad de varianzas no corregibles por transformaciones, así que se usó U de Mann-Withney.

Aunque hubo diferencias en todas las sesiones menos la segunda, interesa que el PCA de Wistar fue mayor al final y a la sesión 5 (que es cuando otros estudios se detienen).

Para la sesión 12 38\% de Wistar eran {\itshape sign-trackers}, 38\% intermedios y 29\% {\itshape goal-trackers}.
Para SHR los porcentajes fueron de 2\%, 54\% y 44\% respectivamente.
La diferencia en proporciones fue estadísticamente significativa.

Las variables individuales del PCA se analizaron también y en general se confirmaron los hallazgos.

Dada la hiperactividad de SHR se quería determinar si su mayor número de entradas a comedero eran un artefacto o reflejaban verdaderamente un proceso de aprendizaje.
Para ello se analizó el número de entradas a comedero durante el ITI y no se encontraron diferencias entre grupos.

\section{Discussion}

SHR mostraron un índice PCA significativamente menor que Wistar.
Esto sugiere un déficit motivacional en SHR.

Se encontraron diferencias entre cepas solamente en el número de respuestas, pero no en la probabilidad o latencia a presionar la palanca.
En el comedero sí hubo diferencias en los tres aspectos del índice.
Este análisis podría ser relevante para estudios posteriores.

La baja atribución de saliencia incentiva es sorprendente si se considera que SHR muestra altos niveles de acción impulsiva medidos por DRL, y que se ha mostrado una correlación positiva entre ambas variables.
Esto sugiere que el mecanismo implicado en la acción impulsiva en SHR se relaciona con variables distintas de la atribución de saliencia incentiva.

El déficit motivacional mostrado modela bien lo encontrado en pacientes con ADHD y sugiere la necesidad de continuar la investigación con modelos animales para analizar el papel del constructo de la saliencia incentiva en la comprensión del ADHD.

Dado que SHR tiene disfunción dopaminérgica y dada la relación entre ese sistema y la atribución de saliencia, los resultados coinciden con la correlación entre el déficit motivacional en ADHD y el funcionamiento de las vías dopaminérgicas.

Los hallazgos sugieren una explicación para la correlación entre un diagnóstico de ADHD en la infancia y depresión más tarde en la vida---se ha visto que quienes sufren depresión tienen un {\itshape liking} consumatorio normal, pero poco placer anticipatorio, lo que se ha relacionado con déficit en la saliencia incentiva.

Los hallazgos podrían ser base para intervenciones que aminoren los déficit motivacionales que caracterizan a pacientes de ADHD.


\end{document}
