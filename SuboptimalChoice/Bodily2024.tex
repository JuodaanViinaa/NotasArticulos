\documentclass[a4paper,12pt]{article}
\usepackage[utf8]{inputenc}
\usepackage[T1]{fontenc}
\usepackage[spanish]{babel}
\spanishdecimal{.}
\usepackage{csquotes}
\usepackage{anysize}
\usepackage{graphicx}
\marginsize{25mm}{25mm}{25mm}{25mm}

\title{Narrative framing may increase human suboptimal choice behavior}
\author{Jessica Stagner Bodily \and Kent D. Bodily \and Robert A. Southern \and Erin E. Baum \and Vincent M. Edwards}
\date{2024}

\begin{document}
{\scshape\bfseries \maketitle}

Cuando la elección y el resultado están separados por otro estímulo la preferencia por la recompensa mayor se ve disminuida.

La elección subóptima de las palomas depende de que los estímulos de la alternativa subóptima predigan confiablemente las consecuencias.

Una suposición es que la preferencia no es por los estímulos predictivos, sino un rechazo por los estímulos no predictivos ({\slshape what the fuck}).
Zentall y Stagner evaluaron palomas en elección subóptima, pero manipularon la magnitud en lugar de la probabilidad de reforzamiento, y encontraron que aun en ausencia de incertidumbre en la alternativa de mayor magnitud, las palomas preferían la de menor magnitud.
También se ha encontrado elección subóptima en ratas ({\scshape what the fuck}) y en macacos rhesus.

En humanos se ha encontrado preferencia subóptima de alrededor del 30\%, pero resalta que los participantes elegían la alternativa subóptima ocasionalmente aun después de mucha exposición a las contingencias.
Es decir, la ejecución de los humanos no es óptima en el sentido de que no es perfecta, y eso coincide con la experiencia humana fuera del laboratorio.

Este estudio se enfoca en los efectos de los estímulos dentro de la tarea.
Anteriormente se han utilizado colores o formas arbitrarios, pero quizá su naturaleza abstracta resulta en elección menos que óptima ({\itshape ¿de dónde sale esa suposición?}).
Esto es apoyado por el {\itshape Jason Four-Card Problem}, en el que se pide a participantes determinar qué cartas (de cuatro disponibles) voltear para probar la oración de ``si una carta tiene un número par de un lado, tiene un parche rojo del otro''.
La estrategia óptima es tratar de desacreditar la afirmación, pero los participantes a menudo intentan confirmarla.
Se ha mostrado que al poner el problema en un contexto social los participantes suelen resolver el problema sin errores.

Así, se intuye que al poner elección subóptima con una narrativa realista la ejecución podría mejorar.
Sin embargo, si la ejecución depende solamente de las probabilidades de reforzamiento, entonces se esperarían resultados similares a los que ya se han encontrado antes.

Este estudio presentó a humanos una de cuatro versiones de la tarea: una réplica de la tarea estándar con estímulos de colores predictivos de reforzamiento, una con estímulos de color no predictivos, una con una narrativa realista y estímulos predictivos, y una con narrativa realista sin estímulos predictivos.

La recolección de datos se realizó en una sola sesión, lo que reduce la experiencia y cantidad de ensayos de elección, pero esto se mitigó reduciendo la duración usada para los estímulos del eslabón terminal ({\itshape what the fuck}): se ha encontrado que duraciones menores de TL llevan a ejecución óptima bajo ciertas condiciones.
Se utilizaron 2s de duración en este experimento, lo que permitía experimentar una mayor cantidad de ensayos y por lo tanto más oportunidades para demostrar preferencia.

\section{Method}

96 participantes menos 4 por sesgo de posición, todos de universidad y que recibieron créditos por su participación.

La duración de TL fue de 2 s para todas las condiciones.

Los participantes experimentaron sesiones individuales en habitaciones separadas con un investigador sentado fuera.
El reforzamiento se daba mediante la presentación de una imagen de una moneda con la leyenda ``coins received'' y un sonido de máquina registradora.

Había ensayos libres y forzados.
Cada sesión tenía 180 ensayos divididos en seis bloques de 20 ensayos forzados y 10 de elección.

Los participantes elegían con el teclado de una computadora.
En la condición abstracta elegían una figura; en la narrativa, un turno (día o noche en un restaurante) con la meta de obtener la mayor cantidad posible de monedas.

En las condiciones no predictivas cada alternativa llevaba a eslabones terminales no relacionados con la consecuencia final; en las predictivas los estímulos sí se relacionaban con la consecuencia final.

\section{Results}

Se encontró un efecto principal de la predicción de reforzamiento, pero no de la condición de narrativa ni la interacción.
Los participantes de la condición abstracta no predictiva mostraron una preferencia temprana por la alternativa óptima.

\section{Discussion}

Hacia el final de la sesión de prueba la preferencia en las condiciones abstracta predictiva, narrativa predictiva, y abstracta no predictiva fue similar a la de estudios previos.

Contrario a lo que sugieren estudios previos, los participantes de la condición narrativa predictiva no tenían mayor tendencia a la optimalidad que los demás, lo que implica que una narrativa realista no mejoró la ejecución en la tarea.

La elección de los participantes de la condición narrativa no predictiva no difirió del azar.
Es posible que la narrativa funcionara como un distractor.


\end{document}
