\documentclass[a4paper,12pt]{article}
\usepackage[utf8]{inputenc}
\usepackage[T1]{fontenc}
\usepackage[spanish]{babel}
\usepackage{csquotes}
\usepackage{anysize}
\usepackage{graphicx}
\marginsize{25mm}{25mm}{25mm}{25mm}

\title{Good news is better than bad news, but bad news is not worse than no news}
\author{Brittany Sears \and Roger M. Dunn \and Jeffrey M. Pisklak \and Marcia L. Spetch \and Margaret A. McDevitt}
\date{2022}

\begin{document}
{\scshape\bfseries \maketitle}

En elección subóptima se ha demostrado que la condición de señalización es determinante de la elección de las palomas: presentar estímulos no discriminativos en ambas alternativas produce elección óptima por parte de las palomas.
Investigación temprana mostró que variables temporales también modulan el grado de la preferencia.

Una de las primeras explicaciones para elección subóptima, el modelo de SiGN, proponía que el reforzamiento primario demorado y el reforzamiento condicionado inmediato compiten para determinar la elección. Se elige la alternativa subóptima cuando el reforzamiento primario está lo bastante demorado, lo que permite el control por el reforzamiento condicionado.


\end{document}
