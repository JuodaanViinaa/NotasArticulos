\documentclass[a4paper,12pt]{article}
\usepackage[utf8]{inputenc}
\usepackage[T1]{fontenc}
\usepackage[spanish]{babel}
\usepackage{csquotes}
\usepackage{anysize}
\marginsize{25mm}{25mm}{25mm}{25mm}

\title{Do we expect natural selection to produce rational behavior?}
\author{Alasdair I. Houston, John M. McNamara,\\*
	Mark D. Steer}
\date{2007}

\begin{document}

{\scshape\bfseries \maketitle}

Kacelnik (2006) hace una categorización sobre las definiciones de racionalidad según las disciplinas que las utilizan: filosofía y psicología (PP-rationality), economía (E-rationality), y ecología conductual/biología evolutiva (B-rationality).

En la psicología, el centro no es la racionalidad de los resultados, sino de los procesos y los mecanismos que les dan lugar, pero esta investigación es casi imposible dado que lidia con pensamientos privados de los humanos.

En contraste, la racionalidad de la economía esta dirigida a metas, no a procesos. El énfasis esta en la utilidad (financiera, alimento, o en la moneda de cambio que se elija). Pero aquí la dificultad está en definir cuál es la utilidad que se maximiza: un forrajeador puede no estar siendo irracional, sino maximizando una utilidad distinta de la que se esta analizando. Mientras algo, sin importar qué, se esté maximizando, se puede categorizar a un organismo como racional. Una debilidad adicional de la E-racionalidad esta en la violación relativamente frecuente de sus axiomas: transitividad e independencia.

Las diferencias entre PP- y E-racionalidad son directas: PP trata con creencias internas y no resultados, mientras que E hace lo contrario.

En cierto sentido, la B-racionalidad puede tomarse como un subset de la E-racionali\-dad, en tanto que define un resultado deseado, reemplazando la utilidad con la adecuación ({\slshape fitness}). La adecuación, a diferencia de la utilidad, puede medirse en términos de éxito reproductivo independientemente de las decisiones tomadas por el agente, mientras las funciones de utilidad solo pueden derivarse de las decisiones. El proceso por el que se llega al resultado no es de principal importancia en la B-racionalidad. Sin embargo, se asume que los agentes son producto de un proceso de selección natural que dio forma a sus mecanismos cognitivos y emocionales para maximizar su adecuación. Pero dado que la selección natural es ciega, no se puede esperar que lleve a conducta racional en todas las circunstancias, sino solo en aquellas que el agente encontraría en su entorno natural. Conducta irracional puede aparecer cuando los agentes son colocados en un contexto nuevo. En resumen, la B-racionalidad asume que los animales maximizarán su adecuación, de forma similar a cómo la E-racionalidad asume la maximización de una utilidad desconocida, pero solamente cuando son colocados en su contexto relevante.

Se asume que las preferencias de la E-racionalidad siguen ciertas condiciones que incluyen:

\begin{itemize}
	\item {\slshape Transitividad.} Las preferencias son jerárquicas, es decir, si $a > b$ y $b > c$, entonces $a > c$.
	\item {\slshape Independencia de alternativas irrelevantes.} La preferencia de una alternativa sobre otra prevalece a pesar de añadir o quitar otras alternativas.
\end{itemize}

Cuando estos supuestos se violan, la conducta se toma como irracional.

Los modelos de eleción pueden ser descriptivos (describen la conducta observada) o normativos (especifican qué conducta {\slshape debería}  ser observada con base en medidas como el dinero o el éxito reproductivo). 

{\scshape\bfseries Modelos descriptivos}

{\bfseries Igualación}

Para dos opciones, la tasa de respuestas que emite un agente igualará la tasa de recompensas que haya recibido previamente en las opciones:

$$\frac{B_1}{B_2}=\frac{R_1}{R_2}$$
$$\frac{B_1}{B_1+B_2}=\frac{R_1}{R_1+R_2}$$

Baum (1974) propuso una generalización de esta ley:
$$\frac{B_1}{B_2}=b\left(\frac{R_1}{R_2}\right)^s$$

donde $b$ y ]$s$ son parámetros ajustados llamados sensibilidad y sesgo. Sin embargo, Houston y cols. (1981, 1987) adiverten que deben tratarse solo como tales, pues su relevancia teórica no está muy clara y solo sirven ajustando {\slshape a posteriori}, no hacen predicciones. Aun así, la ley de igualación es ampliamente utilizada para describir la conducta.

La teoría de forrajeo óptimo es una aproximación normativa que busca explicar la conducta en términos de maximización de la adecuación. A menudo se asume que maximizar la tasa de ganancia energética maximiza la adecuación, y se ha discutido la relación de maximización de tasa con igualación. La primera no necesariamente resulta en la segunda.

Dada la multitud de asignaciones de conducta que pueden resultar en igualación, no se puede decir que la ley de igualación maximice la tasa de ganancia. Por ejemplo, al enfrentar un programa de VI con uno que tenga probabilidad constante de recompensa, la igualación resulta en una tasa de ganancias muy por debajo del óptimo. Se puede ver a esta conducta como un efecto de una regla de decisión que evolucionó en otras circunstancias.

\begin{bfseries}
	Si se desea entender la evolución de la conducta de forrajeo, se deberían buscar reglas de decisión que ejecuten bien en las situaciones que es probable que los animales experimenten. No se debe asumir que los animales tienen conocimiento completo de las condiciones del experimento actual. En su lugar, un animal aprende y explota el ambiente.
\end{bfseries}

Hay tres características de los ambientes de forrajeo que son relevantes:
\begin{enumerate}
	\item {\itshape Las recompensas pueden dar información de las recompensas futuras.}
	\item {\itshape El ambiente puede contener a otros forrajeadores.} Las reglas que se desempeñan bien en un ambiente solitario y uno conjunto pueden no ser las mismas.
	\item {\itshape El ambiente puede cambiar.} Se pueden evolucionar conductas óptimas fijas dados parmámetros ambientales fijos, pero un cambio hará que éstas se vuelvan subóptimas.
\end{enumerate}

Para lidiar con estos aspectos del mundo real, se necesita una aproximación con reglas que usen información obtenida de las recompensas para decidir entre las opciones. Un proceso así es el mejoramiento ({\itshape melioration}), según el cual los animales incrementan la alocación de conducta hacia la alternativa con la mayor tasa local.

{\bfseries La hipótesis de reducción de la demora}

Hace predicciones en procedimientos de cadenas concurrentes en los que, tras la elección, se entra en un eslabón terminal previo a la entrega de reforzamiento. Aquellas alternativas que señalen una mayor reducción con respecto al promedio de tiempo de espera previo a la entrega del siguiente reforzador serán prefereidas. Fantino y Dunn (1983) señalan que esta hipótesis predice la violación del principio de independencia de alternativas irrelevantes. Agregar una tercera opción puede cambiar la preferencia de las dos alternativas previas.

{\scshape\bfseries Modelos normativos}

Los hallazgos en los que los orgnaismos toman malas decisiones se han intentado explicar de varias formas:

{\bfseries El resultado es un efecto secundario}

Lo importante es cómo se desempeña la regla en términos de éxito reproductivo. 


\end{document}
