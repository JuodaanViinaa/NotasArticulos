\documentclass[a4paper,12pt]{article}
\usepackage[utf8]{inputenc}
\usepackage[T1]{fontenc}
\usepackage[spanish]{babel}
\spanishdecimal{.}
\usepackage{csquotes}
\usepackage{anysize}
\usepackage{graphicx}
\marginsize{25mm}{25mm}{25mm}{25mm}

\title{Time, uncertainty, and suboptimal choice}
\author{Alejandro Macías \and Valeria V. González \and Armando Machado \and Marco Vasconcelos}
\date{2024}

\begin{document}
{\scshape\bfseries \maketitle}

Una señal que quita ambigüedad sobre el resultado del forrajeo debería ser valiosa.
En elección subóptima es así hasta el punto que los animales renuncian a comida para recibir información.

La búsqueda de un mecanismo ha llevado a proponer que la elección subóptima es un efecto del valor de reforzamiento condicionado diferencial de los estímulos de los eslabones terminales: el estímulo que señala reforzamiento seguro adquiere propiedades reforzantes por su relación predictiva con la comida.
Sin embargo, no se ha resuelto por qué este efecto sería superior al efecto del estímulo inhibidor.
Aun así, se ha mostrado que la confiabilidad de la relación predictiva es crucial.

En este experimento se explora el efecto no de manipular las probabilidades anunciadas por los estímulos, sino de cambiar el tiempo bajo incertidumbre después de la elección.
Al elegir la opción informativa se presenta el estímulo $S^{G}$ durante los primeros $t$ segundos de la demora, seguido de lo cual el color puede mantenerse en $S^{G}$ o cambiar a $S^{R}$.
La incertidumbre se prolonga entonces durante $t$ segundos.
La evidencia indica que posponer la resolución de la ambigüedad debería disminuir la preferencia por la alternativa informativa.

En el segundo experimento se hizo la manipulación complementaria: ahora se encendía $S^{R}$ por default y cambiaba o se mantenía en su color tras $t$ segundos.

Se varió $t$ de 0 a 10 s.
Se espera que cuando $t = 0$ el resultado sea el de elección subóptima normal; y cuando $t = 10$ la preferencia se revierta.
Se predice que la preferencia por la opción informativa disminuirá monotónicamente con $t$.

\section{Experiment 1 - Temporarily reducing the reliability of $S^{G}$}

Todo ensayo informativo empezaba presentando $S^{G}$, que podía cambiar o mantenerse después de $t$ segundos.

Los sujetos fueron 7 palomas en cajas individuales.
Se utilizaron cajas operantes con un panel de respuesta con tres teclas circulares.
Bajo la tecla central había un dispensador de comida, y en la pared opuesta había una luz general.

\subsection{Pretraining}

Las palomas fueron entrenadas con programas de FR con los seis estímulos del experimento (rojo, verde, amarillo, azul, más y círculo).

\subsection{Training}

Cada paloma pasó por 9 longitudes de $t$, cada una de al menos 12 sesiones de duración y hasta alcanzar estabilidad.

Cada sesión consistía en 120 ensayos, 40 de elección y 80 forzados, con las condiciones normales de elección subóptima más la manipulación ya descrita.

Todas las palomas comenzaron con $t = 0$ (procedimiento estándar), seguido de $t = 5$.
Después, cuatro palomas pasaron por $t = 1.5$ \textrightarrow $t = 8.5$, mientras tres más pasaron por 


\end{document}
