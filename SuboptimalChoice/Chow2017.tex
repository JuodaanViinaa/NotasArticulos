\documentclass[a4paper,12pt]{article}
\usepackage[utf8]{inputenc}
\usepackage[T1]{fontenc}
\usepackage[spanish]{babel}
\usepackage{csquotes}
\usepackage{anysize}
\marginsize{25mm}{25mm}{25mm}{25mm}

\title{Suboptimal choice in rats: incentive salience attribution promotes maladaptive decision making}
\author{Jonathan J. Chow, Aaron P. Smith, A. George Wilson,\\* Thomas R. Zentall, Joshua S. Beckmann}
\date{2017}

\begin{document}

{\scshape\bfseries \maketitle}

Despite previous studies demonstrating that a terminal-link stimulus with the greatest predictive probability of reinforcement can produce suboptimal choice in pigeons, similar procedures applied in rodents fail to produce a similar effect; instead, rats tend to behave optimally and choose an alternative that provides the greatest amount of reinforcement, regardless of how predictive a terminal-link stimulus might be. While these findings suggest species differences between pigeons and rats on the suboptimal choice procedure, it is also possible that the specific conditions used within the procedures could greatly influence choice behavior. Conceptually, the stimulus that is present in each terminal-link functions as a conditioned stimulus (CS), as it is predictive of the subsequent reinforcer (unconditioned stimulus; US). However, there are scenarios in which a CS can be attributed with incentive value that goes beyond its predictive function. Importantly, CSs attributed with incentive value serve as more robust conditioned reinforcers. Thus, not al CS function equally despite being equally predictive.

For pigeons, light stimuli are known to elicit sign-tracking behavior (approach and contact with the stimulus, in the form of key pecks), which is often a key feature of stimuli attributed with incentive salience. Thus, for pigeons, it is possible that the use of a light stimulus, which elicits key pecking, could be coupled with incentive value attribution.

Studies using rats have also used lights. Notably, lights are known to elicit goal-tracking behavior, described as approach to the location of reward delivery, when a food US is used and is not accompanied by the attribution of incentive value that has been shown to promote suboptimal choice behavior. It has been shown that rats have a tendency to sign-track to a lever CS, and levers associated with sign-tracking behavior function as robust conditioned reinforcers.

Therefore, in the present study we examined how different terminal-link stimuli, with and without incentive salience (i.e., levers vs. lights), can influence decision-making in rats using a suboptimal choice procedure.

{\scshape\bfseries Procedure}

The authors followed a modified version of the suboptimal choice procedure in which, in the first condition, rats were exposed to equal reward probabilities on either alternative ($p={.}5$). This probability was lowered for the predictive alternative to 0.25 and 0.125 in two following phases. Importantly, they used only one stimulus per alternative (both lights or both levers for different groups), meaning that the conditioned inhibitor ($S^-$) was simply replaced by a blackout.

Discrimination proportions were calculated to ensure the presence of a robust discrimination.

Preference for the predictive alternaive increased with training during the first phase, and a main effect of stimulus was found, which indicates that levers increased the preference for the predictive alternative to a greater extent than lights. After transitioning to the second phase, in which $p={.}25$ for the predictive alternative, preference for the now suboptimal alternative decreased with training. There was still, though, a main effect of stimulus, indicating that levers increased the preference for the informative alternative to a greater extent than lights. Finally, when $p={.}125$ for the informative alternative, preference for it decreased even further. Still, there was a main effect of stimulus, indicating a greater preference for the informative alternative when using levers compared with lights.

These results suggest that predictive stimuli do influence choice in rats, but those imbued with incentive salience promote suboptimal choice  to a greater degree than those without it.

A non linear mixed effects model revealed that, for the lever group, choice was more strongly influenced by conditioned reinforcement than by primary reinforcement, which suggests that the value attributed to a lever drives initial link preference more than a food pellet. The opposite was true for the light group.

Sign-tracking responses were higher for the predictive lever than for the non-predictive lever. Goal tracking was higher for the light stimulus than for the lever stimulus, and was also higher for the predictive stimulus than for the non-predictive stimulus.

All those results indicate that a predictive stimulus will elicit high rates of conditioned responding to a greater extent than non-predictive stimuli. Also, levers primarily elicited sign-tracking, while lights elicited goal-tracking.

After the reversal, probability of reinforcement was switched back to $p={.}5$ and preference reverted back to favouring the informative alternative, which indicates that animals were sensitive to the reversal of contingencies, and there were no side biases.

{\scshape\bfseries Discussion}

Reward associated stiumuli with a high predictive utility (100\%) increase preference for an alternative, and if it is also imbued with incentive salience, its effect is even grater, being able to even promote suboptimal choice. This choice was resistant to decreases in expected value.

Predictive utility of reward associated stimuli does not seem to be enough to promote suboptimal behavior. Rather, incentive salience attribution is necessary to give it a boost.

These results oppose those of Trujano et al. Since no stimulus was used as a $S^-$ in both groups, conditioned inhibition must have been equal across them. Yet, suboptimal choice still emerged only for the lever group. This indicates that conditioned inhibition by itself cannot account for suboptimal behavior.

\end{document}
