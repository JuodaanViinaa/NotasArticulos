\documentclass[a4paper,12pt]{article}
\usepackage[utf8]{inputenc}
\usepackage[T1]{fontenc}
\usepackage[spanish]{babel}
\usepackage{csquotes}
\usepackage{anysize}
\usepackage{graphicx}
\marginsize{25mm}{25mm}{25mm}{25mm}

\title{Suboptimal choice: a review and quantification of the signal for good news (SiGN) model}
\author{Roger M. Dunn \and Jeffrey M. Pisklak \and Margaret A. McDevitt \and Marcia L. Spetch}
\date{2023}

\begin{document}
{\scshape\bfseries \maketitle}

Aunque la elección subóptima puede venir de procesos adaptativos en su entorno natural, disminuye la cantidad total de alimento obtenida.
Es inconsistente con la ley del efecto, aprendizaje por refuerzo, y teoría de forrajeo óptimo.
Pero el estudio de conducta aparentemente subóptima puede permitir entender las variables que controlan la conducta.

Los primeros reportes de conducta subóptima mostraron que se podía propiciar cuando el eslabón inicial requiere una sola respuesta, cuando hay gran demora entre la respuesta y la consecuencia (TL largos), y especialmente cuando la alternativa subóptima es informativa.

Un intento temprano para explicar la elección subóptima fue el modelo de Signal for Good News. Sus supuestos son que en programas de cadenas concurrentes probabilísticas la elección es dirigida por las diferencias en las tasas de reforzamiento primario (comida) y secundario (entrada en los eslabones terminales de la cadena); que un evento es un reforzador condicionado en función de la reducción de la demora a la comida;que los estímulos que predicen una reducción en la demora más allá por la señalada por la propia entrada al eslabón terminal son buenos reforzadores condicionados; y que las demoras señaladas a la omisión de comida tienen como única función crear un contexto de incertidumbre.
El modelo da cuenta cualitativamente de los efectos de los parámetros temporales de la tarea y la relevancia de las señales.

Otras explicaciones


\end{document}
