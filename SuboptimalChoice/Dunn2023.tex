\documentclass[a4paper,12pt]{article}
\usepackage[utf8]{inputenc}
\usepackage[T1]{fontenc}
\usepackage[spanish]{babel}
\usepackage{csquotes}
\usepackage{anysize}
\usepackage{graphicx}
\marginsize{25mm}{25mm}{25mm}{25mm}

\title{Suboptimal choice: a review and quantification of the signal for good news (SiGN) model}
\author{Roger M. Dunn \and Jeffrey M. Pisklak \and Margaret A. McDevitt \and Marcia L. Spetch}
\date{2023}

\begin{document}
{\scshape\bfseries \maketitle}

Aunque la elección subóptima puede venir de procesos adaptativos en su entorno natural, disminuye la cantidad total de alimento obtenida.
Es inconsistente con la ley del efecto, aprendizaje por refuerzo, y teoría de forrajeo óptimo.
Pero el estudio de conducta aparentemente subóptima puede permitir entender las variables que controlan la conducta.

Los primeros reportes de conducta subóptima mostraron que se podía propiciar cuando el eslabón inicial requiere una sola respuesta, cuando hay gran demora entre la respuesta y la consecuencia (TL largos), y especialmente cuando la alternativa subóptima es informativa.

Un intento temprano para explicar la elección subóptima fue el modelo de Signal for Good News. Sus supuestos son que en programas de cadenas concurrentes probabilísticas la elección es dirigida por las diferencias en las tasas de reforzamiento primario (comida) y secundario (entrada en los eslabones terminales de la cadena); que un evento es un reforzador condicionado en función de la reducción de la demora a la comida;que los estímulos que predicen una reducción en la demora más allá por la señalada por la propia entrada al eslabón terminal son buenos reforzadores condicionados; y que las demoras señaladas a la omisión de comida tienen como única función crear un contexto de incertidumbre.
El modelo da cuenta cualitativamente de los efectos de los parámetros temporales de la tarea y la relevancia de las señales.

Otras explicaciones han aparecido: decaimiento hiperbólico, contraste, consideraciones de forrajeo, información temporal, diferencias en probabilidades de reforzamiento en eslabones terminales, decaimiento de asociabilidad, persecución de errores de predicción anticipados y búsqueda de información, y mecanismos evolutivos que enfatizan las señales confiables de comida.
Algunos modelos han sido formalizados matemáticamente.
Curiosamente, SiGN no lo ha sido a pesar de ser de las explicaciones más antiguas. Aquí se intenta formalizar.

\section{Suboptimal concurrent-chains procedure and literature review}

\subsection{Signaled 50\% versus 100\% food outcomes}

Experimentos que comparan 50\% señalado contra 100\% suelen tener mucha variabilidad entre sujetos, lo que lleva a sospechar que las palomas son indiferentes y solo eligen debido a sesgos.
Aunque aun la indiferencia entre las dos alternativas es en sí misma subóptima, la diferencia entre indiferencia y preferencia por la alternativa subóptima tiene implicaciones importantes para el modelamiento.

Los hallazgos generales de experimentos con estas probabilidades, con una sola respuesta como requisito, y con longitudes iguales de eslabón terminal, indican que cuando las demoras son cortas (menos de 10s) las palomas tienden a preferir la alternativa óptima; pero al alargarlas incrementa la elección por la alternativa señalada.

\subsection{Signaled 20\% versus 50\% food outcomes}

Estos estudios tienen eslabones iniciales de FR 1; y terminales de FT 10s.
Se ha encontrado una preferencia consistente por la alternativa subóptima.

\subsection{Role of signals}

Correlacionar los estímulos terminales con la comida en la alternativa subóptima incrementa la preferencia de las palomas por ella.
Si no existe correlación, no se desarrolla la preferencia subóptima.

También importa la contigüidad entre la respuesta y la presentación de la señal: al introducir un {\itshape gap} de 5 s entre la respuesta y el inicio del estímulo disminuyó sustancialmente la preferencia por la alternativa subóptima.
Además, cuando el {\itshape gap} se inserta antes del encendido de la señal $S^{+}$ en la alternativa de 50\%, disminuyó en gran medida la elección subóptima, lo que no ocurrió si el {\itshape gap} se insertaba antes de la señal $S^{-}$, o antes de la señal de la alternativa de 100\%.
Esto sugiere que el encendido de la señal de comida en un contexto de incertidumbre es importante para la elección subóptima, pero la señal no discriminativa no lo es.

\subsection{Role of partial signals}

Aproximaciones recientes intentan determinar el efecto de alternativas parcialmente señaladas.
Fortes mostró que agregar entregas de comida después de algunas de las presentaciones de $S^{-}$ tenía poco efecto en la elección, pero parecía devaluar la alternativa subóptima al evaluar con un procedimiento de ajuste de demora.
González {\itshape et al.} manipularon la probabilidad de comida tras un estímulo específico manteniendo la probabilidad global de las alternativas constante, y encontraron que el grado en que los distintos estímulos predicen la comida se correlaciona directamente con la preferencia.
Sears {\itshape et al.} aisló los roles de $S^{+}$ y $S^{-}$ igualando la probabilidad de reforzamiento primario y proveyendo una única señal ``pura'' para algunas alternativas, {\itshape e.g.,} la alternativa de ``buenas noticias'' llevaba a una señal de comida segura o una señal incierta;'la alternativa de ``malas noticias'' llevaba a una señal de omisión segura o una señal incierta; la alternativa no señalada llevaba siempre a estímulos inciertos.
Las tres alternativas tenían la misma tasa de reforzamiento primario, pero la preferencia fue ``buenas noticias'' > ``malas noticias'' > ``no señalada''.

\section{The role of temporal variables}

Se ha manipulado la duración de los eslabones iniciales y terminales, y el intervalo entre ensayos.

\subsection{Terminal-Link duration and schedule}

Spetch {\itshape et al.} encontraron que eslabones terminales cortos (5 - 10s) llevaban a baja elección subóptima, pero eslabones largos la incrementaban.
McDevitt encontró mayor elección subóptima con eslabones de 20s que con eslabones de 5s.

Un problema con el estudio de eslabones terminales es que el requisito de respuestas varía entre estudios: usar TF es ideal dado que la duración de los eslabones es constante sin importar la conducta, pero algunos estudios usan IF, lo que extiende la demora.
Otros estudios han utilizado VI, lo que complica el modelamiento dado que se ha encontrado que los organismos no tratan a los programas VI y FI de la misma forma.

\subsection{Duration of the $S^{-}$ terminal link}

Manipular la duración de los eslabones no reforzados independientemente parece tener poco efecto en la elección subóptima y en la elección en general: valores extremos no parecen afectar las preferencias.
% Nota: ¿Podría manipular independientemente los eslabones terminales negativos en escape? Quizá haciendo eso evitaría que mis animales sin escape cambien su preferencia, pero los de escape sí lo podrían hacer.

\subsection{Initial-Link duration and schedule}

Suele haber mayor elección subóptima con programas largos de VI que con FR 1.

\subsection{Duration of the intertrial interval}

El ITI parece tener un efecto casi nulo en la elección: Spetch {\itshape et al.} (1990) no encontraron efectos con intervalos que variaron entre 0 y 40s.
% Esto puede funcionar a mi favor: parece que los ITI por sí mismos no afectan a la elección. Quizá con el escape sí lo hagan.

\section{Probability of food on each alternative}

Disminuir la probabilidad de comida debajo de 1.0 en la alternativa óptima lleva a mayor elección subóptima, suponiendo que sea no señalada.
Disminuir la probabilidad de la alternativa subóptima afecta poco la elección, mientras se mantenga por encima de .1.

\section{Probabilities of differing amounts}

Procedimiento de magnitudes: palomas prefieren la alternativa subóptima todavía.

\section{Exposure to the contingencies}

Algunos estudios usan criterios de estabilidad en lugar de números fijos de sesiones, lo que lleva a cantidades distintas entre sujetos.
Estas diferencias en el entrenamiento pueden llevar a diferencias en elección, dado que se ha encontrado que la suboptimalidad incrementa con el entrenamiento.

Existen diferencias entre estudios en la cantidad de ensayos forzados.
Estudios sin ensayos forzados tienen proporciones de elección relativamente bajas, y viceversa.
Belke y Spetch (1994) usaron un procedimiento en el que se repetía todo ensayo que no terminaba en comida, forzando a las palomas a permanecer en la alternativa subóptima hasta obtener comida.
Esto incrementó la preferencia por la alternativa subóptima.

\section{Other variables}

No parece importar que las alternativas no señaladas tengan un solo estímulo o dos.

Mayor privación de alimento y entornos pobres llevan a mayor elección subóptima.

La modalidad de la respuesta parece ser importante: se ha encontrado que las palomas tienen preferencia por la alternativa óptima cuando el operando de respuesta es un pedal.

\section{Nonavian species}

En ratas se han encontrado resultados contradictorios.
Chow sugiere que la luz y tono usados como estímulos no adquieren saliencia incentiva dado que evocan conducta de goal tracking más que sign tracking.
Zentall sugirió que la elección subóptima en ratas puede depender de su localización en su sistema conductual y si el estímulo evoca conducta de búsqueda general o focal.
Sugiere que los estímulos visuales evocan conducta focal en palomas, pero no en ratas.
Cunningham y Shahan sugieren que factores temporales importan, y encontraron más suboptimalidad con TL largos.
Martínez sugirió que la inhibición condicionada es importante, y encontró optimalidad cuando una palanca fue usada como $S^{-}$.
Alba encontró suboptimalidad con TL largos solo cuando los estímulos eran compuestos de tono más luz.

Las ratas son menos propensas que las palomas a la elección subóptima, pero su suboptimalidad parece incrementar con el TL.

\subsection{Humans}

Lalli (2000) evaluaron a niños con retraso en el desarrollo y encontraron resultados similares a los de palomas.
Molet encontró una preferencia modesta por la alternativa subóptima en jugadores.
McDevitt encontró preferencia óptima en un procedimiento en humanos con la versión de magnitudes, igual que Stagner (2020).

Hay evidencia de que los humanos eligen la información anticipada aun cuando no tiene efecto en la recompensa final, y la preferencia por la información incrementa con la demora entre la elección y la recompensa.
Algunos experimentos en humanos describen las contingencias de manera explícita, pero hay evidencia que indica que la elección es distinta cuando la contingencias se describan comparada con cuando se aprenden por experiencia.
Aun así, los efectos similares de la duración de la demora y el énfasis en las buenas noticias indican aspectos en común entre la elección subóptima animal y la búsqueda de información en humanos.

\section{Summary and implications}

En palomas y estorninos los estímulos predictivos refuerzan la elección cuando la entrega de comida es probabilística y demorada.
Las palomas eligen las alternativas señaladas incluso si su probabilidad de reforzamiento es sustancialmente menor que la probabilidad de la alternativa no señalada.
El nivel de preferencia depende en gran medida de aspectos temporales: eslabones terminales largos y eslabones iniciales cortos promueven la elección subóptima.
Ni los ITI ni la duración del $S^{-}$ afecta a la elección.

Las ratas tienen menos probabilidad que las palomas de elegir la alternativa subóptima, pero su elección incrementa en función del TL.

Una hipótesis es que la elección por alternativas señaladas es determinada por la reducción de la incertidumbre.
Evidencia en contra se encuentra en el hallazgo de que un $S^{-}$ no sostiene las respuestas de observación a pesar de reducir la incertidumbre.
Un experimento de Sears en el que una alternativa de buenas noticias fue preferida sobre una de malas noticias es también evidencia en contra.

Otra hipótesis es que solo el valor predictivo de $S^{+}$ determina la elección, y ésta no es afectada por su frecuencia o la frecuencia del reforzamiento primario.
Pero en condiciones de 50\% vs 100\% se prefiere la alternativa señalada de 50\% a pesar de que los estímulos no discriminativos sean igualmente predictivos de la demora a la comida.

La diferencia en probabilidades de reforzamiento por sí misma tampoco explica la elección, es decir, el incremento en probabilidad señalado por $S^{+}$ no parece ser el determinante de la preferencia.
El papel que tienen lo IL y TL indica que factores temporales también están implicados.

Las hipótesis de un solo mecanismo no parecen explicar la elección subóptima, por lo que muchas teorías apelan a más de un proceso.
El modelo SiGN asume que la elección es determinada por la interacción entre el reforzamiento primario y el condicionado provisto por la reducción de la demora.


\end{document}
